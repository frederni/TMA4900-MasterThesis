\chapter{Conclusion}

This thesis has explored the use of Gaussian models on binary phenotypes within the framework of the animal model and Bayesian statistics, directly obtaining observation scale heritability. The results, based on a dataset and simulation data, show that the Gaussian models can indeed be useful to easily obtain a heritability estimate, as a simpler and more direct alternative to binomial models with back-transformations. The results are limited to the datasets and models applied in the thesis and do not account for highly unbalanced phenotypes.

Furthermore, simulated data indicate that the introduction of overdispersion does not significantly challenge the Gaussian model. However, the heritability estimates from a Gaussian model diverge from the back-transformed probit model upon the introduction of fixed effects in the simulation study. Additionally, the simulation work has demonstrated that including the variance from a fixed effect in the denominator for heritability may be necessary to get an accurate heritability estimation in a Gaussian model. The models with the application data containing several fixed effects do not exhibit the same effect, and it seems like the Gaussian model provides estimates very close to the probit models' estimates. In general, using a Gaussian model also limits the use for further inference, as its general predictive abilities are significantly worse on binary phenotypes than a binomial model.

The findings show that a Gaussian mixed model is able to capture heritability on an observation scale that closely resembles the state-of-the-art back-transformation techniques for binomial models. Further investigation is required to determine exactly what effects decrease the power of a Gaussian model, although the results in this thesis indicate that it would be applicable in most practical cases. In the long term, the results can provide the foundation for developing a simpler, more effective, and easily interpretable statistical method for estimating additive genetic variance.