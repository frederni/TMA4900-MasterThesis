\chapter*{Sammendrag}

Innenfor kvantitativ genetikk har \textit{dyremodellen} blitt mye brukt til å modellere fenotypiske egenskaper, med både faste og tilfeldige effekter. Ved modellering av binære egenskaper brukes ofte en generell lineær blandet modell (GLMM) med en ikke-lineær lenkefunksjon, som transformerer varianskomponenter til en latent skala som er svært ulik den observerte skalaen.

Denne oppgaven undersøker først hvordan arvbarheten, betegnet som andelen av total varians som er forklart av genetiske faktorer, kan transformeres fra en underliggende skala til den observerte skalaen ved hjelp av terskelmodellen. Det er også ønskelig å finne ut om gaussisk modellering av binære fenotyper kan oppnå resultater sammenlignbart med en binomisk statistisk modell med tilbaketransformasjonsteknikker. Målet er å finne ut om gaussiske modeller kan være tilstrekkelige for å vurdere arvbarhet på en observasjonsskala.

Ved å bruke et bayesiansk statistisk rammeverk opprettes dyremodeller som gaussiske og binomiske med en probit lenkefunksjon, for et datasett med sangspurver og simulerte data, og beregner estimert arvbarhet i begge tilfellene. Den posteriore fordelingen av arvbarhet for den gaussiske modellen sammenlignes med binomialmodellen for å finne ut om en lineær modell kan være tilstrekkelig.

Resultatene viser for det første at man kan tilpasse en gaussisk modell på en binær fenotype og skalere den tilbake til den underliggende kontinuerlige skalaen. Reskaleringen håndterer modeller med faste effekter, men overestimerer arvbarhet i modeller med et svært ubalansert fenotypisk gjennomsnitt. I tillegg demonstreres det at gaussiske modeller oppnår posteriore fordelinger av arvelighet nær binomiske modeller tilbaketransformert til observasjonsskala. Fordelingene avviker imidlertid mer ved å introduksjon av overdispersjon eller faste effekter i kombinasjon med stor additiv genetisk varians.

Samlet sett viser funnene at det kan være aktuelt å bruke gaussisk modell i stedet for en binomisk modell, i sammenheng med den additive genetiske variansen i en dyremodell, dog avhengig av nøyaktigheten som kreves. Under visse omstendigheter, for eksempel én enkelt dominerende fast effekt, oppnår den gaussiske modellen dårligere resultater enn en binomisk modell, og krever i tillegg å inkludere varianskomponenten fra de faste effektene i beregningen av arvbarhet. Disse begrensningene ville derimot ikke påvirket arvbarheten i de fleste praktiske tilfeller, noe som indikerer at enklere, lineære og mer tolkbare modeller kan gi lovende estimater av arvbarhet innen kvantitativ genetikk.