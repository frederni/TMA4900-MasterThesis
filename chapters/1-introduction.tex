\chapter{Introduction}
\label{chap:introduction}

Quantitative genetics is a subfield of genetics and biological research that relates statistical methods and models to genetic and biological observations. In particular, quantitative genetics is useful for inference to understand how different genetic components and environmental factors contribute to individual traits such as height, probability of disease, or overall fitness. The field has applications in scientific areas such as biodiversity research, agriculture, and medicine \autocite{medicine2003, agriculture2016, reid2021}. Rather than studying alleles at any specific locus in the genome, quantitative genetics uses overall summaries of the individually minor effects of alleles at many different loci (location/position of a particular gene), making it viable for statistical analysis \autocite{aase2021}. 

In the context of quantitative genetics, the response variable in a statistical model is often the phenotypic trait, which is an observable characteristic of an individual within the studied population, either directly or indirectly.  Examples of phenotypic traits can be categorical, such as hair color in humans, continuous, such as height, or binary, such as juvenile survival (i.e., an individual surviving to a defined adulthood threshold in a natural habitat).
We can differentiate between the phenotype and the genotype, where the latter is the complete genetic composition of an individual \autocite{baker2008molecular}. Traditionally, the distinction between phenotype and genotype is that the appearance (physical structure) of an individual is its phenotype \autocite{genotype-def}. However, it is important to reiterate that a phenotypic trait is not necessarily directly observable, but can be an aggregation of observations or otherwise implicitly inferred from observations of an individual.

A phenotypic value of an individual is determined by the individual's genotype and the environment (defined as nongenetic factors), expressed as $P=G+E$ \autocite{primer-ecological-genetics}. In other words, we consider an additive partitioning of the phenotypic trait into genetic components ($G$) and environmental components ($E$). By partitioning the genetic and environmental components of a phenotypic trait, we can alongside genetic data from, e.g., pedigrees, estimate the additive genetic variance. For a given allele, we define additive genetic variance as the deviation between the phenotypic mean from inheritance, and the allele's relative effect on a phenotype \autocite{vadef}. In cases with a large additive genetic variance, the rate of evolutionary selection becomes large, allowing the population to adapt faster to new factors such as environmental change or the emergence of new natural predators.

The heritability of a trait is another important parameter in quantitative genetics, defined as the proportion between additive genetic variance and total phenotypic variance \autocite{falconer1996introduction}. Therefore, heritability is a standardization of additive genetic variance that allows analysis and comparison between studies and populations.
%As such, heritability can be interpreted as the proportion of a phenotype which \textit{cannot} be explained by environmental factors.
Understanding heritability is important to predict the response of populations to selection.
However, there are apparent disadvantages to using heritability. Within wild population studies, environmental factors are not controlled, making it difficult to distinguish between environmental and genetic factors, and thus introducing bias to the heritability estimate. Furthermore, in human behavioral genetics, heritability can be misleading due to the multicausal nature of human traits \autocite{moore2017heritability}. Also note that additive genetic variance is relative to total phenotypic variance and is only concerned with the extent to which individuals differ in terms of their genetic makeup within the studied group, and not the individuals themselves \autocite{gazzaniga2010}. An alternative metric for a trait's ability to evolve is evolvability, where the additive genetic variance is divided by the square of the phenotypic mean of the trait \autocite{hansen2011}. Although such alternative metrics exist, we will limit ourselves to estimating the heritability of a trait for the purposes of this thesis.

In general, modeling biological traits in nature can become a complex task due to the multitude of genetic and environmental causes that can influence a trait. Historically, researchers have used continuous Gaussian traits to model complex processes \autocite{century_after}. The \textit{animal model} is a mixed effects model using measures of relatedness in the population using, e.g., pedigree data \autocite{kruuk2004}.
While Gaussian models have been useful in many cases, for instance, with continuous response types, this paradigm may not be able to capture the complex biological processes contributing to a phenotype, for example in the case of binary traits.

Another approach is to use binomial regression within the framework of the animal model. %, which introduces non-linearity that can obscure the interpretability of factors of interest
In a linear model, the variance components estimated are on the same scale as the response, which we call the observation scale. In binomial models, this is no longer the case due to the nonlinear relationship between the response and the fitted values, and consequently the variance components attain a \textit{latent scale}, not directly related to the biological observation-level response. Recently, methods have been used to scale the latent variance back on the observation scale \autocite{de2016general}. With some closed-form exceptions, the method requires a series of numerical integrations, considerably increasing the time complexity. Therefore, our research question is whether a linear model could instead estimate the heritability of a binary trait. Although a linear model fitted onto a binary response will likely violate the model assumptions, our aim is to investigate if the violations lead to significant bias in the heritability estimate.

The goal of this thesis consists of two parts. The first part will investigate how one can use a Gaussian model to obtain estimates without nonlinear transformations. The second part will compare the new method to other modern techniques for back-transformation. Overall, the goal is to explore the degree to which we can approximate heritability by fitting a linear mixed model (LMM) instead of a generalized linear mixed model (GLMM) for binary traits. By comparing the performance of these types of models, we aim to shed light on whether a simpler approach may be sufficient for modeling heritability in certain cases. Ultimately, this research could contribute to a better understanding of the complex interplay between genetic and environmental factors in the shaping of biological traits and provide insights into the appropriate modeling techniques to study heritability. If using a linear model yields viable results, we would not need to fit generalized linear mixed models requiring back-transformations, effectively simplifying the statistical work for biological research in the context of heritability.

To achieve these goals, we performed statistical modeling on two datasets measuring a binary trait. The first dataset is from simulated data and the other dataset measures the juvenile survival of song sparrows (\textit{Melospiza melodia}) living on Mandarte Island \autocite{smith2006}. The model specifications in the song sparrow data are inspired by previous work \autocite{reid2021, rekkebo2021} on the same dataset, where logistic or probit regression models were used to analyze binary traits. Furthermore, we use a Bayesian statistical framework based on integrated nested Laplace approximations (INLA) to fit the different models and evaluate the deviation between the linear and binomial models. Since the variance and hence heritability obtain a latent scale, the thesis also reports on the different methods to transform the latent heritability back into an interpretable scale to be comparable to the Gaussian model.