\chapter*{Abstract}

In the field of quantitative genetics, the \textit{animal model} has been widely used to model phenotypic traits, incorporating fixed and random effects. When modeling binary traits, a general linear mixed model (GLMM) with a nonlinear link function is often employed,  transforming variance components onto a latent scale that differs from the observed scale.

In this thesis, we first examine how heritability, denoted as the proportion of variance in a trait explained by genetic factors, can be translated from an underlying scale to the observed scale using the threshold model. We also aim to determine whether fitting binary features to a Gaussian model can achieve results similar to a binomial model with more complex back-transformation techniques. The aim is to establish whether Gaussian models can be sufficient when assessing observation-scale heritability.

Using a Bayesian statistical framework, we fit animal models as Gaussian and binomial with a probit link function for a dataset of song sparrows and simulated data, calculating estimated heritabilities for both datasets. We compare the posterior density between the Gaussian and binomial models to determine if a simpler linear model may be sufficient.

The results demonstrate that one can fit a Gaussian model on a binary trait and re-scale it back to the underlying liability scale. The rescaling handles models with fixed effects, but it is prone to overestimation in the presence of highly unbalanced traits. We also demonstrate that Gaussian models obtain posterior distributions of heritability close to a binomial model's heritability back-transformed to the observation scales. However, the distributions deviate more by introducing fixed effects whose variance contributes significantly to the total variance.

Overall, the findings show that using a linear model rather than a binomial model to assess additive genetic variance in an animal model for binary data can be viable depending on the accuracy required. Under certain circumstances, such as a dominating fixed effect, the Gaussian model performs worse than a binomial model and requires including variance from fixed effects in the heritability computations. However, these constraints would not affect performance in most practical cases, indicating that simpler and more interpretable approaches can give valid estimates of heritability.