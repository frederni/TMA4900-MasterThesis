\hypertarget{r-code-acknowledgements}{%
\section*{R code acknowledgements}\label{r-code-acknowledgements}}

The application dataset and a large portion of data preprocessing is
provided by Jane Reid. The re-implementation into INLA is also largely
based on the work from Stefanie Muff.

\hypertarget{data-loading}{%
\section*{Data loading}\label{data-loading}}

We start by installing and importing required packages.

\begin{Shaded}
\begin{Highlighting}[]
\NormalTok{req.packages }\OtherTok{\textless{}{-}} \FunctionTok{c}\NormalTok{(}\StringTok{"BiocManager"}\NormalTok{, }\StringTok{"ggplot2"}\NormalTok{, }\StringTok{"latex2exp"}\NormalTok{, }\StringTok{"nadiv"}\NormalTok{,}
                  \StringTok{"QGglmm"}\NormalTok{, }\StringTok{"cowplot"}\NormalTok{, }\StringTok{"reshape2"}\NormalTok{, }\StringTok{"showtext"}\NormalTok{)}
\NormalTok{to.install }\OtherTok{\textless{}{-}}\NormalTok{ req.packages[}
  \FunctionTok{is.na}\NormalTok{(}\FunctionTok{match}\NormalTok{(req.packages, }\FunctionTok{installed.packages}\NormalTok{()[,}\DecValTok{1}\NormalTok{]))}
\NormalTok{  ]}
\ControlFlowTok{if}\NormalTok{(}\FunctionTok{length}\NormalTok{(to.install) }\SpecialCharTok{\textgreater{}}\NormalTok{ 0L)\{}
  \FunctionTok{install.packages}\NormalTok{(to.install)}
\NormalTok{\}}
\ControlFlowTok{for}\NormalTok{(pack }\ControlFlowTok{in}\NormalTok{ req.packages)\{}
  \FunctionTok{suppressPackageStartupMessages}\NormalTok{(}
    \FunctionTok{library}\NormalTok{(pack,}\AttributeTok{character.only =} \ConstantTok{TRUE}\NormalTok{, }\AttributeTok{quietly =} \ConstantTok{TRUE}\NormalTok{))}
\NormalTok{\}}

\ControlFlowTok{if}\NormalTok{ (}\SpecialCharTok{!}\FunctionTok{require}\NormalTok{(}\StringTok{"GeneticsPed"}\NormalTok{, }\AttributeTok{quietly =} \ConstantTok{TRUE}\NormalTok{)) \{}
\NormalTok{  BiocManager}\SpecialCharTok{::}\FunctionTok{install}\NormalTok{(}\StringTok{"GeneticsPed"}\NormalTok{)}
\NormalTok{\}}
\ControlFlowTok{if}\NormalTok{ (}\SpecialCharTok{!}\FunctionTok{require}\NormalTok{(}\StringTok{"MCMCglmm"}\NormalTok{, }\AttributeTok{quietly =} \ConstantTok{TRUE}\NormalTok{)) \{}
  \CommentTok{\# rbv patch}
  \FunctionTok{install.packages}\NormalTok{(}\StringTok{"../MCMCglmm{-}rbv{-}patch.tar.gz"}\NormalTok{)}
\NormalTok{\}}
\ControlFlowTok{if}\NormalTok{ (}\SpecialCharTok{!}\FunctionTok{require}\NormalTok{(}\StringTok{"INLA"}\NormalTok{, }\AttributeTok{quietly =} \ConstantTok{TRUE}\NormalTok{)) \{}
  \FunctionTok{install.packages}\NormalTok{(}
    \StringTok{"INLA"}\NormalTok{, }\AttributeTok{repos=}\FunctionTok{c}\NormalTok{(}\FunctionTok{getOption}\NormalTok{(}\StringTok{"repos"}\NormalTok{),}
                    \AttributeTok{INLA=}\StringTok{"https://inla.r{-}inla{-}download.org/R/stable"}\NormalTok{),}
    \AttributeTok{dep=}\ConstantTok{TRUE}\NormalTok{)}

\NormalTok{\}}
\FunctionTok{library}\NormalTok{(MCMCglmm)}
\FunctionTok{library}\NormalTok{(MASS)}
\FunctionTok{library}\NormalTok{(bdsmatrix)}
\FunctionTok{library}\NormalTok{(INLA)}

\FunctionTok{library}\NormalTok{(GeneticsPed) }\CommentTok{\# }

\CommentTok{\# Plotting libraries and settings}

\FunctionTok{library}\NormalTok{(grid)}
\NormalTok{texfont }\OtherTok{\textless{}{-}} \StringTok{"CMU Serif"}
\FunctionTok{showtext\_auto}\NormalTok{()}
\FunctionTok{font.paths}\NormalTok{(}\FunctionTok{file.path}\NormalTok{(}
  \FunctionTok{Sys.getenv}\NormalTok{(}\StringTok{"LOCALAPPDATA"}\NormalTok{), }\StringTok{"Microsoft"}\NormalTok{,}
  \StringTok{"Windows"}\NormalTok{, }\StringTok{"Fonts"}
\NormalTok{))}
\FunctionTok{font\_add}\NormalTok{(texfont, }\AttributeTok{regular =} \StringTok{"cmunrm.ttf"}\NormalTok{)}
\FunctionTok{theme\_set}\NormalTok{(}\FunctionTok{theme\_bw}\NormalTok{() }\SpecialCharTok{+} \FunctionTok{theme}\NormalTok{(}\AttributeTok{text =} \FunctionTok{element\_text}\NormalTok{(}\AttributeTok{family =}\NormalTok{ texfont,}
                                                 \AttributeTok{size =} \DecValTok{20}\NormalTok{)))}

\CommentTok{\# Dataset import}
\NormalTok{qg.data.gg.inds }\OtherTok{\textless{}{-}} \FunctionTok{read.table}\NormalTok{(}\StringTok{"../data/qg.data.gg.inds.steffi.txt"}\NormalTok{,}
  \AttributeTok{header =} \ConstantTok{TRUE}
\NormalTok{)}
\NormalTok{d.ped }\OtherTok{\textless{}{-}} \FunctionTok{read.table}\NormalTok{(}\StringTok{"../data/ped.prune.inds.steffi.txt"}\NormalTok{,}
  \AttributeTok{header =} \ConstantTok{TRUE}
\NormalTok{)}
\NormalTok{d.Q }\OtherTok{\textless{}{-}} \FunctionTok{read.table}\NormalTok{(}\StringTok{"../data/Q.data.steffi.txt"}\NormalTok{, }\AttributeTok{header =} \ConstantTok{TRUE}\NormalTok{)}

\NormalTok{qg.data.gg.inds}\SpecialCharTok{$}\NormalTok{natalyr.id }\OtherTok{\textless{}{-}}\NormalTok{ qg.data.gg.inds}\SpecialCharTok{$}\NormalTok{natalyr.no}
\end{Highlighting}
\end{Shaded}

Some global settings:

\begin{Shaded}
\begin{Highlighting}[]
\NormalTok{SAVE.PLOT }\OtherTok{\textless{}{-}} \ConstantTok{TRUE}
\NormalTok{n.samples }\OtherTok{\textless{}{-}} \DecValTok{10000}
\CommentTok{\# iid N(0,1) noise in songsparrow formula:}
\NormalTok{FORMULA.EXTRA.IID.NOISE }\OtherTok{\textless{}{-}} \ConstantTok{FALSE} 
\end{Highlighting}
\end{Shaded}

Below we do a couple more preprocessing steps, namely:

\begin{itemize}
\tightlist
\item
  build a pedigree structure from the denormalized table with
  \texttt{prepPed}.
\item
  assign new IDs to each individual starting at \(1\).
\item
  keep a mapping record between the original IDs (ninecode) and the new
  \(1\)-indexed IDs.
\item
  use this mapping to transform the IDs for each individual's Dam and
  Sire to the same.
\item
  compute the inverse of the relatedness matrix.
\end{itemize}

\begin{Shaded}
\begin{Highlighting}[]
\CommentTok{\# Scale the continuous variances for stability}
\NormalTok{qg.data.gg.inds}\SpecialCharTok{$}\NormalTok{f.coef.sc }\OtherTok{\textless{}{-}} \FunctionTok{scale}\NormalTok{(qg.data.gg.inds}\SpecialCharTok{$}\NormalTok{f.coef,}
                                   \AttributeTok{scale =} \ConstantTok{FALSE}\NormalTok{)}
\NormalTok{qg.data.gg.inds}\SpecialCharTok{$}\NormalTok{g1.sc }\OtherTok{\textless{}{-}} \FunctionTok{scale}\NormalTok{(qg.data.gg.inds}\SpecialCharTok{$}\NormalTok{g1,}
                               \AttributeTok{scale =} \ConstantTok{FALSE}\NormalTok{)}
\NormalTok{qg.data.gg.inds}\SpecialCharTok{$}\NormalTok{natalyr.no.sc }\OtherTok{\textless{}{-}} \FunctionTok{scale}\NormalTok{(qg.data.gg.inds}\SpecialCharTok{$}\NormalTok{natalyr.no,}
                                       \AttributeTok{scale =} \ConstantTok{FALSE}\NormalTok{)}
\NormalTok{qg.data.gg.inds}\SpecialCharTok{$}\NormalTok{brood.date.sc }\OtherTok{\textless{}{-}} \FunctionTok{scale}\NormalTok{(qg.data.gg.inds}\SpecialCharTok{$}\NormalTok{brood.date,}
                                       \AttributeTok{scale =} \ConstantTok{FALSE}\NormalTok{)}

\CommentTok{\# Binarize \textasciigrave{}sex\textasciigrave{} covariate}
\NormalTok{qg.data.gg.inds}\SpecialCharTok{$}\NormalTok{sex }\OtherTok{\textless{}{-}}\NormalTok{ qg.data.gg.inds}\SpecialCharTok{$}\NormalTok{sex.use.x1 }\SpecialCharTok{{-}} \DecValTok{1}
\end{Highlighting}
\end{Shaded}

\hypertarget{deriving-a}{%
\subsection*{\texorpdfstring{Deriving
\emph{A}}{Deriving A}}\label{deriving-a}}

For INLA we need IDs that run from 1 to the number of individuals

\begin{Shaded}
\begin{Highlighting}[]
\NormalTok{d.ped }\OtherTok{\textless{}{-}}\NormalTok{ nadiv}\SpecialCharTok{::}\FunctionTok{prepPed}\NormalTok{(d.ped)}
\NormalTok{d.ped}\SpecialCharTok{$}\NormalTok{id }\OtherTok{\textless{}{-}} \FunctionTok{seq\_len}\NormalTok{(}\FunctionTok{nrow}\NormalTok{(d.ped))}

\CommentTok{\# Maps to keep track of the Ninecode to ID relations}
\NormalTok{d.map }\OtherTok{\textless{}{-}}\NormalTok{ d.ped[, }\FunctionTok{c}\NormalTok{(}\StringTok{"ninecode"}\NormalTok{, }\StringTok{"id"}\NormalTok{)]}
\NormalTok{d.map}\SpecialCharTok{$}\NormalTok{g1 }\OtherTok{\textless{}{-}}\NormalTok{ d.Q[}\FunctionTok{match}\NormalTok{(d.map}\SpecialCharTok{$}\NormalTok{ninecode, d.Q}\SpecialCharTok{$}\NormalTok{ninecode), }\StringTok{"g1"}\NormalTok{]}
\NormalTok{d.map}\SpecialCharTok{$}\NormalTok{foc0 }\OtherTok{\textless{}{-}}\NormalTok{ d.Q[}\FunctionTok{match}\NormalTok{(d.map}\SpecialCharTok{$}\NormalTok{ninecode, d.Q}\SpecialCharTok{$}\NormalTok{ninecode), }\StringTok{"foc0"}\NormalTok{]}

\CommentTok{\# Give mother and father the id}
\NormalTok{d.ped}\SpecialCharTok{$}\NormalTok{mother.id }\OtherTok{\textless{}{-}}\NormalTok{ d.map[}\FunctionTok{match}\NormalTok{(d.ped}\SpecialCharTok{$}\NormalTok{gendam, d.map}\SpecialCharTok{$}\NormalTok{ninecode), }\StringTok{"id"}\NormalTok{]}
\NormalTok{d.ped}\SpecialCharTok{$}\NormalTok{father.id }\OtherTok{\textless{}{-}}\NormalTok{ d.map[}\FunctionTok{match}\NormalTok{(d.ped}\SpecialCharTok{$}\NormalTok{gensire, d.map}\SpecialCharTok{$}\NormalTok{ninecode), }\StringTok{"id"}\NormalTok{]}

\CommentTok{\# A can finally be constructed using \textasciigrave{}nadiv\textasciigrave{}}
\NormalTok{Cmatrix }\OtherTok{\textless{}{-}}\NormalTok{ nadiv}\SpecialCharTok{::}\FunctionTok{makeAinv}\NormalTok{(}
\NormalTok{  d.ped[, }\FunctionTok{c}\NormalTok{(}\StringTok{"id"}\NormalTok{, }\StringTok{"mother.id"}\NormalTok{, }\StringTok{"father.id"}\NormalTok{)])}\SpecialCharTok{$}\NormalTok{Ainv}

\CommentTok{\# Stores ID twice (to allow for extra IID random effect)}

\NormalTok{qg.data.gg.inds}\SpecialCharTok{$}\NormalTok{id }\OtherTok{\textless{}{-}}\NormalTok{ d.map[}
  \FunctionTok{match}\NormalTok{(qg.data.gg.inds}\SpecialCharTok{$}\NormalTok{ninecode, d.map}\SpecialCharTok{$}\NormalTok{ninecode),}
  \StringTok{"id"}
\NormalTok{]}
\NormalTok{qg.data.gg.inds}\SpecialCharTok{$}\NormalTok{u }\OtherTok{\textless{}{-}} \FunctionTok{seq\_len}\NormalTok{(}\FunctionTok{nrow}\NormalTok{(qg.data.gg.inds))}
\end{Highlighting}
\end{Shaded}

\hypertarget{inla}{%
\section*{INLA}\label{inla}}

The general INLA formula is provided below, where \texttt{f()} encode
the random effect:

\begin{Shaded}
\begin{Highlighting}[]
\NormalTok{formula.inla.scaled }\OtherTok{\textless{}{-}}\NormalTok{ surv.ind.to.ad }\SpecialCharTok{\textasciitilde{}}\NormalTok{ f.coef.sc }\SpecialCharTok{+}\NormalTok{ g1.sc }\SpecialCharTok{+} 
\NormalTok{  natalyr.no.sc }\SpecialCharTok{+}\NormalTok{ brood.date.sc }\SpecialCharTok{+}\NormalTok{ sex }\SpecialCharTok{+}
  \FunctionTok{f}\NormalTok{(nestrec, }\AttributeTok{model =} \StringTok{"iid"}\NormalTok{, }\AttributeTok{hyper =} \FunctionTok{list}\NormalTok{(}
    \AttributeTok{prec =} \FunctionTok{list}\NormalTok{(}
      \AttributeTok{initial =} \FunctionTok{log}\NormalTok{(}\DecValTok{1} \SpecialCharTok{/} \FloatTok{0.05}\NormalTok{),}
      \AttributeTok{prior =} \StringTok{"pc.prec"}\NormalTok{,}
      \AttributeTok{param =} \FunctionTok{c}\NormalTok{(}\DecValTok{1}\NormalTok{, }\FloatTok{0.05}\NormalTok{)}
\NormalTok{    ) }\CommentTok{\# PC priors}
\NormalTok{  )) }\SpecialCharTok{+}
  \FunctionTok{f}\NormalTok{(natalyr.id, }\AttributeTok{model =} \StringTok{"iid"}\NormalTok{, }\AttributeTok{hyper =} \FunctionTok{list}\NormalTok{(}
    \AttributeTok{prec =} \FunctionTok{list}\NormalTok{(}
      \AttributeTok{initial =} \FunctionTok{log}\NormalTok{(}\DecValTok{1} \SpecialCharTok{/} \FloatTok{0.25}\NormalTok{),}
      \AttributeTok{prior =} \StringTok{"pc.prec"}\NormalTok{,}
      \AttributeTok{param =} \FunctionTok{c}\NormalTok{(}\DecValTok{1}\NormalTok{, }\FloatTok{0.05}\NormalTok{)}
\NormalTok{    ) }\CommentTok{\# PC priors}
\NormalTok{  )) }\SpecialCharTok{+}
  \FunctionTok{f}\NormalTok{(id,}
    \AttributeTok{model =} \StringTok{"generic0"}\NormalTok{, }\CommentTok{\# Here we need to specify the covariance matrix}
    \AttributeTok{Cmatrix =}\NormalTok{ Cmatrix, }\CommentTok{\#    via the inverse (Cmatrix)}
    \AttributeTok{constr =} \ConstantTok{FALSE}\NormalTok{,}
    \AttributeTok{hyper =} \FunctionTok{list}\NormalTok{(}
      \AttributeTok{prec =} \FunctionTok{list}\NormalTok{(}\AttributeTok{initial =} \FunctionTok{log}\NormalTok{(}\DecValTok{1} \SpecialCharTok{/} \DecValTok{10}\NormalTok{), }\AttributeTok{prior =} \StringTok{"pc.prec"}\NormalTok{,}
                  \AttributeTok{param =} \FunctionTok{c}\NormalTok{(}\DecValTok{1}\NormalTok{, }\FloatTok{0.05}\NormalTok{))}
\NormalTok{    ) }\CommentTok{\# PC priors}
\NormalTok{  )}
\ControlFlowTok{if}\NormalTok{ (FORMULA.EXTRA.IID.NOISE) \{}
\NormalTok{  formula.inla.scaled }\OtherTok{\textless{}{-}} \FunctionTok{update}\NormalTok{(}
\NormalTok{    formula.inla.scaled,}
    \SpecialCharTok{\textasciitilde{}}\NormalTok{ . }\SpecialCharTok{+} \FunctionTok{f}\NormalTok{(u,}
      \AttributeTok{model =} \StringTok{"iid"}\NormalTok{, }\AttributeTok{constr =} \ConstantTok{TRUE}\NormalTok{,}
      \AttributeTok{hyper =} \FunctionTok{list}\NormalTok{(}\AttributeTok{prec =} \FunctionTok{list}\NormalTok{(}
        \AttributeTok{initial =} \FunctionTok{log}\NormalTok{(}\DecValTok{1}\NormalTok{),}
        \AttributeTok{fixed =} \ConstantTok{TRUE}
\NormalTok{      ))}
\NormalTok{    )}
\NormalTok{  )}
\NormalTok{\}}
\end{Highlighting}
\end{Shaded}

Now we call INLA models. Note that we pass some control arguments to the
function call. We compute DIC (Deviance information criterion) for all
models with \texttt{dic} flag in \texttt{control.compute}. For the
binomial models, we want to be able to use \texttt{QGglmm} and average
over all fixed effects. This is done by supplying the \emph{latent
marginal predicted values}, which are not computed unless you pass the
\texttt{return.marginals.predictor} flag set to true. We also want to
set the \texttt{CPO} flag to true in the Gaussian model, so we can look
a bit at its ``residuals'' (PIT values). Lastly, the
\texttt{control.family} argument is used to pass the link functions for
binomial models.

\begin{Shaded}
\begin{Highlighting}[]
\NormalTok{fit.inla.probit }\OtherTok{\textless{}{-}} \FunctionTok{inla}\NormalTok{(}
  \AttributeTok{formula =}\NormalTok{ formula.inla.scaled, }\AttributeTok{family =} \StringTok{"binomial"}\NormalTok{,}
  \AttributeTok{data =}\NormalTok{ qg.data.gg.inds,}
  \AttributeTok{control.compute =} \FunctionTok{list}\NormalTok{(}\AttributeTok{dic =} \ConstantTok{TRUE}\NormalTok{,}
                         \AttributeTok{return.marginals.predictor =} \ConstantTok{TRUE}\NormalTok{),}
  \AttributeTok{control.family =} \FunctionTok{list}\NormalTok{(}\AttributeTok{link =} \StringTok{"probit"}\NormalTok{),}
\NormalTok{)}


\NormalTok{fit.inla.gaussian }\OtherTok{\textless{}{-}} \FunctionTok{inla}\NormalTok{(}
  \AttributeTok{formula =}\NormalTok{ formula.inla.scaled, }\AttributeTok{family =} \StringTok{"gaussian"}\NormalTok{,}
  \AttributeTok{data =}\NormalTok{ qg.data.gg.inds,}
  \AttributeTok{control.compute =} \FunctionTok{list}\NormalTok{(}\AttributeTok{dic =} \ConstantTok{TRUE}\NormalTok{, }\AttributeTok{cpo =} \ConstantTok{TRUE}\NormalTok{)}
\NormalTok{)}

\FunctionTok{data.frame}\NormalTok{(}
  \AttributeTok{Gaussian =}\NormalTok{ fit.inla.gaussian}\SpecialCharTok{$}\NormalTok{dic}\SpecialCharTok{$}\NormalTok{dic,}
  \AttributeTok{Probit =}\NormalTok{ fit.inla.probit}\SpecialCharTok{$}\NormalTok{dic}\SpecialCharTok{$}\NormalTok{dic,}
  \AttributeTok{row.names =} \StringTok{"Deviance Information Criteria"}
\NormalTok{)}
\end{Highlighting}
\end{Shaded}

\hypertarget{latent-heritability}{%
\section*{Latent heritability}\label{latent-heritability}}

Below is a function to get \(h^2_\text{obs}\), \(h^2_\text{lat}\) and
\(h^2_\Phi\) based on a fitted model.

\begin{Shaded}
\begin{Highlighting}[]
\NormalTok{get.h2 }\OtherTok{\textless{}{-}} \ControlFlowTok{function}\NormalTok{(inla.fit, n, }\AttributeTok{use.scale =} \ConstantTok{FALSE}\NormalTok{, }\AttributeTok{model =} \ConstantTok{NA}\NormalTok{,}
                   \AttributeTok{include.fixed =} \ConstantTok{FALSE}\NormalTok{) \{}
  \CommentTok{\#\textquotesingle{} Get heritability}
  \CommentTok{\#\textquotesingle{}}
  \CommentTok{\#\textquotesingle{} Get n samples of heritability (h\^{}2) from INLA object}
  \CommentTok{\#\textquotesingle{} @param inla.fit fitted model}
  \CommentTok{\#\textquotesingle{} @param n number of samples}
  \CommentTok{\#\textquotesingle{} @param use.scale flag for adding link variance to denominator}
  \CommentTok{\#\textquotesingle{} @param model string representation of model type}
  \CommentTok{\#\textquotesingle{} @param include.fixed flag for including fixed effects variance}
  \CommentTok{\#\textquotesingle{} @return n{-}sized vector of heritability samples}
\NormalTok{  samples }\OtherTok{\textless{}{-}} \FunctionTok{inla.hyperpar.sample}\NormalTok{(}\AttributeTok{n =}\NormalTok{ n, inla.fit)}
\NormalTok{  denominator }\OtherTok{\textless{}{-}} \DecValTok{0}
  \ControlFlowTok{for}\NormalTok{ (cname }\ControlFlowTok{in} \FunctionTok{colnames}\NormalTok{(samples)) \{}
\NormalTok{    denominator }\OtherTok{\textless{}{-}}\NormalTok{ denominator }\SpecialCharTok{+} \DecValTok{1} \SpecialCharTok{/}\NormalTok{ samples[, cname]}
\NormalTok{  \}}
  \ControlFlowTok{if}\NormalTok{ (include.fixed) \{}
    \CommentTok{\# Grab variance from summary object in INLA fit}
\NormalTok{    denominator }\OtherTok{\textless{}{-}}\NormalTok{ denominator }\SpecialCharTok{+} \FunctionTok{sum}\NormalTok{(inla.fit}\SpecialCharTok{$}\NormalTok{summary.fixed[, }\StringTok{"sd"}\NormalTok{]}\SpecialCharTok{\^{}}\DecValTok{2}\NormalTok{)}
\NormalTok{  \}}

  \ControlFlowTok{if}\NormalTok{ (use.scale) \{}
\NormalTok{    scales.dictionary }\OtherTok{\textless{}{-}} \FunctionTok{list}\NormalTok{(}
      \AttributeTok{binom1.probit =} \DecValTok{1}\NormalTok{, }\AttributeTok{binom1.logit =}\NormalTok{ pi}\SpecialCharTok{\^{}}\DecValTok{2} \SpecialCharTok{/} \DecValTok{3}\NormalTok{,}
      \AttributeTok{round =} \FloatTok{0.25}
\NormalTok{    )}
\NormalTok{    scale.param }\OtherTok{\textless{}{-}} \FunctionTok{get}\NormalTok{(model, scales.dictionary)}
\NormalTok{    denominator }\OtherTok{\textless{}{-}}\NormalTok{ denominator }\SpecialCharTok{+}\NormalTok{ scale.param}
\NormalTok{  \}}

\NormalTok{  h2.inla }\OtherTok{\textless{}{-}}\NormalTok{ (}\DecValTok{1} \SpecialCharTok{/}\NormalTok{ samples[, }\StringTok{"Precision for id"}\NormalTok{]) }\SpecialCharTok{/}\NormalTok{ denominator}
  \FunctionTok{return}\NormalTok{(h2.inla)}
\NormalTok{\}}

\NormalTok{threshold.scaling.param }\OtherTok{\textless{}{-}} \ControlFlowTok{function}\NormalTok{(p) \{}
  \CommentTok{\# h\^{}2\_liab = threshold.scaling.param * h\^{}2\_obs}
\NormalTok{  p }\SpecialCharTok{*}\NormalTok{ (}\DecValTok{1} \SpecialCharTok{{-}}\NormalTok{ p) }\SpecialCharTok{/}\NormalTok{ (}\FunctionTok{dnorm}\NormalTok{(}\FunctionTok{qnorm}\NormalTok{(p)))}\SpecialCharTok{\^{}}\DecValTok{2}
\NormalTok{\}}
\end{Highlighting}
\end{Shaded}

\hypertarget{simulation-data}{%
\section*{Simulation data}\label{simulation-data}}

We implement a general simulation method that, based on the passed
arguments, generates simulation data and fits animal models onto the
dichotomized response.

\begin{Shaded}
\begin{Highlighting}[]
\NormalTok{simulated.heritability }\OtherTok{\textless{}{-}} \ControlFlowTok{function}\NormalTok{(}\AttributeTok{NeNc =} \FloatTok{0.5}\NormalTok{, }\AttributeTok{idgen =} \DecValTok{100}\NormalTok{, }\AttributeTok{nGen =} \DecValTok{9}\NormalTok{,}
                                   \AttributeTok{sigmaA =} \FloatTok{0.8}\NormalTok{, }\AttributeTok{linear.predictor =} \ConstantTok{NA}\NormalTok{,}
                                   \AttributeTok{simulated.formula =} \ConstantTok{NA}\NormalTok{,}
                                   \AttributeTok{dichotomize =} \StringTok{"round"}\NormalTok{,}
                                   \AttributeTok{pc.prior =} \ConstantTok{NA}\NormalTok{, }\AttributeTok{probit.model =} \ConstantTok{FALSE}\NormalTok{,}
                                   \AttributeTok{simulated.formula.probit =} \ConstantTok{NA}\NormalTok{,}
                                   \AttributeTok{DIC =} \ConstantTok{FALSE}\NormalTok{) \{}
  \CommentTok{\#\textquotesingle{} Simulate and fit animal model}
  \CommentTok{\#\textquotesingle{}}
  \CommentTok{\#\textquotesingle{} Generate pedigree, fit Gaussian (INLA) model and provide}
  \CommentTok{\#\textquotesingle{} heritability estimate.}
  \CommentTok{\#\textquotesingle{} @param NeNc Effective/Census population mean, used to determine}
  \CommentTok{\#\textquotesingle{} number of fathers and mothers per generation,}
  \CommentTok{\#\textquotesingle{} @param idgen Number of individuals per generation in pedigree}
  \CommentTok{\#\textquotesingle{} @param nGen Number of generations in pedigree}
  \CommentTok{\#\textquotesingle{} @param sigmaA:  Additive genetic variance}
  \CommentTok{\#\textquotesingle{} @param linear.predictor Callable function of two parameters, the}
  \CommentTok{\#\textquotesingle{} first \textquotesingle{}u\textquotesingle{} (breeding values), the second for the data}
  \CommentTok{\#\textquotesingle{} @param simulated.formula Formula expression using the response}
  \CommentTok{\#\textquotesingle{} name \textasciigrave{}simulated.response\textasciigrave{}, param \textasciigrave{}id\textasciigrave{} and \textasciigrave{}Cmatrix\textasciigrave{}, both of}
  \CommentTok{\#\textquotesingle{} which are defined locally in this method.}
  \CommentTok{\#\textquotesingle{} @param dichotomize Dichotomization method (round, binomial, etc)}
  \CommentTok{\#\textquotesingle{} @param pc.prior (optional) parameters for PC prior}
  \CommentTok{\#\textquotesingle{} @param probit.model (optional) flag to fit binomial probit model}
  \CommentTok{\#\textquotesingle{} in addition to the Gaussian model.}
  \CommentTok{\#\textquotesingle{} @param simulated.formula.probit (opt) Formula for probit model}
  \CommentTok{\#\textquotesingle{} @param DIC (optional) Flag for computing DIC for the models}
  \CommentTok{\#\textquotesingle{}}
  \CommentTok{\#\textquotesingle{} @return A list with the following items:}
  \CommentTok{\#\textquotesingle{} heritability: Posterior latent heritability samples}
  \CommentTok{\#\textquotesingle{} summary: List of mean, standard deviation and quantiles of h\^{}2}
  \CommentTok{\#\textquotesingle{} p: Portion of \textasciigrave{}TRUE\textasciigrave{} observations in simulated response}
  \CommentTok{\#\textquotesingle{} simulated.response:  The observed values in simulation dataset}
  \CommentTok{\#\textquotesingle{} fit: Gaussian fitted model}
  \CommentTok{\#\textquotesingle{} fit.probit: Probit fitted model, if \textasciigrave{}probit.model=F\textasciigrave{}, is \textasciigrave{}NULL\textasciigrave{}.}
\NormalTok{  ped0 }\OtherTok{\textless{}{-}} \FunctionTok{generatePedigree}\NormalTok{(}
    \AttributeTok{nId =}\NormalTok{ idgen, }\AttributeTok{nGeneration =}\NormalTok{ nGen,}
    \AttributeTok{nFather =}\NormalTok{ idgen }\SpecialCharTok{*}\NormalTok{ NeNc, }\AttributeTok{nMother =}\NormalTok{ idgen }\SpecialCharTok{*}\NormalTok{ NeNc}
\NormalTok{  )}
  \CommentTok{\# Set correct format for pedigree}
\NormalTok{  pedigree }\OtherTok{\textless{}{-}}\NormalTok{ ped0[, }\FunctionTok{c}\NormalTok{(}\DecValTok{1}\NormalTok{, }\DecValTok{3}\NormalTok{, }\DecValTok{2}\NormalTok{, }\DecValTok{5}\NormalTok{)]}
  \FunctionTok{names}\NormalTok{(pedigree) }\OtherTok{\textless{}{-}} \FunctionTok{c}\NormalTok{(}\StringTok{"id"}\NormalTok{, }\StringTok{"dam"}\NormalTok{, }\StringTok{"sire"}\NormalTok{, }\StringTok{"sex"}\NormalTok{)}

  \CommentTok{\# Generate random breeding values}
  \CommentTok{\# The following will CRASH if you don\textquotesingle{}t}
  \CommentTok{\# use the patched MCMCglmm package!}
\NormalTok{  u }\OtherTok{\textless{}{-}} \FunctionTok{rbv}\NormalTok{(pedigree[, }\FunctionTok{c}\NormalTok{(}\DecValTok{1}\NormalTok{, }\DecValTok{2}\NormalTok{, }\DecValTok{3}\NormalTok{)], sigmaA)}

\NormalTok{  simulated.d.ped }\OtherTok{\textless{}{-}}\NormalTok{ nadiv}\SpecialCharTok{::}\FunctionTok{prepPed}\NormalTok{(pedigree, }\AttributeTok{gender =} \StringTok{"sex"}\NormalTok{)}
  \CommentTok{\# Binarize from (1,2) to (0,1)}
\NormalTok{  simulated.d.ped}\SpecialCharTok{$}\NormalTok{sex }\OtherTok{\textless{}{-}}\NormalTok{ simulated.d.ped}\SpecialCharTok{$}\NormalTok{sex }\SpecialCharTok{{-}} \DecValTok{1} 
\NormalTok{  simulated.Cmatrix }\OtherTok{\textless{}{-}}\NormalTok{ nadiv}\SpecialCharTok{::}\FunctionTok{makeAinv}\NormalTok{(pedigree[, }\FunctionTok{c}\NormalTok{(}\DecValTok{1}\NormalTok{, }\DecValTok{2}\NormalTok{, }\DecValTok{3}\NormalTok{)])}\SpecialCharTok{$}\NormalTok{Ainv}
  \CommentTok{\# Make index to allow iid noise random effect}
\NormalTok{  simulated.d.ped}\SpecialCharTok{$}\NormalTok{ind }\OtherTok{\textless{}{-}} \FunctionTok{seq\_len}\NormalTok{(}\FunctionTok{nrow}\NormalTok{(simulated.d.ped))}

  \CommentTok{\# Generating "true" y\_i}

  \ControlFlowTok{if}\NormalTok{ (dichotomize }\SpecialCharTok{==} \StringTok{"binom1.logit"}\NormalTok{) \{}
\NormalTok{    simulated.response }\OtherTok{\textless{}{-}} \FunctionTok{rbinom}\NormalTok{(}\FunctionTok{length}\NormalTok{(u),}
      \AttributeTok{size =} \DecValTok{1}\NormalTok{,}
      \AttributeTok{prob =} \FunctionTok{pnorm}\NormalTok{(}\FunctionTok{linear.predictor}\NormalTok{(u, simulated.d.ped))}
\NormalTok{    )}
\NormalTok{  \} }\ControlFlowTok{else} \ControlFlowTok{if}\NormalTok{ (dichotomize }\SpecialCharTok{==} \StringTok{"round"}\NormalTok{) \{}
    \CommentTok{\# This assumes mean of \textbackslash{}eta\_i is 0}
\NormalTok{    simulated.response }\OtherTok{\textless{}{-}} \FunctionTok{ifelse}\NormalTok{(}
      \FunctionTok{linear.predictor}\NormalTok{(u, simulated.d.ped) }\SpecialCharTok{\textless{}=} \DecValTok{0}\NormalTok{, }\DecValTok{0}\NormalTok{, }\DecValTok{1}
\NormalTok{    )}
\NormalTok{  \} }\ControlFlowTok{else} \ControlFlowTok{if}\NormalTok{ (dichotomize }\SpecialCharTok{==} \StringTok{"round\_balanced"}\NormalTok{) \{}
    \CommentTok{\# Get balanced residuals for unbalanced linear predictor}
\NormalTok{    cutoff }\OtherTok{\textless{}{-}} \FunctionTok{mean}\NormalTok{(}\FunctionTok{linear.predictor}\NormalTok{(u, simulated.d.ped))}
\NormalTok{    simulated.response }\OtherTok{\textless{}{-}} \FunctionTok{ifelse}\NormalTok{(}
      \FunctionTok{linear.predictor}\NormalTok{(u, simulated.d.ped) }\SpecialCharTok{\textless{}=}\NormalTok{ cutoff, }\DecValTok{0}\NormalTok{, }\DecValTok{1}
\NormalTok{    )}
\NormalTok{  \} }\ControlFlowTok{else} \ControlFlowTok{if}\NormalTok{ (}\FunctionTok{is.numeric}\NormalTok{(dichotomize)) \{}
    \FunctionTok{stopifnot}\NormalTok{(dichotomize }\SpecialCharTok{\textgreater{}=} \DecValTok{0} \SpecialCharTok{\&}\NormalTok{ dichotomize }\SpecialCharTok{\textless{}=} \DecValTok{1}\NormalTok{)}
\NormalTok{    eta\_values }\OtherTok{\textless{}{-}} \FunctionTok{linear.predictor}\NormalTok{(u, simulated.d.ped)}
\NormalTok{    cutoff }\OtherTok{\textless{}{-}} \FunctionTok{quantile}\NormalTok{(eta\_values, }\DecValTok{1} \SpecialCharTok{{-}}\NormalTok{ dichotomize)}
    \CommentTok{\# e.g. dichotomize=0.1, p should be about 0.1}
\NormalTok{    simulated.response }\OtherTok{\textless{}{-}} \FunctionTok{ifelse}\NormalTok{(eta\_values }\SpecialCharTok{\textless{}=}\NormalTok{ cutoff, }\DecValTok{0}\NormalTok{, }\DecValTok{1}\NormalTok{)}
\NormalTok{  \} }\ControlFlowTok{else}\NormalTok{ \{}
    \FunctionTok{stop}\NormalTok{(}\FunctionTok{paste0}\NormalTok{(}
      \StringTok{"Unknown dichotomization method \textquotesingle{}"}\NormalTok{, dichotomize, }\StringTok{"\textquotesingle{}. "}\NormalTok{,}
      \StringTok{"Consider using \textquotesingle{}binom1.logit\textquotesingle{} or \textquotesingle{}round\textquotesingle{}."}
\NormalTok{    ))}
\NormalTok{  \}}
\NormalTok{  p }\OtherTok{\textless{}{-}} \FunctionTok{mean}\NormalTok{(simulated.response) }\CommentTok{\# portion of true responses}


  \CommentTok{\# Model fitting LMM for binary trait}
  \CommentTok{\# First reload formula environment to access local variables}
  \FunctionTok{environment}\NormalTok{(simulated.formula) }\OtherTok{\textless{}{-}} \FunctionTok{environment}\NormalTok{()}
  \FunctionTok{environment}\NormalTok{(simulated.formula.probit) }\OtherTok{\textless{}{-}} \FunctionTok{environment}\NormalTok{()}

\NormalTok{  simulated.fit.inla }\OtherTok{\textless{}{-}} \FunctionTok{inla}\NormalTok{(}
    \AttributeTok{formula =}\NormalTok{ simulated.formula, }\AttributeTok{family =} \StringTok{"gaussian"}\NormalTok{,}
    \AttributeTok{data =}\NormalTok{ simulated.d.ped, }\AttributeTok{control.compute =} \FunctionTok{list}\NormalTok{(}\AttributeTok{dic =}\NormalTok{ DIC)}
\NormalTok{  )}

  \CommentTok{\# Checks for error status in INLA fit,}
  \ControlFlowTok{if}\NormalTok{ (simulated.fit.inla}\SpecialCharTok{$}\NormalTok{mode}\SpecialCharTok{$}\NormalTok{mode.status }\SpecialCharTok{!=} \DecValTok{0}\NormalTok{) \{}
    \FunctionTok{cat}\NormalTok{(}\StringTok{"}\SpecialCharTok{\textbackslash{}n}\StringTok{[WARNING], INLA status"}\NormalTok{,}
\NormalTok{        simulated.fit.inla}\SpecialCharTok{$}\NormalTok{mode}\SpecialCharTok{$}\NormalTok{mode.status, }\StringTok{"}\SpecialCharTok{\textbackslash{}n}\StringTok{"}\NormalTok{)}
\NormalTok{  \}}
\NormalTok{  heritability }\OtherTok{\textless{}{-}} \FunctionTok{get.h2}\NormalTok{(simulated.fit.inla, }\DecValTok{10000}\NormalTok{) }

  \CommentTok{\# Also fit probit if specified}
  \ControlFlowTok{if}\NormalTok{ (probit.model) \{}
\NormalTok{    fit.probit }\OtherTok{\textless{}{-}} \FunctionTok{inla}\NormalTok{(}
      \AttributeTok{formula =}\NormalTok{ simulated.formula.probit, }\AttributeTok{family =} \StringTok{"binomial"}\NormalTok{,}
      \AttributeTok{data =}\NormalTok{ simulated.d.ped,}
      \AttributeTok{control.compute =} \FunctionTok{list}\NormalTok{(}\AttributeTok{return.marginals.predictor =} \ConstantTok{TRUE}\NormalTok{,}
                             \AttributeTok{dic =}\NormalTok{ DIC)}
\NormalTok{    )}
\NormalTok{  \} }\ControlFlowTok{else}\NormalTok{ \{}
\NormalTok{    fit.probit }\OtherTok{\textless{}{-}} \ConstantTok{NULL}
\NormalTok{  \}}
  \FunctionTok{list}\NormalTok{(}
    \AttributeTok{heritability =}\NormalTok{ heritability,}
    \AttributeTok{summary =} \FunctionTok{list}\NormalTok{(}
      \AttributeTok{mean =} \FunctionTok{mean}\NormalTok{(heritability),}
      \AttributeTok{standard.deviation =} \FunctionTok{sd}\NormalTok{(heritability),}
      \AttributeTok{quantiles =} \FunctionTok{quantile}\NormalTok{(heritability, }\AttributeTok{probs =} \FunctionTok{c}\NormalTok{(}\FloatTok{0.025}\NormalTok{, }\FloatTok{0.5}\NormalTok{, }\FloatTok{0.975}\NormalTok{))}
\NormalTok{    ),}
    \AttributeTok{p =}\NormalTok{ p,}
    \AttributeTok{simulated.response =}\NormalTok{ simulated.response,}
    \AttributeTok{fit =}\NormalTok{ simulated.fit.inla,}
    \AttributeTok{fit.probit =}\NormalTok{ fit.probit}
\NormalTok{  )}
\NormalTok{\}}
\end{Highlighting}
\end{Shaded}

Now we try to run it through the model pipeline. Here we try
\(\eta_i = u_i + e_i\) (extra iid effect) and
\(y_i = u_i+\varepsilon_i\).

\begin{Shaded}
\begin{Highlighting}[]
\NormalTok{simulated.formula }\OtherTok{\textless{}{-}}\NormalTok{ simulated.response }\SpecialCharTok{\textasciitilde{}}  \FunctionTok{f}\NormalTok{(id,}
  \AttributeTok{model =} \StringTok{"generic0"}\NormalTok{,}
  \AttributeTok{Cmatrix =}\NormalTok{ simulated.Cmatrix,}
  \AttributeTok{constr =} \ConstantTok{FALSE}\NormalTok{,}
  \AttributeTok{hyper =} \FunctionTok{list}\NormalTok{(}
    \AttributeTok{prec =} \FunctionTok{list}\NormalTok{(}\AttributeTok{initial =} \FunctionTok{log}\NormalTok{(}\DecValTok{1} \SpecialCharTok{/} \DecValTok{10}\NormalTok{), }\AttributeTok{prior =} \StringTok{"pc.prec"}\NormalTok{,}
                \AttributeTok{param =} \FunctionTok{c}\NormalTok{(}\DecValTok{1}\NormalTok{, }\FloatTok{0.05}\NormalTok{))}
\NormalTok{  ))}
\NormalTok{simulated.formula.probit }\OtherTok{\textless{}{-}}\NormalTok{ simulated.response }\SpecialCharTok{\textasciitilde{}}  \FunctionTok{f}\NormalTok{(id,}
  \AttributeTok{model =} \StringTok{"generic0"}\NormalTok{,}
  \AttributeTok{Cmatrix =}\NormalTok{ simulated.Cmatrix,}
  \AttributeTok{constr =} \ConstantTok{FALSE}\NormalTok{,}
  \AttributeTok{hyper =} \FunctionTok{list}\NormalTok{(}
    \AttributeTok{prec =} \FunctionTok{list}\NormalTok{(}\AttributeTok{initial =} \FunctionTok{log}\NormalTok{(}\DecValTok{1} \SpecialCharTok{/} \DecValTok{10}\NormalTok{), }\AttributeTok{prior =} \StringTok{"pc.prec"}\NormalTok{,}
                \AttributeTok{param =} \FunctionTok{c}\NormalTok{(}\DecValTok{1}\NormalTok{, }\FloatTok{0.05}\NormalTok{))}
\NormalTok{  )) }\SpecialCharTok{+} \FunctionTok{f}\NormalTok{(ind, }\AttributeTok{model=}\StringTok{"iid"}\NormalTok{, }\AttributeTok{hyper=}\FunctionTok{list}\NormalTok{(}\AttributeTok{prec=}\FunctionTok{list}\NormalTok{(}
    \AttributeTok{prior=}\StringTok{"pc.prec"}\NormalTok{, }\AttributeTok{param=}\FunctionTok{c}\NormalTok{(}\DecValTok{1}\NormalTok{,}\FloatTok{0.05}\NormalTok{))))}

\NormalTok{get.simulated.threshold.value }\OtherTok{\textless{}{-}} \ControlFlowTok{function}\NormalTok{(}
\NormalTok{    NeNc, idgen, nGen, sigmaA, linear.predictor, simulated.formula,}
    \AttributeTok{Vp =} \ConstantTok{NA}\NormalTok{) \{}
  \CommentTok{\#\textquotesingle{} Get h\^{}2 statistics from simulation}
  \CommentTok{\#\textquotesingle{}}
  \CommentTok{\#\textquotesingle{} Makes a dataframe showing observation h\^{}2 and}
  \CommentTok{\#\textquotesingle{} liability h\^{}2 with 95\% confidence interval. Most parameters are }
  \CommentTok{\#\textquotesingle{} for generating pedigree and not covered in this docstring.}
  \CommentTok{\#\textquotesingle{} @param Vp Total phenotypic variance, used to get "true" h\^{}2}
  \CommentTok{\#\textquotesingle{} @return A dataframe with said statistics}
  \FunctionTok{stopifnot}\NormalTok{(}\StringTok{"Vp required to get true h\^{}2"} \OtherTok{=} \SpecialCharTok{!}\FunctionTok{is.na}\NormalTok{(Vp))}
\NormalTok{  sim.result }\OtherTok{\textless{}{-}} \FunctionTok{simulated.heritability}\NormalTok{(}
\NormalTok{    NeNc, idgen, nGen, sigmaA,}
\NormalTok{    linear.predictor, simulated.formula}
\NormalTok{  )}
\NormalTok{  threshold.scaled.h2 }\OtherTok{\textless{}{-}} \FunctionTok{threshold.scaling.param}\NormalTok{(sim.result}\SpecialCharTok{$}\NormalTok{p) }\SpecialCharTok{*}
\NormalTok{    sim.result}\SpecialCharTok{$}\NormalTok{heritability}
\NormalTok{  simulation.h2.true }\OtherTok{\textless{}{-}}\NormalTok{ sigmaA }\SpecialCharTok{/}\NormalTok{ (Vp)}

  \FunctionTok{data.frame}\NormalTok{(}
    \AttributeTok{Simulation =} \FunctionTok{c}\NormalTok{(}
\NormalTok{      simulation.h2.true, }\FunctionTok{mean}\NormalTok{(threshold.scaled.h2),}
      \FunctionTok{paste0}\NormalTok{(}\StringTok{"("}\NormalTok{, }\FunctionTok{paste}\NormalTok{(}\FunctionTok{format}\NormalTok{(}
        \FunctionTok{quantile}\NormalTok{(threshold.scaled.h2, }\AttributeTok{probs =} \FunctionTok{c}\NormalTok{(}\FloatTok{0.025}\NormalTok{, }\FloatTok{0.975}\NormalTok{)),}
        \AttributeTok{digits =} \DecValTok{4}
\NormalTok{      ), }\AttributeTok{collapse =} \StringTok{", "}\NormalTok{), }\StringTok{")"}\NormalTok{),}
      \FunctionTok{mean}\NormalTok{(sim.result}\SpecialCharTok{$}\NormalTok{summary}\SpecialCharTok{$}\NormalTok{mean)}
\NormalTok{    ),}
    \AttributeTok{row.names =} \FunctionTok{c}\NormalTok{(}
      \StringTok{"True h\^{}2"}\NormalTok{, }\StringTok{"Estimated h\^{}2\_obs, mean"}\NormalTok{,}
      \StringTok{"95\% Confidence interval"}\NormalTok{, }\StringTok{"Estimated latent mean"}
\NormalTok{    )}
\NormalTok{  )}
\NormalTok{\}}

\FunctionTok{get.simulated.threshold.value}\NormalTok{(}
  \AttributeTok{NeNc =} \FloatTok{0.5}\NormalTok{, }\AttributeTok{idgen =} \DecValTok{100}\NormalTok{, }\AttributeTok{nGen =} \DecValTok{9}\NormalTok{, }\AttributeTok{sigmaA =} \FloatTok{0.05}\NormalTok{,}
  \ControlFlowTok{function}\NormalTok{(u, .) \{}
\NormalTok{    u }\SpecialCharTok{+} \FunctionTok{rnorm}\NormalTok{(}\FunctionTok{length}\NormalTok{(u))}
\NormalTok{  \},}
\NormalTok{  simulated.formula,}
  \AttributeTok{Vp =} \FloatTok{0.05} \SpecialCharTok{+} \DecValTok{1}
\NormalTok{)}
\end{Highlighting}
\end{Shaded}

Further, we quantitatively look into estimates for different
\(\sigma^2_A\):

\hypertarget{performance-over-varying-v_a}{%
\subsection*{\texorpdfstring{Performance over varying
\(V_A\)}{Performance over varying V\_A}}\label{performance-over-varying-v_a}}

\begin{Shaded}
\begin{Highlighting}[]
\NormalTok{plot.h2.deviation }\OtherTok{\textless{}{-}} \ControlFlowTok{function}\NormalTok{(}
    \AttributeTok{dichotomize =} \StringTok{"round"}\NormalTok{,}
    \AttributeTok{title =} \StringTok{"Simulation heritability"}\NormalTok{,}
    \AttributeTok{SAVE.PLOT =} \ConstantTok{TRUE}\NormalTok{, }\AttributeTok{plot.fn =} \ConstantTok{NA}\NormalTok{, }\AttributeTok{sigma.scale =} \StringTok{"log"}\NormalTok{,}
    \AttributeTok{lin.pred =} \ConstantTok{NULL}\NormalTok{,}
    \AttributeTok{dynamic.priors =} \ConstantTok{FALSE}\NormalTok{, }\AttributeTok{simulated.formula =} \ConstantTok{NULL}\NormalTok{, }\AttributeTok{Ve =} \ConstantTok{NULL}\NormalTok{,}
    \AttributeTok{fixedeffects =} \ConstantTok{FALSE}\NormalTok{) \{}
  \CommentTok{\#\textquotesingle{} Plot h\^{}2 estimate, alongside true value, for a series of V\_A}
  \CommentTok{\#\textquotesingle{}}
  \CommentTok{\#\textquotesingle{} For each V\_A, generate simulation and fit a Gaussian model.}
  \CommentTok{\#\textquotesingle{} Then, plot the obtained h\^{}2 for observation and liability scale,}
  \CommentTok{\#\textquotesingle{} alongside the true value.}
  \CommentTok{\#\textquotesingle{} @param dichotomize Dichotomization method, used for simulation}
  \CommentTok{\#\textquotesingle{} @param title ggplot legend title used as title for all (sub)plots}
  \CommentTok{\#\textquotesingle{} @param SAVE.PLOT flag for saving plot to disk}
  \CommentTok{\#\textquotesingle{} @param plot.fn (optional) String to add to the end of the }
  \CommentTok{\#\textquotesingle{} filename, before file extension, when saving the plot.}
  \CommentTok{\#\textquotesingle{} @param sigma.scale either "log" or "small", deciding what values,}
  \CommentTok{\#\textquotesingle{} and the spacing between values of V\_A to be iterated over.}
  \CommentTok{\#\textquotesingle{} @param lin.pred linear predictor for simulation}
  \CommentTok{\#\textquotesingle{} @param dynamic.priors Flag for changing model priors based on V\_A}
  \CommentTok{\#\textquotesingle{} @param simulated.formula Formula for simulatin}
  \CommentTok{\#\textquotesingle{} @param Ve Residual variance, or fixed effects variance}
  \CommentTok{\#\textquotesingle{} for that model}
  \CommentTok{\#\textquotesingle{} @param fixedeffects Flag to determine if model has fixed effects}
  \CommentTok{\#\textquotesingle{} @return List with item "p" for the ggplot object.}

  \ControlFlowTok{if}\NormalTok{ (sigma.scale }\SpecialCharTok{==} \StringTok{"log"}\NormalTok{) \{}
\NormalTok{    sigmaA.list }\OtherTok{\textless{}{-}} \FunctionTok{c}\NormalTok{(}\DecValTok{1}\SpecialCharTok{:}\DecValTok{10} \SpecialCharTok{\%o\%} \DecValTok{10}\SpecialCharTok{\^{}}\NormalTok{(}\SpecialCharTok{{-}}\DecValTok{3}\SpecialCharTok{:}\DecValTok{3}\NormalTok{)) }\CommentTok{\# Log scale[10\^{}{-}3,  10\^{}3]}
\NormalTok{  \} }\ControlFlowTok{else} \ControlFlowTok{if}\NormalTok{ (sigma.scale }\SpecialCharTok{==} \StringTok{"small"}\NormalTok{) \{ }\CommentTok{\# Linear scale [10\^{}{-}3, 0.259]}
\NormalTok{    sigmaA.list }\OtherTok{\textless{}{-}} \FunctionTok{seq}\NormalTok{(}\FloatTok{0.001}\NormalTok{, }\FloatTok{0.26}\NormalTok{, }\AttributeTok{by =} \FloatTok{0.01}\NormalTok{)}
\NormalTok{  \} }\ControlFlowTok{else}\NormalTok{ \{}
    \FunctionTok{stop}\NormalTok{(}\StringTok{"Unrecognized scale for sigmaA."}\NormalTok{)}
\NormalTok{  \}}
\NormalTok{  pc.U.list }\OtherTok{\textless{}{-}} \FunctionTok{c}\NormalTok{(}\FunctionTok{rep}\NormalTok{(}\DecValTok{10}\NormalTok{, }\DecValTok{10}\NormalTok{) }\SpecialCharTok{\%o\%} \DecValTok{10}\SpecialCharTok{\^{}}\NormalTok{(}\SpecialCharTok{{-}}\DecValTok{3}\SpecialCharTok{:}\DecValTok{3}\NormalTok{))}
  \CommentTok{\# pc.U.list \textless{}{-} 2*sigmaA.list \# Alternative dynamic prior}

\NormalTok{  estimates }\OtherTok{\textless{}{-}} \FunctionTok{c}\NormalTok{()}
\NormalTok{  latent }\OtherTok{\textless{}{-}} \FunctionTok{c}\NormalTok{()}
\NormalTok{  true.vals }\OtherTok{\textless{}{-}} \FunctionTok{c}\NormalTok{()}
\NormalTok{  est.CI.u }\OtherTok{\textless{}{-}} \FunctionTok{c}\NormalTok{()}
\NormalTok{  est.CI.l }\OtherTok{\textless{}{-}} \FunctionTok{c}\NormalTok{()}
\NormalTok{  plist }\OtherTok{\textless{}{-}} \FunctionTok{c}\NormalTok{()}
\NormalTok{  iter.num }\OtherTok{\textless{}{-}} \DecValTok{0}
\NormalTok{  Ve0 }\OtherTok{\textless{}{-}}\NormalTok{ Ve}
  \ControlFlowTok{for}\NormalTok{ (sigmaA }\ControlFlowTok{in}\NormalTok{ sigmaA.list) \{}
    \FunctionTok{cat}\NormalTok{(}\StringTok{"\textgreater{}"}\NormalTok{)}
    \ControlFlowTok{if}\NormalTok{ (dynamic.priors) \{}
\NormalTok{      iter.num }\OtherTok{\textless{}{-}}\NormalTok{ iter.num }\SpecialCharTok{+} \DecValTok{1}
      \ControlFlowTok{if}\NormalTok{ (sigmaA }\SpecialCharTok{\textless{}} \DecValTok{1}\NormalTok{) \{}
\NormalTok{        pc.prior }\OtherTok{\textless{}{-}} \FunctionTok{c}\NormalTok{(}\DecValTok{1}\NormalTok{, }\FloatTok{0.05}\NormalTok{)}
\NormalTok{      \} }\ControlFlowTok{else}\NormalTok{ \{}
\NormalTok{        pc.prior }\OtherTok{\textless{}{-}} \FunctionTok{c}\NormalTok{(pc.U.list[iter.num], }\FloatTok{0.05}\NormalTok{)}
\NormalTok{      \}}
\NormalTok{    \} }\ControlFlowTok{else}\NormalTok{ \{}
\NormalTok{      pc.prior }\OtherTok{\textless{}{-}} \FunctionTok{c}\NormalTok{(}\DecValTok{1}\NormalTok{, }\FloatTok{0.05}\NormalTok{)}
\NormalTok{    \}}
    \ControlFlowTok{if}\NormalTok{ (}\FunctionTok{is.null}\NormalTok{(simulated.formula)) \{ }\CommentTok{\# Defaults eta = a\_i + e}
\NormalTok{      simulated.formula }\OtherTok{\textless{}{-}}\NormalTok{ simulated.response }\SpecialCharTok{\textasciitilde{}}  \FunctionTok{f}\NormalTok{(id,}
        \AttributeTok{model =} \StringTok{"generic0"}\NormalTok{,}
        \AttributeTok{Cmatrix =}\NormalTok{ simulated.Cmatrix,}
        \AttributeTok{constr =} \ConstantTok{FALSE}\NormalTok{,}
        \AttributeTok{hyper =} \FunctionTok{list}\NormalTok{(}
          \AttributeTok{prec =} \FunctionTok{list}\NormalTok{(}\AttributeTok{initial =} \FunctionTok{log}\NormalTok{(sigmaA), }\AttributeTok{prior =} \StringTok{"pc.prec"}\NormalTok{,}
                      \AttributeTok{param =}\NormalTok{ pc.prior)}
\NormalTok{        )}
\NormalTok{      )}
\NormalTok{    \}}
    \ControlFlowTok{if}\NormalTok{ (}\FunctionTok{is.null}\NormalTok{(lin.pred)) \{ }\CommentTok{\# The \textquotesingle{}usual\textquotesingle{} linear predictor}
\NormalTok{      lin.pred }\OtherTok{\textless{}{-}} \ControlFlowTok{function}\NormalTok{(u, .) u }\SpecialCharTok{+} \FunctionTok{rnorm}\NormalTok{(}\FunctionTok{length}\NormalTok{(u))}
\NormalTok{    \}}
\NormalTok{    result }\OtherTok{\textless{}{-}} \FunctionTok{simulated.heritability}\NormalTok{(}
      \AttributeTok{NeNc =} \FloatTok{0.5}\NormalTok{, }\AttributeTok{idgen =} \DecValTok{100}\NormalTok{, }\AttributeTok{nGen =} \DecValTok{9}\NormalTok{, }\AttributeTok{sigmaA =}\NormalTok{ sigmaA,}
      \AttributeTok{linear.predictor =}\NormalTok{ lin.pred,}
      \AttributeTok{simulated.formula =}\NormalTok{ simulated.formula,}
      \AttributeTok{dichotomize =}\NormalTok{ dichotomize,}
      \AttributeTok{pc.prior =}\NormalTok{ pc.prior}
\NormalTok{    )}

    \ControlFlowTok{if}\NormalTok{ (}\FunctionTok{is.null}\NormalTok{(Ve)) \{}
      \CommentTok{\# Fallback residual variance}
\NormalTok{      Ve }\OtherTok{\textless{}{-}} \DecValTok{1}
\NormalTok{    \}}
    \ControlFlowTok{if}\NormalTok{ (fixedeffects) \{}
      \CommentTok{\# beta\^{}2 * Var(x\_fixedeffect):}
\NormalTok{      Ve }\OtherTok{\textless{}{-}}\NormalTok{ Ve }\SpecialCharTok{*} \FunctionTok{var}\NormalTok{(result}\SpecialCharTok{$}\NormalTok{simulated.response) }
\NormalTok{      posterior }\OtherTok{\textless{}{-}} \FunctionTok{get.h2}\NormalTok{(result}\SpecialCharTok{$}\NormalTok{fit, }\DecValTok{10000}\NormalTok{, }\AttributeTok{include.fixed =} \ConstantTok{TRUE}\NormalTok{)}
\NormalTok{    \} }\ControlFlowTok{else}\NormalTok{ \{}
\NormalTok{      posterior }\OtherTok{\textless{}{-}}\NormalTok{ result}\SpecialCharTok{$}\NormalTok{heritability}
\NormalTok{    \}}

\NormalTok{    simulation.h2.true }\OtherTok{\textless{}{-}}\NormalTok{ sigmaA }\SpecialCharTok{/}\NormalTok{ (sigmaA }\SpecialCharTok{+}\NormalTok{ Ve)}

\NormalTok{    latent }\OtherTok{\textless{}{-}} \FunctionTok{c}\NormalTok{(latent, }\FunctionTok{mean}\NormalTok{(posterior)) }\CommentTok{\# Observatoin{-}level}
\NormalTok{    threshold.scaled.h2 }\OtherTok{\textless{}{-}} \FunctionTok{threshold.scaling.param}\NormalTok{(result}\SpecialCharTok{$}\NormalTok{p)}\SpecialCharTok{*}\NormalTok{posterior}

\NormalTok{    true.vals }\OtherTok{\textless{}{-}} \FunctionTok{c}\NormalTok{(true.vals, simulation.h2.true)}
\NormalTok{    estimates.CI }\OtherTok{\textless{}{-}} \FunctionTok{quantile}\NormalTok{(threshold.scaled.h2,}
                             \AttributeTok{probs =} \FunctionTok{c}\NormalTok{(}\FloatTok{0.025}\NormalTok{, }\FloatTok{0.975}\NormalTok{))}
\NormalTok{    estimates }\OtherTok{\textless{}{-}} \FunctionTok{c}\NormalTok{(estimates, }\FunctionTok{mean}\NormalTok{(threshold.scaled.h2))}
\NormalTok{    est.CI.l }\OtherTok{\textless{}{-}} \FunctionTok{c}\NormalTok{(est.CI.l, estimates.CI[}\DecValTok{1}\NormalTok{])}
\NormalTok{    est.CI.u }\OtherTok{\textless{}{-}} \FunctionTok{c}\NormalTok{(est.CI.u, estimates.CI[}\DecValTok{2}\NormalTok{])}
\NormalTok{    plist }\OtherTok{\textless{}{-}} \FunctionTok{c}\NormalTok{(plist, result}\SpecialCharTok{$}\NormalTok{p)}
    \CommentTok{\# Reset Ve}
\NormalTok{    Ve }\OtherTok{\textless{}{-}}\NormalTok{ Ve0}
\NormalTok{  \}}
\NormalTok{  res }\OtherTok{\textless{}{-}} \FunctionTok{data.frame}\NormalTok{(}
    \AttributeTok{estimates =}\NormalTok{ estimates,}
    \AttributeTok{true.vals =}\NormalTok{ true.vals, }\AttributeTok{latent =}\NormalTok{ latent,}
    \AttributeTok{est.CI.l =}\NormalTok{ est.CI.l, }\AttributeTok{est.CI.u =}\NormalTok{ est.CI.u,}
    \AttributeTok{sigmaA =}\NormalTok{ sigmaA.list,}
    \AttributeTok{plist =}\NormalTok{ plist}
\NormalTok{  )}
  \CommentTok{\# Plotting}
\NormalTok{  p }\OtherTok{\textless{}{-}} \FunctionTok{ggplot}\NormalTok{(}\AttributeTok{data =}\NormalTok{ res, }\FunctionTok{aes}\NormalTok{(}\AttributeTok{x =}\NormalTok{ sigmaA)) }\SpecialCharTok{+}
    \FunctionTok{geom\_ribbon}\NormalTok{(}\FunctionTok{aes}\NormalTok{(}\AttributeTok{ymin =}\NormalTok{ est.CI.l, }\AttributeTok{ymax =}\NormalTok{ est.CI.u), }\AttributeTok{alpha =} \FloatTok{0.1}\NormalTok{) }\SpecialCharTok{+}
    \FunctionTok{geom\_line}\NormalTok{(}\FunctionTok{aes}\NormalTok{(}\AttributeTok{y =}\NormalTok{ true.vals, }\AttributeTok{color =} \StringTok{"atrue"}\NormalTok{), }\AttributeTok{size =} \FloatTok{1.5}\NormalTok{) }\SpecialCharTok{+}
    \FunctionTok{geom\_line}\NormalTok{(}\FunctionTok{aes}\NormalTok{(}\AttributeTok{y =}\NormalTok{ estimates, }\AttributeTok{color =} \StringTok{"bliab"}\NormalTok{), }\AttributeTok{size =} \FloatTok{1.5}\NormalTok{) }\SpecialCharTok{+}
    \FunctionTok{geom\_line}\NormalTok{(}\FunctionTok{aes}\NormalTok{(}\AttributeTok{y =}\NormalTok{ latent, }\AttributeTok{color =} \StringTok{"obs"}\NormalTok{), }\AttributeTok{size =} \FloatTok{1.5}\NormalTok{) }\SpecialCharTok{+}
    \FunctionTok{xlab}\NormalTok{(}\FunctionTok{TeX}\NormalTok{(}\StringTok{"$}\SpecialCharTok{\textbackslash{}\textbackslash{}}\StringTok{sigma\_A\^{}2$"}\NormalTok{)) }\SpecialCharTok{+}
    \FunctionTok{ylab}\NormalTok{(}\FunctionTok{TeX}\NormalTok{(}\StringTok{"$h\^{}2$"}\NormalTok{)) }\SpecialCharTok{+}
    \FunctionTok{scale\_x\_log10}\NormalTok{() }\SpecialCharTok{+}
    \FunctionTok{scale\_color\_manual}\NormalTok{(}
      \AttributeTok{name =}\NormalTok{ title,}
      \AttributeTok{values =} \FunctionTok{c}\NormalTok{(}
        \StringTok{"atrue"} \OtherTok{=} \StringTok{"darkred"}\NormalTok{, }\StringTok{"bliab"} \OtherTok{=} \StringTok{"steelblue"}\NormalTok{,}
        \StringTok{"obs"} \OtherTok{=} \StringTok{"chartreuse3"}
\NormalTok{      ),}
      \AttributeTok{labels =} \FunctionTok{c}\NormalTok{(}
        \FunctionTok{expression}\NormalTok{(}\StringTok{"True "} \SpecialCharTok{*}\NormalTok{ h[}\StringTok{"liab"}\NormalTok{]}\SpecialCharTok{\^{}}\DecValTok{2}\NormalTok{),}
        \FunctionTok{expression}\NormalTok{(}\StringTok{"Fitted "} \SpecialCharTok{*}\NormalTok{ h[}\StringTok{"liab"}\NormalTok{]}\SpecialCharTok{\^{}}\DecValTok{2}\NormalTok{),}
        \FunctionTok{expression}\NormalTok{(}\StringTok{"Fitted "} \SpecialCharTok{*}\NormalTok{ h[}\StringTok{"obs"}\NormalTok{]}\SpecialCharTok{\^{}}\DecValTok{2}\NormalTok{)}
\NormalTok{      )}
\NormalTok{    ) }\SpecialCharTok{+}
    \FunctionTok{theme}\NormalTok{(}\AttributeTok{text =} \FunctionTok{element\_text}\NormalTok{(}\AttributeTok{size =} \DecValTok{18}\NormalTok{), }\AttributeTok{legend.text.align =} \DecValTok{0}\NormalTok{)}
  \ControlFlowTok{if}\NormalTok{ (SAVE.PLOT) \{}
    \FunctionTok{ggsave}\NormalTok{(}
      \FunctionTok{paste0}\NormalTok{(}
        \StringTok{"../figures/simulation\_deviance\_"}\NormalTok{,}
        \ControlFlowTok{if}\NormalTok{ (}\SpecialCharTok{!}\FunctionTok{is.na}\NormalTok{(plot.fn)) plot.fn, }\StringTok{".pdf"}
\NormalTok{      ), p }\SpecialCharTok{+} \FunctionTok{theme}\NormalTok{(}\AttributeTok{legend.position =} \StringTok{"none"}\NormalTok{),}
      \AttributeTok{width =} \DecValTok{20}\NormalTok{, }\AttributeTok{height =} \DecValTok{20}\NormalTok{,}
      \AttributeTok{units =} \StringTok{"cm"}
\NormalTok{    )}
    \CommentTok{\# Save legend as separate plot}
\NormalTok{    p.legend }\OtherTok{\textless{}{-}}\NormalTok{ cowplot}\SpecialCharTok{::}\FunctionTok{get\_legend}\NormalTok{(p)}
    \FunctionTok{pdf}\NormalTok{(}\FunctionTok{paste0}\NormalTok{(}
      \StringTok{"../figures/simulation\_deviance"}\NormalTok{,}
      \ControlFlowTok{if}\NormalTok{ (fixedeffects) }\StringTok{"\_fixedeffects"}\NormalTok{, }\StringTok{"\_legend.pdf"}
\NormalTok{    ), }\AttributeTok{width =} \FloatTok{7.87402}\NormalTok{, }\AttributeTok{height =} \FloatTok{7.87402}\NormalTok{)}
    \FunctionTok{grid.newpage}\NormalTok{()}
    \FunctionTok{grid.draw}\NormalTok{(p.legend)}
    \FunctionTok{dev.off}\NormalTok{()}
\NormalTok{  \}}
  \FunctionTok{return}\NormalTok{(}\FunctionTok{list}\NormalTok{(}\AttributeTok{p =}\NormalTok{ p))}
\NormalTok{\}}
\end{Highlighting}
\end{Shaded}

We also implement a method to average over several runs, reducing
stochasticity. This should be ran on a remote node rather than locally,
as just one run takes several minutes.

\begin{Shaded}
\begin{Highlighting}[]
\NormalTok{multiple.h2.dev }\OtherTok{\textless{}{-}} \ControlFlowTok{function}\NormalTok{(sigma.scale, ntimes,}
                            \AttributeTok{title =} \StringTok{"Simulation heritability"}\NormalTok{, ...)\{}
  \CommentTok{\#\textquotesingle{} Run plot\_h2\_deviation \textasciigrave{}ntimes\textasciigrave{}}
  \CommentTok{\#\textquotesingle{} }
  \CommentTok{\#\textquotesingle{} Repeats the runs several times, and outputs error plot with}
  \CommentTok{\#\textquotesingle{} mean +{-} SD}
  \CommentTok{\#\textquotesingle{} @param sigma.scale Sigma values to test, either \textquotesingle{}log\textquotesingle{} or \textquotesingle{}small\textquotesingle{}}
  \CommentTok{\#\textquotesingle{} @param ntimes Number of repeated runs}
  \CommentTok{\#\textquotesingle{} @param title Legend title, charcter}
  \CommentTok{\#\textquotesingle{} @param ... Additional parameters for plot\_h2\_deviation}
  \CommentTok{\#\textquotesingle{} @return Dataframe with all info to plot the errorplots}

  \ControlFlowTok{if}\NormalTok{ (sigma.scale }\SpecialCharTok{==} \StringTok{"log"}\NormalTok{) \{}
\NormalTok{    sigmaA.list }\OtherTok{\textless{}{-}} \FunctionTok{c}\NormalTok{(}\DecValTok{1}\SpecialCharTok{:}\DecValTok{10} \SpecialCharTok{\%o\%} \DecValTok{10}\SpecialCharTok{\^{}}\NormalTok{(}\SpecialCharTok{{-}}\DecValTok{3}\SpecialCharTok{:}\DecValTok{3}\NormalTok{))}
\NormalTok{  \} }\ControlFlowTok{else} \ControlFlowTok{if}\NormalTok{ (sigma.scale }\SpecialCharTok{==} \StringTok{"small"}\NormalTok{) \{}
\NormalTok{    sigmaA.list }\OtherTok{\textless{}{-}} \FunctionTok{seq}\NormalTok{(}\FloatTok{0.001}\NormalTok{, }\FloatTok{0.26}\NormalTok{, }\AttributeTok{by =} \FloatTok{0.01}\NormalTok{)}
\NormalTok{  \} }\ControlFlowTok{else}\NormalTok{ \{}
    \FunctionTok{stop}\NormalTok{(}\StringTok{"Unrecognized scale for sigmaA."}\NormalTok{)}
\NormalTok{  \}}
\NormalTok{  n.sigma }\OtherTok{\textless{}{-}} \FunctionTok{length}\NormalTok{(sigmaA.list)}
  \CommentTok{\# Initialize containers: each col is one run}
\NormalTok{  all.h2obs }\OtherTok{\textless{}{-}} \FunctionTok{matrix}\NormalTok{(}\AttributeTok{ncol =}\NormalTok{ ntimes, }\AttributeTok{nrow =}\NormalTok{ n.sigma)}
\NormalTok{  all.h2liab }\OtherTok{\textless{}{-}} \FunctionTok{matrix}\NormalTok{(}\AttributeTok{ncol =}\NormalTok{ ntimes, }\AttributeTok{nrow =}\NormalTok{ n.sigma)}
\NormalTok{  all.truevals }\OtherTok{\textless{}{-}} \FunctionTok{matrix}\NormalTok{(}\AttributeTok{ncol =}\NormalTok{ ntimes, }\AttributeTok{nrow =}\NormalTok{ n.sigma)}
  \ControlFlowTok{for}\NormalTok{(i }\ControlFlowTok{in} \DecValTok{1}\SpecialCharTok{:}\NormalTok{ntimes)\{}
    \FunctionTok{cat}\NormalTok{(}\FunctionTok{paste0}\NormalTok{(}\StringTok{"}\SpecialCharTok{\textbackslash{}n}\StringTok{ [Run "}\NormalTok{, i ,}\StringTok{"/"}\NormalTok{, ntimes, }\StringTok{"]}\SpecialCharTok{\textbackslash{}n}\StringTok{"}\NormalTok{))}
\NormalTok{    res }\OtherTok{\textless{}{-}} \FunctionTok{plot.h2.deviation}\NormalTok{(}\AttributeTok{SAVE.PLOT =} \ConstantTok{FALSE}\NormalTok{,}
                             \AttributeTok{sigma.scale =}\NormalTok{ sigma.scale, ...)}
    \FunctionTok{cat}\NormalTok{(}\StringTok{"Latent:"}\NormalTok{, res}\SpecialCharTok{$}\NormalTok{p}\SpecialCharTok{$}\NormalTok{data}\SpecialCharTok{$}\NormalTok{latent)}
\NormalTok{    all.h2obs[, i] }\OtherTok{\textless{}{-}}\NormalTok{ res}\SpecialCharTok{$}\NormalTok{p}\SpecialCharTok{$}\NormalTok{data}\SpecialCharTok{$}\NormalTok{latent}
\NormalTok{    all.h2liab[, i] }\OtherTok{\textless{}{-}}\NormalTok{ res}\SpecialCharTok{$}\NormalTok{p}\SpecialCharTok{$}\NormalTok{data}\SpecialCharTok{$}\NormalTok{estimate}
\NormalTok{    all.truevals[, i] }\OtherTok{\textless{}{-}}\NormalTok{ res}\SpecialCharTok{$}\NormalTok{p}\SpecialCharTok{$}\NormalTok{data}\SpecialCharTok{$}\NormalTok{true.vals}
\NormalTok{  \}}
\NormalTok{  plot.data }\OtherTok{\textless{}{-}} \FunctionTok{data.frame}\NormalTok{(}
    \AttributeTok{model =} \FunctionTok{rep}\NormalTok{(}\FunctionTok{c}\NormalTok{(}\StringTok{"h2obs"}\NormalTok{, }\StringTok{"h2liab"}\NormalTok{, }\StringTok{"true"}\NormalTok{),}
                \AttributeTok{each =}\NormalTok{ n.sigma, }\AttributeTok{times =} \DecValTok{2}\NormalTok{),}
    \AttributeTok{top =} \FunctionTok{c}\NormalTok{(}\FunctionTok{rowMeans}\NormalTok{(all.h2obs) }\SpecialCharTok{+} \FunctionTok{apply}\NormalTok{(all.h2obs, }\DecValTok{1}\NormalTok{, sd),}
            \FunctionTok{rowMeans}\NormalTok{(all.h2liab) }\SpecialCharTok{+} \FunctionTok{apply}\NormalTok{(all.h2liab, }\DecValTok{1}\NormalTok{, sd),}
            \FunctionTok{rowMeans}\NormalTok{(all.truevals) }\SpecialCharTok{+} \FunctionTok{apply}\NormalTok{(all.truevals, }\DecValTok{1}\NormalTok{, sd)}
\NormalTok{    ),}
    \AttributeTok{mid =} \FunctionTok{c}\NormalTok{(}\FunctionTok{rowMeans}\NormalTok{(all.h2obs), }\FunctionTok{rowMeans}\NormalTok{(all.h2liab),}
            \FunctionTok{rowMeans}\NormalTok{(all.truevals)}
\NormalTok{    ),}
    \AttributeTok{btm =} \FunctionTok{c}\NormalTok{(}\FunctionTok{rowMeans}\NormalTok{(all.h2obs) }\SpecialCharTok{{-}} \FunctionTok{apply}\NormalTok{(all.h2obs, }\DecValTok{1}\NormalTok{, sd),}
            \FunctionTok{rowMeans}\NormalTok{(all.h2liab) }\SpecialCharTok{{-}} \FunctionTok{apply}\NormalTok{(all.h2liab, }\DecValTok{1}\NormalTok{, sd),}
            \FunctionTok{rowMeans}\NormalTok{(all.truevals) }\SpecialCharTok{{-}} \FunctionTok{apply}\NormalTok{(all.truevals, }\DecValTok{1}\NormalTok{, sd)}
\NormalTok{    ),}
    \AttributeTok{xax =} \FunctionTok{rep}\NormalTok{(sigmaA.list, }\AttributeTok{times=}\DecValTok{3}\NormalTok{)}
\NormalTok{  )}
  \FunctionTok{return}\NormalTok{(plot.data)}
\NormalTok{\}}
\end{Highlighting}
\end{Shaded}

We test out a single run over several values below.

\begin{Shaded}
\begin{Highlighting}[]
\DocumentationTok{\#\#\# Plots for sigmaA in (10\^{}{-}3, 10\^{}3)}

\NormalTok{simulation.res }\OtherTok{\textless{}{-}} \FunctionTok{plot.h2.deviation}\NormalTok{(}
  \AttributeTok{plot.fn =} \StringTok{"round"}\NormalTok{,}
  \AttributeTok{SAVE.PLOT =} \ConstantTok{TRUE}\NormalTok{, }\AttributeTok{dynamic.priors =} \ConstantTok{TRUE}
\NormalTok{)}
\NormalTok{simulation.res}\SpecialCharTok{$}\NormalTok{p}

\DocumentationTok{\#\#\# Plot for small values og sigmaA, but finer grid}
\NormalTok{simulation2.res }\OtherTok{\textless{}{-}} \FunctionTok{plot.h2.deviation}\NormalTok{(}
  \AttributeTok{SAVE.PLOT =} \ConstantTok{TRUE}\NormalTok{, }\AttributeTok{sigma.scale =} \StringTok{"small"}\NormalTok{,}
  \AttributeTok{plot.fn =} \StringTok{"small"}\NormalTok{, }\AttributeTok{dynamic.priors =} \ConstantTok{TRUE}
\NormalTok{)}
\NormalTok{simulation2.res}\SpecialCharTok{$}\NormalTok{p}

\DocumentationTok{\#\#\# Standard plotting with different dichotomization techniques}
\FunctionTok{plot.h2.deviation}\NormalTok{(}
  \AttributeTok{dichotomize =} \StringTok{"binom1.logit"}\NormalTok{,}
  \AttributeTok{title =} \StringTok{"Simulation heritability"}\NormalTok{,}
  \AttributeTok{plot.fn =} \StringTok{"binom"}\NormalTok{, }\AttributeTok{dynamic.priors =} \ConstantTok{TRUE}
\NormalTok{)}
\end{Highlighting}
\end{Shaded}

If multiple runs have been done on remote server, loads and plots the
results here,

\begin{Shaded}
\begin{Highlighting}[]
\CommentTok{\# Multiple h\^{}2 deviation plotter}
\NormalTok{markov.plotter }\OtherTok{\textless{}{-}} \ControlFlowTok{function}\NormalTok{(df, }\AttributeTok{legend.name=}\StringTok{""}\NormalTok{)\{}
  \FunctionTok{ggplot}\NormalTok{(df, }\FunctionTok{aes}\NormalTok{(}\AttributeTok{x=}\NormalTok{xax)) }\SpecialCharTok{+}
    \FunctionTok{geom\_pointrange}\NormalTok{(}\FunctionTok{aes}\NormalTok{(}\AttributeTok{ymax=}\NormalTok{top, }\AttributeTok{ymin=}\NormalTok{btm, }\AttributeTok{y=}\NormalTok{mid, }\AttributeTok{color=}\NormalTok{model)) }\SpecialCharTok{+}
    \FunctionTok{geom\_line}\NormalTok{(}\FunctionTok{aes}\NormalTok{(}\AttributeTok{y=}\NormalTok{mid, }\AttributeTok{color=}\NormalTok{model)) }\SpecialCharTok{+} 
    \FunctionTok{xlab}\NormalTok{(}\FunctionTok{TeX}\NormalTok{(}\StringTok{"$}\SpecialCharTok{\textbackslash{}\textbackslash{}}\StringTok{sigma\_A\^{}2$"}\NormalTok{)) }\SpecialCharTok{+}
    \FunctionTok{ylab}\NormalTok{(}\FunctionTok{TeX}\NormalTok{(}\StringTok{"$h\^{}2$"}\NormalTok{)) }\SpecialCharTok{+}
    \FunctionTok{scale\_x\_log10}\NormalTok{() }\SpecialCharTok{+}
    \FunctionTok{scale\_color\_manual}\NormalTok{(}
      \AttributeTok{name =}\NormalTok{ legend.name,}
      \AttributeTok{values =} \FunctionTok{c}\NormalTok{(}
        \StringTok{"true"} \OtherTok{=} \StringTok{"darkred"}\NormalTok{, }\StringTok{"h2liab"} \OtherTok{=} \StringTok{"steelblue"}\NormalTok{,}
        \StringTok{"h2obs"} \OtherTok{=} \StringTok{"chartreuse3"}
\NormalTok{      ),}
      \AttributeTok{labels =} \FunctionTok{c}\NormalTok{(}
        \FunctionTok{expression}\NormalTok{(}\StringTok{"Fitted "} \SpecialCharTok{*}\NormalTok{ h[}\StringTok{"liab"}\NormalTok{]}\SpecialCharTok{\^{}}\DecValTok{2}\NormalTok{),}
        \FunctionTok{expression}\NormalTok{(}\StringTok{"Fitted "} \SpecialCharTok{*}\NormalTok{ h[}\StringTok{"obs"}\NormalTok{]}\SpecialCharTok{\^{}}\DecValTok{2}\NormalTok{),}
        \FunctionTok{expression}\NormalTok{(}\StringTok{"True "} \SpecialCharTok{*}\NormalTok{ h[}\StringTok{"liab"}\NormalTok{]}\SpecialCharTok{\^{}}\DecValTok{2}\NormalTok{)}
\NormalTok{      )}
\NormalTok{    ) }\SpecialCharTok{+}
    \FunctionTok{theme}\NormalTok{(}\AttributeTok{text =} \FunctionTok{element\_text}\NormalTok{(}\AttributeTok{size =} \DecValTok{18}\NormalTok{), }\AttributeTok{legend.position =} \StringTok{"none"}\NormalTok{)}
\NormalTok{\}}
\FunctionTok{load}\NormalTok{(}\StringTok{"markovh2dev\_50\_runs.Rdata"}\NormalTok{)}
\NormalTok{mp1 }\OtherTok{\textless{}{-}} \FunctionTok{markov.plotter}\NormalTok{(markov.result1)}
\NormalTok{mp2 }\OtherTok{\textless{}{-}} \FunctionTok{markov.plotter}\NormalTok{(markov.result2)}
\NormalTok{mp3 }\OtherTok{\textless{}{-}} \FunctionTok{markov.plotter}\NormalTok{(markov.result3,}
                      \AttributeTok{legend.name=}\StringTok{"Simulation heritability"}\NormalTok{)}
\FunctionTok{ggsave}\NormalTok{(}\StringTok{"../figures/simulation\_deviance\_round.pdf"}\NormalTok{, mp1)}
\FunctionTok{ggsave}\NormalTok{(}\StringTok{"../figures/simulation\_deviance\_small.pdf"}\NormalTok{, mp2)}
\FunctionTok{ggsave}\NormalTok{(}\StringTok{"../figures/simulation\_deviance\_binom.pdf"}\NormalTok{, mp3)}
\CommentTok{\# Store legend separately}
\NormalTok{mp3.legend }\OtherTok{\textless{}{-}}\NormalTok{ cowplot}\SpecialCharTok{::}\FunctionTok{get\_legend}\NormalTok{(}
\NormalTok{  mp3 }\SpecialCharTok{+}\FunctionTok{theme}\NormalTok{(}\AttributeTok{legend.position =} \StringTok{"right"}\NormalTok{, }\AttributeTok{legend.text.align =} \DecValTok{0}\NormalTok{))}
\FunctionTok{pdf}\NormalTok{(}\StringTok{"../figures/simulation\_deviance\_legend.pdf"}\NormalTok{,}
    \AttributeTok{width =} \FloatTok{7.87402}\NormalTok{, }\AttributeTok{height =} \FloatTok{7.87402}\NormalTok{)}
\FunctionTok{grid.newpage}\NormalTok{()}
\FunctionTok{grid.draw}\NormalTok{(mp3.legend)}
\FunctionTok{dev.off}\NormalTok{()}
\end{Highlighting}
\end{Shaded}

\hypertarget{without-residual-in-linear-predictor}{%
\subsection*{Without residual in linear
predictor}\label{without-residual-in-linear-predictor}}

We also check what happens if we rerun with \(\eta_i=a_i\), i.e.~without
residuals on underlying scale. Similar results.

\begin{Shaded}
\begin{Highlighting}[]
\FunctionTok{get.simulated.threshold.value}\NormalTok{(}
  \FloatTok{0.5}\NormalTok{, }\DecValTok{100}\NormalTok{, }\DecValTok{9}\NormalTok{, }\DecValTok{10}\NormalTok{, }\ControlFlowTok{function}\NormalTok{(u, .) u,}
\NormalTok{  simulated.formula, }\DecValTok{10}
\NormalTok{)}
\end{Highlighting}
\end{Shaded}

Now we want to look into how \(\mathbf{A}\) (the relatedness matrix)
looks like.

\begin{Shaded}
\begin{Highlighting}[]
\NormalTok{plot.A.matrix }\OtherTok{\textless{}{-}} \ControlFlowTok{function}\NormalTok{(pedigree, }\AttributeTok{title.append =} \ConstantTok{NULL}\NormalTok{) \{}
  \CommentTok{\#\textquotesingle{} Plot histogram of relatedness from pedigree}
  \CommentTok{\#\textquotesingle{}}
  \CommentTok{\#\textquotesingle{} Compute relatedness matrix from pedigree and output }
  \CommentTok{\#\textquotesingle{} histogram of its off{-}diagonal values.}
  \CommentTok{\#\textquotesingle{} @param pedigree Pedigree dataframe/object}
  \CommentTok{\#\textquotesingle{} @param title.append (optional) extra text in plot title}
  \CommentTok{\#\textquotesingle{} @return ggplot object of histogram}
\NormalTok{  A.matrix }\OtherTok{\textless{}{-}}\NormalTok{ nadiv}\SpecialCharTok{::}\FunctionTok{makeA}\NormalTok{(pedigree)}
\NormalTok{  A.diag }\OtherTok{\textless{}{-}} \FunctionTok{diag}\NormalTok{(A.matrix)}
\NormalTok{  A.nondiag }\OtherTok{\textless{}{-}}\NormalTok{ A.matrix}
  \FunctionTok{diag}\NormalTok{(A.nondiag) }\OtherTok{\textless{}{-}} \ConstantTok{NA}
  \FunctionTok{ggplot}\NormalTok{(}\AttributeTok{data =} \FunctionTok{data.frame}\NormalTok{(}\AttributeTok{values =}\NormalTok{ A.nondiag}\SpecialCharTok{@}\NormalTok{x)) }\SpecialCharTok{+}
    \FunctionTok{geom\_histogram}\NormalTok{(}\FunctionTok{aes}\NormalTok{(}\AttributeTok{x =}\NormalTok{ values, }\AttributeTok{y =}\NormalTok{ ..density..),}
                   \AttributeTok{binwidth =} \FloatTok{0.005}\NormalTok{) }\SpecialCharTok{+}
    \FunctionTok{ylim}\NormalTok{(}\FunctionTok{c}\NormalTok{(}\DecValTok{0}\NormalTok{, }\DecValTok{9}\NormalTok{)) }\SpecialCharTok{+}
    \FunctionTok{xlim}\NormalTok{(}\FunctionTok{c}\NormalTok{(}\DecValTok{0}\NormalTok{, }\FloatTok{0.4}\NormalTok{)) }\SpecialCharTok{+}
    \FunctionTok{labs}\NormalTok{(}
      \AttributeTok{x =} \StringTok{"Relatedness value"}\NormalTok{, }\AttributeTok{y =} \StringTok{"Density"}\NormalTok{,}
      \AttributeTok{title =} \FunctionTok{paste0}\NormalTok{(}\StringTok{"Off{-}diagonal values"}\NormalTok{, title.append)}
\NormalTok{    )}
\NormalTok{\}}

\CommentTok{\# For the simulation data:}
\NormalTok{ped0 }\OtherTok{\textless{}{-}} \FunctionTok{generatePedigree}\NormalTok{(}
  \AttributeTok{nId =} \DecValTok{100}\NormalTok{, }\AttributeTok{nGeneration =} \DecValTok{24}\NormalTok{,}
  \AttributeTok{nFather =} \FloatTok{0.5} \SpecialCharTok{*} \DecValTok{100}\NormalTok{, }\AttributeTok{nMother =} \FloatTok{0.5} \SpecialCharTok{*} \DecValTok{100}
\NormalTok{)}
\NormalTok{pedigree }\OtherTok{\textless{}{-}}\NormalTok{ ped0[, }\FunctionTok{c}\NormalTok{(}\DecValTok{1}\NormalTok{, }\DecValTok{3}\NormalTok{, }\DecValTok{2}\NormalTok{)]}
\FunctionTok{names}\NormalTok{(pedigree) }\OtherTok{\textless{}{-}} \FunctionTok{c}\NormalTok{(}\StringTok{"id"}\NormalTok{, }\StringTok{"dam"}\NormalTok{, }\StringTok{"sire"}\NormalTok{)}
\NormalTok{simulated.d.ped }\OtherTok{\textless{}{-}}\NormalTok{ nadiv}\SpecialCharTok{::}\FunctionTok{prepPed}\NormalTok{(pedigree)}
\FunctionTok{plot.A.matrix}\NormalTok{(simulated.d.ped,}
              \AttributeTok{title.append =} \StringTok{", 24 generation simulation"}\NormalTok{)}
\ControlFlowTok{if}\NormalTok{ (SAVE.PLOT) \{}
  \FunctionTok{ggsave}\NormalTok{(}\StringTok{"../figures/relatedness{-}offdiagonal{-}sim.pdf"}\NormalTok{,}
    \AttributeTok{width =} \DecValTok{20}\NormalTok{, }\AttributeTok{height =} \DecValTok{20}\NormalTok{, }\AttributeTok{units =} \StringTok{"cm"}
\NormalTok{  )}
\NormalTok{\}}
\CommentTok{\# Song sparrow data}
\FunctionTok{plot.A.matrix}\NormalTok{(d.ped[, }\FunctionTok{c}\NormalTok{(}\StringTok{"id"}\NormalTok{, }\StringTok{"mother.id"}\NormalTok{, }\StringTok{"father.id"}\NormalTok{)],}
              \StringTok{", Song sparrow data"}\NormalTok{)}
\ControlFlowTok{if}\NormalTok{ (SAVE.PLOT) \{}
  \FunctionTok{ggsave}\NormalTok{(}\StringTok{"../figures/relatedness{-}offdiagonal{-}songsparrow.pdf"}\NormalTok{,}
    \AttributeTok{width =} \DecValTok{20}\NormalTok{, }\AttributeTok{height =} \DecValTok{20}\NormalTok{, }\AttributeTok{units =} \StringTok{"cm"}
\NormalTok{  )}
\NormalTok{\}}
\end{Highlighting}
\end{Shaded}

\hypertarget{residual-analyses-on-gaussian-model}{%
\section*{Residual analyses on Gaussian
model}\label{residual-analyses-on-gaussian-model}}

\begin{itemize}
\tightlist
\item
  The first plot are the sorted PIT values over quantiles, analogous to
  a Q-Q plot in frequentist data. It shows a clear non-linear trend but
  rather a sigmoid-like curve.
\item
  The second plot shows the PIT values across the different posterior
  fitted value means. Here we expect no clear pattern for well-behaved
  models, which is not the case in our model.
\item
  The third plot is the residuals \(y_i - \hat{y_i}\) with 95\% credible
  interval. Here, we see a clear separation of those
\end{itemize}

\begin{Shaded}
\begin{Highlighting}[]
\NormalTok{pit.g }\OtherTok{\textless{}{-}}\NormalTok{ fit.inla.gaussian}\SpecialCharTok{$}\NormalTok{cpo}\SpecialCharTok{$}\NormalTok{pit }\CommentTok{\# PIT{-}values}

\CommentTok{\# \textless{}Plot 1\textgreater{} Analogous to QQ{-}plot so should be linear}
\CommentTok{\# {-}{-}{-}{-}{-}{-}{-}{-}}
\FunctionTok{ggplot}\NormalTok{(}\AttributeTok{data =} \FunctionTok{data.frame}\NormalTok{(}
  \AttributeTok{Quantiles =} \FunctionTok{seq\_along}\NormalTok{(}\DecValTok{1}\SpecialCharTok{:}\FunctionTok{length}\NormalTok{(pit.g)) }\SpecialCharTok{/}\NormalTok{ (}\FunctionTok{length}\NormalTok{(pit.g) }\SpecialCharTok{+} \DecValTok{1}\NormalTok{),}
  \AttributeTok{PIT =} \FunctionTok{sort}\NormalTok{(pit.g)}
\NormalTok{)) }\SpecialCharTok{+}
  \FunctionTok{geom\_point}\NormalTok{(}\FunctionTok{aes}\NormalTok{(}\AttributeTok{x =}\NormalTok{ Quantiles, }\AttributeTok{y =}\NormalTok{ PIT)) }\SpecialCharTok{+}
  \FunctionTok{ggtitle}\NormalTok{(}\StringTok{"Sorted PIT values for Gaussian model"}\NormalTok{) }\SpecialCharTok{+}
  \FunctionTok{theme}\NormalTok{(}\AttributeTok{text =} \FunctionTok{element\_text}\NormalTok{(}\AttributeTok{size =} \DecValTok{14}\NormalTok{))}
\ControlFlowTok{if}\NormalTok{ (SAVE.PLOT) }\FunctionTok{ggsave}\NormalTok{(}\StringTok{"../figures/PIT{-}sorted.pdf"}\NormalTok{)}

\CommentTok{\# \textless{}Plot 2\textgreater{} Posterior mean fitted values as a function of PIT values}
\CommentTok{\# {-}{-}{-}{-}{-}{-}{-}{-}    analagous to "Residuals vs fitted"}

\FunctionTok{ggplot}\NormalTok{(}
  \FunctionTok{cbind}\NormalTok{(fit.inla.gaussian}\SpecialCharTok{$}\NormalTok{summary.fitted.values, pit.g),}
  \FunctionTok{aes}\NormalTok{(}\AttributeTok{x =}\NormalTok{ mean, }\AttributeTok{y =}\NormalTok{ pit.g)}
\NormalTok{) }\SpecialCharTok{+}
  \FunctionTok{geom\_point}\NormalTok{() }\SpecialCharTok{+}
  \FunctionTok{geom\_smooth}\NormalTok{() }\SpecialCharTok{+}
  \FunctionTok{labs}\NormalTok{(}
    \AttributeTok{title =} \StringTok{"PIT values over posterior mean fitted values"}\NormalTok{,}
    \AttributeTok{x =} \StringTok{"Posterior fitted values (mean)"}\NormalTok{,}
    \AttributeTok{y =} \StringTok{"PIT value"}
\NormalTok{  ) }\SpecialCharTok{+}
  \FunctionTok{theme}\NormalTok{(}\AttributeTok{text =} \FunctionTok{element\_text}\NormalTok{(}\AttributeTok{size =} \DecValTok{14}\NormalTok{))}
\ControlFlowTok{if}\NormalTok{ (SAVE.PLOT) }\FunctionTok{ggsave}\NormalTok{(}\StringTok{"../figures/PIT{-}over{-}fitted.pdf"}\NormalTok{)}


\CommentTok{\# \textless{}Plot 3\textgreater{} Plot of \textquotesingle{}residuals\textquotesingle{}, i.e. difference in true data and the}
\CommentTok{\# {-}{-}{-}{-}{-}{-}{-}{-}    mean of the fitted values}
\NormalTok{df.resid }\OtherTok{\textless{}{-}}\NormalTok{ qg.data.gg.inds}\SpecialCharTok{$}\NormalTok{surv.ind.to.ad }\SpecialCharTok{{-}}
\NormalTok{  fit.inla.gaussian}\SpecialCharTok{$}\NormalTok{summary.fitted.values}
\FunctionTok{rownames}\NormalTok{(df.resid) }\OtherTok{\textless{}{-}} \FunctionTok{seq\_len}\NormalTok{(}\FunctionTok{nrow}\NormalTok{(df.resid))}
\NormalTok{df.resid}\SpecialCharTok{$}\NormalTok{class }\OtherTok{\textless{}{-}}\NormalTok{ qg.data.gg.inds}\SpecialCharTok{$}\NormalTok{surv.ind.to.ad}

\FunctionTok{ggplot}\NormalTok{(}
  \AttributeTok{data =}\NormalTok{ df.resid,}
  \FunctionTok{aes}\NormalTok{(}
    \AttributeTok{x =} \FunctionTok{as.numeric}\NormalTok{(}\FunctionTok{row.names}\NormalTok{(df.resid)),}
    \AttributeTok{y =}\NormalTok{ mean, }\AttributeTok{color =} \FunctionTok{factor}\NormalTok{(class)}
\NormalTok{  )}
\NormalTok{) }\SpecialCharTok{+}
  \FunctionTok{geom\_errorbar}\NormalTok{(}\FunctionTok{aes}\NormalTok{(}\AttributeTok{ymin =} \StringTok{\textasciigrave{}}\AttributeTok{0.025quant}\StringTok{\textasciigrave{}}\NormalTok{, }\AttributeTok{ymax =} \StringTok{\textasciigrave{}}\AttributeTok{0.975quant}\StringTok{\textasciigrave{}}\NormalTok{),}
    \AttributeTok{color =} \StringTok{"darkgrey"}
\NormalTok{  ) }\SpecialCharTok{+}
  \FunctionTok{geom\_point}\NormalTok{() }\SpecialCharTok{+}
  \FunctionTok{scale\_color\_manual}\NormalTok{(}
    \AttributeTok{name =} \StringTok{"Juvenile survival"}\NormalTok{,}
    \AttributeTok{values =} \FunctionTok{c}\NormalTok{(}\StringTok{"darkred"}\NormalTok{, }\StringTok{"steelblue"}\NormalTok{)}
\NormalTok{  ) }\SpecialCharTok{+}
  \FunctionTok{labs}\NormalTok{(}\AttributeTok{title =} \StringTok{"Residuals of Gaussian model"}\NormalTok{, }\AttributeTok{x =} \StringTok{"Index"}\NormalTok{,}
       \AttributeTok{y =} \StringTok{"Residuals"}\NormalTok{) }\SpecialCharTok{+} \FunctionTok{theme}\NormalTok{(}\AttributeTok{text=}\FunctionTok{element\_text}\NormalTok{(}\AttributeTok{size=}\DecValTok{14}\NormalTok{))}
\ControlFlowTok{if}\NormalTok{ (SAVE.PLOT) }\FunctionTok{ggsave}\NormalTok{(}\StringTok{"../figures/Residuals{-}gaussian.pdf"}\NormalTok{)}
\end{Highlighting}
\end{Shaded}

\hypertarget{transformations-of-heritability}{%
\section*{Transformations of
heritability}\label{transformations-of-heritability}}

Before developing methods for transformed heritability, we need to be
able to sample from the marginal fitted values on latent scale.

\begin{Shaded}
\begin{Highlighting}[]
\NormalTok{marginal.latent.mode }\OtherTok{\textless{}{-}} \ControlFlowTok{function}\NormalTok{(fit) \{}
  \CommentTok{\#\textquotesingle{} Helper function}
  \CommentTok{\#\textquotesingle{}}
  \CommentTok{\#\textquotesingle{} Get mode for each marginal linear predictor in \textasciigrave{}fit\textasciigrave{}}
  \CommentTok{\#\textquotesingle{} @param fit Fitted INLA object}
  \CommentTok{\#\textquotesingle{} @return vector of modes}
\NormalTok{  modes }\OtherTok{\textless{}{-}} \FunctionTok{c}\NormalTok{()}
\NormalTok{  iter }\OtherTok{\textless{}{-}} \DecValTok{1}
  \ControlFlowTok{for}\NormalTok{ (predictor }\ControlFlowTok{in} \FunctionTok{names}\NormalTok{(fit}\SpecialCharTok{$}\NormalTok{marginals.linear.predictor)) \{}
\NormalTok{    xy }\OtherTok{\textless{}{-}} \FunctionTok{get}\NormalTok{(predictor, fit}\SpecialCharTok{$}\NormalTok{marginals.linear.predictor)}
\NormalTok{    modes[iter] }\OtherTok{\textless{}{-}}\NormalTok{ xy[, }\StringTok{"x"}\NormalTok{][}\FunctionTok{which.max}\NormalTok{(xy[, }\StringTok{"y"}\NormalTok{])]}
\NormalTok{    iter }\OtherTok{\textless{}{-}}\NormalTok{ iter }\SpecialCharTok{+} \DecValTok{1}
\NormalTok{  \}}
\NormalTok{  modes}
\NormalTok{\}}


\NormalTok{marginal.latent.samples }\OtherTok{\textless{}{-}} \ControlFlowTok{function}\NormalTok{(fit, nsamples) \{}
  \CommentTok{\#\textquotesingle{} Sample values from each predictor in fit}
  \CommentTok{\#\textquotesingle{}}
  \CommentTok{\#\textquotesingle{} Rather than only using mode for each predictor, we use samples}
  \CommentTok{\#\textquotesingle{} from its posterior.}
  \CommentTok{\#\textquotesingle{} @param fit Fitted INLA object}
  \CommentTok{\#\textquotesingle{} @param nsamples Number of samples}
  \CommentTok{\#\textquotesingle{} @return A list of \textasciigrave{}nsamples\textasciigrave{} elements, with each element in the}
  \CommentTok{\#\textquotesingle{} list being a vector of the predictor size}
  \CommentTok{\#\textquotesingle{} (i.e., number of observations in data)}
\NormalTok{  out.transpose }\OtherTok{\textless{}{-}} \FunctionTok{matrix}\NormalTok{(}
    \AttributeTok{nrow =}\NormalTok{ nsamples,}
    \AttributeTok{ncol =} \FunctionTok{length}\NormalTok{(fit}\SpecialCharTok{$}\NormalTok{marginals.linear.predictor)}
\NormalTok{  )}
  \ControlFlowTok{for}\NormalTok{ (i }\ControlFlowTok{in} \FunctionTok{seq\_along}\NormalTok{(fit}\SpecialCharTok{$}\NormalTok{marginals.linear.predictor)) \{}
\NormalTok{    xy }\OtherTok{\textless{}{-}} \FunctionTok{get}\NormalTok{(}
      \FunctionTok{names}\NormalTok{(fit}\SpecialCharTok{$}\NormalTok{marginals.linear.predictor)[i],}
\NormalTok{      fit}\SpecialCharTok{$}\NormalTok{marginals.linear.predictor}
\NormalTok{    )}
\NormalTok{    out.transpose[, i] }\OtherTok{\textless{}{-}} \FunctionTok{inla.rmarginal}\NormalTok{(nsamples, xy)}
\NormalTok{  \}}

  \CommentTok{\# We want list where each list element is one sample (transposed)}
\NormalTok{  out }\OtherTok{\textless{}{-}} \FunctionTok{list}\NormalTok{()}
  \ControlFlowTok{for}\NormalTok{ (i }\ControlFlowTok{in} \DecValTok{1}\SpecialCharTok{:}\NormalTok{nsamples) \{}
\NormalTok{    out[[i]] }\OtherTok{\textless{}{-}}\NormalTok{ out.transpose[i, ]}
\NormalTok{  \}}
\NormalTok{  out}
\NormalTok{\}}


\NormalTok{report.max.skewness }\OtherTok{\textless{}{-}} \ControlFlowTok{function}\NormalTok{(posterior) \{}
  \CommentTok{\#\textquotesingle{} Get skewness for all predictors}
  \CommentTok{\#\textquotesingle{}}
  \CommentTok{\#\textquotesingle{} Computes skewness and prints maximum and minimum skew with index}
  \CommentTok{\#\textquotesingle{} @param posterior list of predictors, assumed form [x, y]}
  \CommentTok{\#\textquotesingle{} @return None (invisible \textasciigrave{}NULL\textasciigrave{})}
  \FunctionTok{library}\NormalTok{(e1071)}
\NormalTok{  iter }\OtherTok{\textless{}{-}} \DecValTok{1}
\NormalTok{  posterior.skews }\OtherTok{\textless{}{-}} \FunctionTok{c}\NormalTok{()}
  \ControlFlowTok{for}\NormalTok{ (predictor }\ControlFlowTok{in} \FunctionTok{names}\NormalTok{(posterior)) \{}
\NormalTok{    posterior.skews[iter] }\OtherTok{\textless{}{-}} \FunctionTok{skewness}\NormalTok{(}\FunctionTok{get}\NormalTok{(}
\NormalTok{      predictor, posterior)[, }\StringTok{"x"}\NormalTok{])}
\NormalTok{    iter }\OtherTok{\textless{}{-}}\NormalTok{ iter }\SpecialCharTok{+} \DecValTok{1}
\NormalTok{  \}}
  \FunctionTok{cat}\NormalTok{(}
    \StringTok{"Minimum skew for list no."}\NormalTok{, }\FunctionTok{which.min}\NormalTok{(posterior.skews), }
  \StringTok{"with skewness"}\NormalTok{,}
    \FunctionTok{min}\NormalTok{(posterior.skews), }\StringTok{"and max for list no."}\NormalTok{,}
  \FunctionTok{which.max}\NormalTok{(posterior.skews),}
    \StringTok{"with skewness"}\NormalTok{, }\FunctionTok{max}\NormalTok{(posterior.skews), }\StringTok{".}\SpecialCharTok{\textbackslash{}n}\StringTok{"}
\NormalTok{  )}
\NormalTok{\}}
\end{Highlighting}
\end{Shaded}

We can now define methods to obtain heritability on the different
scales. The first function computes \(h^2\) on latent scale, or using
the direct transformation by including link variance in denominator. The
second method is more comprehensive and uses the library \texttt{QGglmm}
to obtain estimates on data scale. So far we've only used
\texttt{QGglmm} without averaging over fixed effects. This takes
considerably more time to process, but we still do that one time to
compare the results. Then we compare the heritability on four different
scales

\begin{itemize}
\tightlist
\item
  Using 10k samples from the \emph{marginal linear predictor}, and using
  this to average over
\item
  Grabbing the mode for each \emph{marginal linear predictor}, and
  passing the mode for each 10k sample
\item
  No averaging, i.e.~use intercept value instead.
\item
  Use direct scaling method with link variance.
\end{itemize}

\begin{Shaded}
\begin{Highlighting}[]
\NormalTok{get.h2.from.qgparams }\OtherTok{\textless{}{-}} \ControlFlowTok{function}\NormalTok{(inla.fit,}
\NormalTok{                                 modelname,}
\NormalTok{                                 n,}
                                 \AttributeTok{averaging =} \ConstantTok{FALSE}\NormalTok{,}
                                 \AttributeTok{averaging.mode.only =} \ConstantTok{FALSE}\NormalTok{) \{}
  \CommentTok{\#\textquotesingle{} Get heritability using QGglmm::QGParams()}
  \CommentTok{\#\textquotesingle{}}
  \CommentTok{\#\textquotesingle{} Computes a posterior of data{-}scale heritability using QGParams}
  \CommentTok{\#\textquotesingle{} @param inla.fit Fitted INLA model}
  \CommentTok{\#\textquotesingle{} @param modelname string specifying model type}
  \CommentTok{\#\textquotesingle{} @param n Number of samples for posterior distribution}
  \CommentTok{\#\textquotesingle{} @param averaging Flag for averaging over fixed effects}
  \CommentTok{\#\textquotesingle{} @param averaging.mode.only Other flag to determine what helper}
  \CommentTok{\#\textquotesingle{} to call to.}
  \CommentTok{\#\textquotesingle{} @return List of n observation{-}scale heritabilities}

  \FunctionTok{stopifnot}\NormalTok{(modelname }\SpecialCharTok{\%in\%} \FunctionTok{c}\NormalTok{(}\StringTok{"Gaussian"}\NormalTok{, }\StringTok{"binom1.probit"}\NormalTok{,}
                             \StringTok{"binom1.logit"}\NormalTok{))}
\NormalTok{  samples.posterior }\OtherTok{\textless{}{-}} \FunctionTok{inla.hyperpar.sample}\NormalTok{(}\AttributeTok{n =}\NormalTok{ n, inla.fit)}
\NormalTok{  vp.samples }\OtherTok{\textless{}{-}} \DecValTok{0}
  \ControlFlowTok{for}\NormalTok{ (cname }\ControlFlowTok{in} \FunctionTok{colnames}\NormalTok{(samples.posterior)) \{}
\NormalTok{    vp.samples }\OtherTok{\textless{}{-}}\NormalTok{ vp.samples }\SpecialCharTok{+} \DecValTok{1} \SpecialCharTok{/}\NormalTok{ samples.posterior[, cname]}
\NormalTok{  \}}

  \ControlFlowTok{if}\NormalTok{ (}\SpecialCharTok{!}\NormalTok{averaging) \{}
\NormalTok{    mu }\OtherTok{\textless{}{-}}\NormalTok{ inla.fit}\SpecialCharTok{$}\NormalTok{summary.fixed}\SpecialCharTok{$}\NormalTok{mean[}\DecValTok{1}\NormalTok{] }\CommentTok{\# Intercept}
\NormalTok{    va.samples }\OtherTok{\textless{}{-}} \DecValTok{1} \SpecialCharTok{/}\NormalTok{ samples.posterior[, }\StringTok{"Precision for id"}\NormalTok{]}
\NormalTok{    kwargs }\OtherTok{\textless{}{-}} \FunctionTok{list}\NormalTok{(}\AttributeTok{verbose =} \ConstantTok{FALSE}\NormalTok{)}

\NormalTok{    h2.getter }\OtherTok{\textless{}{-}} \ControlFlowTok{function}\NormalTok{(...) \{}
      \FunctionTok{get}\NormalTok{(}\StringTok{"h2.obs"}\NormalTok{, }\FunctionTok{suppressWarnings}\NormalTok{(}\FunctionTok{QGparams}\NormalTok{(...)))}
\NormalTok{    \}}
\NormalTok{    posterior }\OtherTok{\textless{}{-}} \FunctionTok{mapply}\NormalTok{(h2.getter, mu, va.samples, vp.samples, }
\NormalTok{                        modelname, }\AttributeTok{MoreArgs =}\NormalTok{ kwargs}
\NormalTok{    )}
    \FunctionTok{return}\NormalTok{(posterior)}
\NormalTok{  \} }\ControlFlowTok{else}\NormalTok{ \{}
    \CommentTok{\# Average over fixed effects}
\NormalTok{    vp.samples }\OtherTok{\textless{}{-}} \DecValTok{0}
\NormalTok{    df }\OtherTok{\textless{}{-}} \FunctionTok{data.frame}\NormalTok{(}
      \AttributeTok{va =} \FunctionTok{as.vector}\NormalTok{(}\DecValTok{1} \SpecialCharTok{/}\NormalTok{ samples.posterior[, }\StringTok{"Precision for id"}\NormalTok{]),}
      \AttributeTok{vp =} \FunctionTok{as.vector}\NormalTok{(vp.samples)}
\NormalTok{    )}
    \ControlFlowTok{if}\NormalTok{ (}\SpecialCharTok{!}\NormalTok{averaging.mode.only) \{}
      \CommentTok{\# Bayesian approach}
\NormalTok{      df}\SpecialCharTok{$}\NormalTok{predict }\OtherTok{\textless{}{-}} \FunctionTok{marginal.latent.samples}\NormalTok{(inla.fit, n)}

\NormalTok{      posterior }\OtherTok{\textless{}{-}} \FunctionTok{do.call}\NormalTok{(}\StringTok{"rbind"}\NormalTok{, }\FunctionTok{apply}\NormalTok{(df, }\DecValTok{1}\NormalTok{, }\ControlFlowTok{function}\NormalTok{(row) \{}
        \FunctionTok{QGparams}\NormalTok{(}
          \AttributeTok{predict =}\NormalTok{ row[[}\StringTok{"predict"}\NormalTok{]], }\AttributeTok{var.a =}\NormalTok{ row[[}\StringTok{"va"}\NormalTok{]],}
          \AttributeTok{var.p =}\NormalTok{ row[[}\StringTok{"vp"}\NormalTok{]],}
          \AttributeTok{model =}\NormalTok{ modelname, }\AttributeTok{verbose =} \ConstantTok{FALSE}
\NormalTok{        )}
\NormalTok{      \}))}
\NormalTok{    \} }\ControlFlowTok{else}\NormalTok{ \{}
\NormalTok{      predict.argument }\OtherTok{\textless{}{-}} \FunctionTok{marginal.latent.mode}\NormalTok{(inla.fit)}
\NormalTok{      posterior }\OtherTok{\textless{}{-}} \FunctionTok{do.call}\NormalTok{(}\StringTok{"rbind"}\NormalTok{, }\FunctionTok{apply}\NormalTok{(df, }\DecValTok{1}\NormalTok{, }\ControlFlowTok{function}\NormalTok{(row) \{}
        \FunctionTok{QGparams}\NormalTok{(}
          \AttributeTok{predict =}\NormalTok{ predict.argument, }\AttributeTok{var.a =}\NormalTok{ row[[}\StringTok{"va"}\NormalTok{]],}
          \AttributeTok{var.p =}\NormalTok{ row[[}\StringTok{"vp"}\NormalTok{]],}
          \AttributeTok{model =}\NormalTok{ modelname, }\AttributeTok{verbose =} \ConstantTok{FALSE}
\NormalTok{        )}
\NormalTok{      \}))}
\NormalTok{    \}}
    \FunctionTok{return}\NormalTok{(posterior}\SpecialCharTok{$}\NormalTok{h2.obs)}
\NormalTok{  \}}
\NormalTok{\}}
\end{Highlighting}
\end{Shaded}

Compute the 3 different approaches with \texttt{QGglmm} (this is quite
slow). We also include time estimates here.

\begin{Shaded}
\begin{Highlighting}[]
\NormalTok{ti }\OtherTok{\textless{}{-}} \FunctionTok{Sys.time}\NormalTok{()}
\NormalTok{h2.psi.sparrow }\OtherTok{\textless{}{-}} \FunctionTok{data.frame}\NormalTok{(}
  \AttributeTok{bayesian =}
    \FunctionTok{get.h2.from.qgparams}\NormalTok{(fit.inla.probit, }\StringTok{"binom1.probit"}\NormalTok{,}
\NormalTok{      n.samples,}
      \AttributeTok{averaging =} \ConstantTok{TRUE}\NormalTok{,}
      \AttributeTok{averaging.mode.only =} \ConstantTok{FALSE}
\NormalTok{    )}
\NormalTok{)}
\FunctionTok{cat}\NormalTok{(}
  \StringTok{"}\SpecialCharTok{\textbackslash{}n}\StringTok{{-}}\SpecialCharTok{\textbackslash{}n}\StringTok{Runtime for Bayesian:"}\NormalTok{,}
  \FunctionTok{difftime}\NormalTok{(}\FunctionTok{Sys.time}\NormalTok{(), ti, }\AttributeTok{units =} \StringTok{"secs"}\NormalTok{), }\StringTok{"secs.}\SpecialCharTok{\textbackslash{}n}\StringTok{"}
\NormalTok{)}
\NormalTok{ti }\OtherTok{\textless{}{-}} \FunctionTok{Sys.time}\NormalTok{()}
\NormalTok{h2.psi.sparrow}\SpecialCharTok{$}\NormalTok{frequentist }\OtherTok{\textless{}{-}} \FunctionTok{get.h2.from.qgparams}\NormalTok{(fit.inla.probit,}
  \StringTok{"binom1.probit"}\NormalTok{,}
\NormalTok{  n.samples,}
  \AttributeTok{averaging =} \ConstantTok{TRUE}\NormalTok{,}
  \AttributeTok{averaging.mode.only =} \ConstantTok{TRUE}
\NormalTok{)}
\FunctionTok{cat}\NormalTok{(}
  \StringTok{"}\SpecialCharTok{\textbackslash{}n}\StringTok{{-}}\SpecialCharTok{\textbackslash{}n}\StringTok{Runtime for Frequentist:"}\NormalTok{,}
  \FunctionTok{difftime}\NormalTok{(}\FunctionTok{Sys.time}\NormalTok{(), ti, }\AttributeTok{units =} \StringTok{"secs"}\NormalTok{), }\StringTok{"secs.}\SpecialCharTok{\textbackslash{}n}\StringTok{"}
\NormalTok{)}
\NormalTok{ti }\OtherTok{\textless{}{-}} \FunctionTok{Sys.time}\NormalTok{()}
\NormalTok{h2.psi.sparrow}\SpecialCharTok{$}\NormalTok{noavg }\OtherTok{\textless{}{-}} \FunctionTok{get.h2.from.qgparams}\NormalTok{(fit.inla.probit,}
  \StringTok{"binom1.probit"}\NormalTok{,}
\NormalTok{  n.samples,}
  \AttributeTok{averaging =} \ConstantTok{FALSE}
\NormalTok{)}
\FunctionTok{cat}\NormalTok{(}
  \StringTok{"}\SpecialCharTok{\textbackslash{}n}\StringTok{{-}}\SpecialCharTok{\textbackslash{}n}\StringTok{Runtime for No averaging:"}\NormalTok{,}
  \FunctionTok{difftime}\NormalTok{(}\FunctionTok{Sys.time}\NormalTok{(), ti, }\AttributeTok{units =} \StringTok{"secs"}\NormalTok{), }\StringTok{"secs.}\SpecialCharTok{\textbackslash{}n}\StringTok{"}
\NormalTok{)}
\end{Highlighting}
\end{Shaded}

We do the same for simulation data

\begin{Shaded}
\begin{Highlighting}[]
\CommentTok{\# We store an instance of a gaussian and probit model with V\_A = 1}
\NormalTok{tmp }\OtherTok{\textless{}{-}} \FunctionTok{simulated.heritability}\NormalTok{(}
  \AttributeTok{sigmaA =} \DecValTok{1}\NormalTok{, }\AttributeTok{linear.predictor =} \ControlFlowTok{function}\NormalTok{(u, .) u }\SpecialCharTok{+} \FunctionTok{rnorm}\NormalTok{(}\FunctionTok{length}\NormalTok{(u)),}
  \AttributeTok{simulated.formula =}\NormalTok{ simulated.formula,}
  \AttributeTok{probit.model =} \ConstantTok{TRUE}\NormalTok{,}
  \AttributeTok{simulated.formula.probit =}\NormalTok{ simulated.formula.probit}
\NormalTok{)}
\NormalTok{fit.sim.probit }\OtherTok{\textless{}{-}}\NormalTok{ tmp}\SpecialCharTok{$}\NormalTok{fit.probit}

\NormalTok{ti }\OtherTok{\textless{}{-}} \FunctionTok{Sys.time}\NormalTok{()}
\NormalTok{h2.psi.sim1 }\OtherTok{\textless{}{-}} \FunctionTok{data.frame}\NormalTok{(}
  \AttributeTok{bayesian =}
    \FunctionTok{get.h2.from.qgparams}\NormalTok{(fit.sim.probit, }\StringTok{"binom1.probit"}\NormalTok{,}
\NormalTok{      n.samples,}
      \AttributeTok{averaging =} \ConstantTok{TRUE}\NormalTok{,}
      \AttributeTok{averaging.mode.only =} \ConstantTok{FALSE}
\NormalTok{    )}
\NormalTok{)}
\FunctionTok{cat}\NormalTok{(}
  \StringTok{"}\SpecialCharTok{\textbackslash{}n}\StringTok{{-}}\SpecialCharTok{\textbackslash{}n}\StringTok{Runtime for Bayesian:"}\NormalTok{,}
  \FunctionTok{difftime}\NormalTok{(}\FunctionTok{Sys.time}\NormalTok{(), ti, }\AttributeTok{units =} \StringTok{"secs"}\NormalTok{), }\StringTok{".}\SpecialCharTok{\textbackslash{}n}\StringTok{"}
\NormalTok{)}
\NormalTok{ti }\OtherTok{\textless{}{-}} \FunctionTok{Sys.time}\NormalTok{()}
\NormalTok{h2.psi.sim1}\SpecialCharTok{$}\NormalTok{frequentist }\OtherTok{\textless{}{-}} \FunctionTok{get.h2.from.qgparams}\NormalTok{(fit.sim.probit,}
  \StringTok{"binom1.probit"}\NormalTok{,}
\NormalTok{  n.samples,}
  \AttributeTok{averaging =} \ConstantTok{TRUE}\NormalTok{,}
  \AttributeTok{averaging.mode.only =} \ConstantTok{TRUE}
\NormalTok{)}
\FunctionTok{cat}\NormalTok{(}
  \StringTok{"}\SpecialCharTok{\textbackslash{}n}\StringTok{{-}}\SpecialCharTok{\textbackslash{}n}\StringTok{Runtime for Frequentist:"}\NormalTok{,}
  \FunctionTok{difftime}\NormalTok{(}\FunctionTok{Sys.time}\NormalTok{(), ti, }\AttributeTok{units =} \StringTok{"secs"}\NormalTok{), }\StringTok{".}\SpecialCharTok{\textbackslash{}n}\StringTok{"}
\NormalTok{)}
\NormalTok{ti }\OtherTok{\textless{}{-}} \FunctionTok{Sys.time}\NormalTok{()}
\NormalTok{h2.psi.sim1}\SpecialCharTok{$}\NormalTok{noavg }\OtherTok{\textless{}{-}} \FunctionTok{get.h2.from.qgparams}\NormalTok{(fit.sim.probit,}
  \StringTok{"binom1.probit"}\NormalTok{,}
\NormalTok{  n.samples,}
  \AttributeTok{averaging =} \ConstantTok{FALSE}
\NormalTok{)}
\FunctionTok{cat}\NormalTok{(}
  \StringTok{"}\SpecialCharTok{\textbackslash{}n}\StringTok{{-}}\SpecialCharTok{\textbackslash{}n}\StringTok{Runtime for No averaging:"}\NormalTok{,}
  \FunctionTok{difftime}\NormalTok{(}\FunctionTok{Sys.time}\NormalTok{(), ti, }\AttributeTok{units =} \StringTok{"secs"}\NormalTok{), }\StringTok{".}\SpecialCharTok{\textbackslash{}n}\StringTok{"}
\NormalTok{)}

\CommentTok{\# Also fit for smaller V\_A, i.e. 0.1}
\NormalTok{tmp }\OtherTok{\textless{}{-}} \FunctionTok{simulated.heritability}\NormalTok{(}
  \AttributeTok{sigmaA =} \FloatTok{0.1}\NormalTok{,}
  \AttributeTok{linear.predictor =} \ControlFlowTok{function}\NormalTok{(u, .) u }\SpecialCharTok{+} \FunctionTok{rnorm}\NormalTok{(}\FunctionTok{length}\NormalTok{(u)),}
  \AttributeTok{simulated.formula =}\NormalTok{ simulated.formula,}
  \AttributeTok{probit.model =} \ConstantTok{TRUE}\NormalTok{,}
  \AttributeTok{simulated.formula.probit =}\NormalTok{ simulated.formula.probit}
\NormalTok{)}

\NormalTok{fit.sim.probit }\OtherTok{\textless{}{-}}\NormalTok{ tmp}\SpecialCharTok{$}\NormalTok{fit.probit}

\NormalTok{h2.psi.sim2 }\OtherTok{\textless{}{-}} \FunctionTok{data.frame}\NormalTok{(}
  \AttributeTok{bayesian =}
    \FunctionTok{get.h2.from.qgparams}\NormalTok{(fit.sim.probit, }\StringTok{"binom1.probit"}\NormalTok{,}
\NormalTok{      n.samples,}
      \AttributeTok{averaging =} \ConstantTok{TRUE}\NormalTok{,}
      \AttributeTok{averaging.mode.only =} \ConstantTok{FALSE}
\NormalTok{    )}
\NormalTok{)}
\NormalTok{h2.psi.sim2}\SpecialCharTok{$}\NormalTok{frequentist }\OtherTok{\textless{}{-}} \FunctionTok{get.h2.from.qgparams}\NormalTok{(fit.sim.probit,}
  \StringTok{"binom1.probit"}\NormalTok{,}
\NormalTok{  n.samples,}
  \AttributeTok{averaging =} \ConstantTok{TRUE}\NormalTok{,}
  \AttributeTok{averaging.mode.only =} \ConstantTok{TRUE}
\NormalTok{)}
\NormalTok{h2.psi.sim2}\SpecialCharTok{$}\NormalTok{noavg }\OtherTok{\textless{}{-}} \FunctionTok{get.h2.from.qgparams}\NormalTok{(fit.sim.probit,}
  \StringTok{"binom1.probit"}\NormalTok{,}
\NormalTok{  n.samples,}
  \AttributeTok{averaging =} \ConstantTok{FALSE}
\NormalTok{)}
\end{Highlighting}
\end{Shaded}

\begin{Shaded}
\begin{Highlighting}[]
\NormalTok{plot.qgglmm.heritability }\OtherTok{\textless{}{-}} \ControlFlowTok{function}\NormalTok{(h2.psi, dataset, SAVE.PLOT,}
                                     \AttributeTok{plot.title =} \ConstantTok{NA}\NormalTok{,}
                                     \AttributeTok{fn.append =} \ConstantTok{NULL}\NormalTok{) \{}
  \CommentTok{\#\textquotesingle{} Plot heriability density}
  \CommentTok{\#\textquotesingle{}}
  \CommentTok{\#\textquotesingle{} Compares posterior heritability using different transformations}
  \CommentTok{\#\textquotesingle{} @param h2.psi Dataframe of n rows and a column for each}
  \CommentTok{\#\textquotesingle{} back{-}transformation technique}
  \CommentTok{\#\textquotesingle{} (bayesian, frequentist, no averaging, phi).}
  \CommentTok{\#\textquotesingle{} @param dataset Either \textquotesingle{}application\textquotesingle{} or \textquotesingle{}simulation\textquotesingle{} specifying}
  \CommentTok{\#\textquotesingle{} which dataset is used}
  \CommentTok{\#\textquotesingle{} @param SAVE.PLOT Flag to store plot to disk}
  \CommentTok{\#\textquotesingle{} @param plot.title (optional) title for plot. No title if unused.}
  \CommentTok{\#\textquotesingle{} @param fn.append (optional) string to append to filename}
  \CommentTok{\#\textquotesingle{} @return ggplot object}
\NormalTok{  color.map }\OtherTok{\textless{}{-}} \FunctionTok{c}\NormalTok{(}\AttributeTok{application =} \StringTok{"Dark2"}\NormalTok{, }\AttributeTok{simulation =} \StringTok{"Spectral"}\NormalTok{)}
  \FunctionTok{stopifnot}\NormalTok{(dataset }\SpecialCharTok{\%in\%} \FunctionTok{names}\NormalTok{(color.map))}
\NormalTok{  p }\OtherTok{\textless{}{-}} \FunctionTok{ggplot}\NormalTok{(}\AttributeTok{data =} \FunctionTok{melt}\NormalTok{(h2.psi)) }\SpecialCharTok{+}
    \FunctionTok{geom\_density}\NormalTok{(}\FunctionTok{aes}\NormalTok{(}\AttributeTok{x =}\NormalTok{ value, }\AttributeTok{fill =}\NormalTok{ variable), }\AttributeTok{alpha =} \FloatTok{0.5}\NormalTok{) }\SpecialCharTok{+}
    \FunctionTok{scale\_fill\_brewer}\NormalTok{(}
      \AttributeTok{palette =}\NormalTok{ color.map[dataset],}
      \AttributeTok{labels =} \FunctionTok{c}\NormalTok{(}
        \FunctionTok{expression}\NormalTok{(h[Psi]}\SpecialCharTok{\^{}}\DecValTok{2} \SpecialCharTok{*} \StringTok{", Bayesian"}\NormalTok{),}
        \FunctionTok{expression}\NormalTok{(h[Psi]}\SpecialCharTok{\^{}}\DecValTok{2} \SpecialCharTok{*} \StringTok{", Frequentist"}\NormalTok{),}
        \FunctionTok{expression}\NormalTok{(h[Psi]}\SpecialCharTok{\^{}}\DecValTok{2} \SpecialCharTok{*} \StringTok{", No averaging"}\NormalTok{)}
\NormalTok{      )}
\NormalTok{    ) }\SpecialCharTok{+}
    \FunctionTok{ylab}\NormalTok{(}\StringTok{"Density"}\NormalTok{) }\SpecialCharTok{+}
    \FunctionTok{xlab}\NormalTok{(}\StringTok{"Heritability"}\NormalTok{) }\SpecialCharTok{+}
    \FunctionTok{theme}\NormalTok{(}\AttributeTok{legend.text.align =} \DecValTok{0}\NormalTok{, }\AttributeTok{legend.title =} \FunctionTok{element\_blank}\NormalTok{()) }\SpecialCharTok{+}
\NormalTok{    \{}
      \ControlFlowTok{if}\NormalTok{ (}\SpecialCharTok{!}\FunctionTok{is.na}\NormalTok{(plot.title)) }\FunctionTok{ggtitle}\NormalTok{(plot.title)}
\NormalTok{    \} }\SpecialCharTok{+}
\NormalTok{    \{}
      \ControlFlowTok{if}\NormalTok{ (dataset }\SpecialCharTok{==} \StringTok{"simulation"}\NormalTok{) }\FunctionTok{xlim}\NormalTok{(}\FunctionTok{c}\NormalTok{(}\DecValTok{0}\NormalTok{,}\FunctionTok{quantile}\NormalTok{(}
        \FunctionTok{melt}\NormalTok{(h2.psi)}\SpecialCharTok{$}\NormalTok{value,}\FloatTok{0.99}\NormalTok{)))}
\NormalTok{    \}}

  \ControlFlowTok{if}\NormalTok{ (SAVE.PLOT) \{}
    \FunctionTok{set\_null\_device}\NormalTok{(cairo\_pdf)}
\NormalTok{    p.legend }\OtherTok{\textless{}{-}}\NormalTok{ cowplot}\SpecialCharTok{::}\FunctionTok{get\_legend}\NormalTok{(p)}
    \FunctionTok{pdf}\NormalTok{(}\FunctionTok{paste0}\NormalTok{(}\StringTok{"../figures/qgglmm{-}comparison{-}"}\NormalTok{, dataset,}
               \StringTok{"{-}legend.pdf"}\NormalTok{),}
      \AttributeTok{width =} \DecValTok{3}\NormalTok{, }\AttributeTok{height =} \DecValTok{3}
\NormalTok{    )}
    \FunctionTok{grid.newpage}\NormalTok{()}
    \FunctionTok{grid.draw}\NormalTok{(p.legend)}
    \FunctionTok{dev.off}\NormalTok{()}
    \FunctionTok{ggsave}\NormalTok{(}
      \FunctionTok{paste0}\NormalTok{(}
        \StringTok{"../figures/qgglmm{-}comparison{-}"}\NormalTok{,}
\NormalTok{        dataset, fn.append, }\StringTok{".pdf"}
\NormalTok{      ),}
\NormalTok{      p }\SpecialCharTok{+} \ControlFlowTok{if}\NormalTok{ (dataset }\SpecialCharTok{==} \StringTok{"simulation"}\NormalTok{) }\FunctionTok{theme}\NormalTok{(}\AttributeTok{legend.position =} \StringTok{"none"}\NormalTok{),}
      \AttributeTok{width =} \DecValTok{20}\NormalTok{, }\AttributeTok{height =} \DecValTok{10}\NormalTok{, }\AttributeTok{units =} \StringTok{"cm"}
\NormalTok{    )}
\NormalTok{  \}}
\NormalTok{  p}
\NormalTok{\}}

\FunctionTok{plot.qgglmm.heritability}\NormalTok{(h2.psi.sparrow, }\StringTok{"application"}\NormalTok{,}
\NormalTok{  SAVE.PLOT,}
  \AttributeTok{plot.title =} \ConstantTok{NA} 
\NormalTok{)}
\FunctionTok{plot.qgglmm.heritability}\NormalTok{(h2.psi.sim1, }\StringTok{"simulation"}\NormalTok{,}
\NormalTok{  SAVE.PLOT,}
  \AttributeTok{plot.title =} \ConstantTok{NA}\NormalTok{, }
  \AttributeTok{fn.append =} \StringTok{"va1"}
\NormalTok{)}
\FunctionTok{plot.qgglmm.heritability}\NormalTok{(h2.psi.sim2, }\StringTok{"simulation"}\NormalTok{,}
\NormalTok{  SAVE.PLOT,}
  \AttributeTok{plot.title =} \ConstantTok{NA}\NormalTok{, }
  \AttributeTok{fn.append =} \StringTok{"va0.1"}
\NormalTok{)}
\end{Highlighting}
\end{Shaded}

\hypertarget{compare-different-scales}{%
\subsection*{Compare different scales}\label{compare-different-scales}}

First, we fit the simulation probit model and get simulation-based
heritability on all scales. Now we compute the heritability on the
different scales - for song sparrow data. First we make a function to
help us obtain all heritability scales in a dataframe.

\begin{Shaded}
\begin{Highlighting}[]
\NormalTok{get.all.heritabilities }\OtherTok{\textless{}{-}} \ControlFlowTok{function}\NormalTok{(fit.gaussian, fit.probit, p, n,}
                                   \AttributeTok{fixed =} \ConstantTok{FALSE}\NormalTok{) \{}
  \CommentTok{\#\textquotesingle{} Get h\^{}2 for all scales}
  \CommentTok{\#\textquotesingle{}}
  \CommentTok{\#\textquotesingle{} For a Gaussian and probit fit, computes heritability on all scales}
  \CommentTok{\#\textquotesingle{} @param fit.gaussian Fitted Gaussian model}
  \CommentTok{\#\textquotesingle{} @param fit.probit Fitted Probit model}
  \CommentTok{\#\textquotesingle{} @param p Phenotypic mean for the data, used in threshold formula}
  \CommentTok{\#\textquotesingle{} @param n Number of samples}
  \CommentTok{\#\textquotesingle{} @param fixed Flag for including fixed effects variance}
  \CommentTok{\#\textquotesingle{} @return Dataframe of \textasciigrave{}n\textasciigrave{} rows with columns \textquotesingle{}gaussian\textquotesingle{},}
  \CommentTok{\#\textquotesingle{} \textquotesingle{}guassian.liability\textquotesingle{}, \textquotesingle{}probit.latent\textquotesingle{}, \textquotesingle{}probit.scaled\textquotesingle{},}
  \CommentTok{\#\textquotesingle{} \textquotesingle{}probit.qgglmm\textquotesingle{}}
\NormalTok{  out }\OtherTok{\textless{}{-}} \FunctionTok{data.frame}\NormalTok{(}\AttributeTok{gaussian =} \FunctionTok{get.h2}\NormalTok{(fit.gaussian, n,}
                                      \AttributeTok{include.fixed =}\NormalTok{ fixed))}
\NormalTok{  out}\SpecialCharTok{$}\NormalTok{gaussian.liability }\OtherTok{\textless{}{-}} \FunctionTok{threshold.scaling.param}\NormalTok{(p) }\SpecialCharTok{*}\NormalTok{ out}\SpecialCharTok{$}\NormalTok{gaussian}
\NormalTok{  out}\SpecialCharTok{$}\NormalTok{probit.latent }\OtherTok{\textless{}{-}} \FunctionTok{get.h2}\NormalTok{(fit.probit, n, }\AttributeTok{include.fixed =}\NormalTok{ fixed)}
\NormalTok{  out}\SpecialCharTok{$}\NormalTok{probit.scaled }\OtherTok{\textless{}{-}} \FunctionTok{get.h2}\NormalTok{(fit.probit, n,}
    \AttributeTok{model =} \StringTok{"binom1.probit"}\NormalTok{,}
    \AttributeTok{use.scale =} \ConstantTok{TRUE}\NormalTok{, }\AttributeTok{include.fixed =}\NormalTok{ fixed}
\NormalTok{  )}
\NormalTok{  out}\SpecialCharTok{$}\NormalTok{probit.qgglmm }\OtherTok{\textless{}{-}} \FunctionTok{get.h2.from.qgparams}\NormalTok{(fit.probit,}
                                            \StringTok{"binom1.probit"}\NormalTok{, n,}
    \AttributeTok{averaging =} \ConstantTok{TRUE}
\NormalTok{  )}
\NormalTok{  out}
\NormalTok{\}}
\end{Highlighting}
\end{Shaded}

\begin{Shaded}
\begin{Highlighting}[]
\CommentTok{\# Application data}
\NormalTok{heritability }\OtherTok{\textless{}{-}} \FunctionTok{get.all.heritabilities}\NormalTok{(}
\NormalTok{  fit.inla.gaussian, fit.inla.probit,}
  \FunctionTok{mean}\NormalTok{(qg.data.gg.inds}\SpecialCharTok{$}\NormalTok{surv.ind.to.ad),}
\NormalTok{  n.samples,}
  \AttributeTok{fixed =} \ConstantTok{FALSE}
\NormalTok{)}
\end{Highlighting}
\end{Shaded}

\begin{Shaded}
\begin{Highlighting}[]
\NormalTok{simulation.res2 }\OtherTok{\textless{}{-}} \FunctionTok{simulated.heritability}\NormalTok{(}\FloatTok{0.5}\NormalTok{, }\DecValTok{100}\NormalTok{, }\DecValTok{9}\NormalTok{,}
  \AttributeTok{sigmaA =} \FloatTok{0.5}\NormalTok{,}
  \AttributeTok{linear.predictor =} \ControlFlowTok{function}\NormalTok{(u, .) u }\SpecialCharTok{+} \FunctionTok{rnorm}\NormalTok{(}\FunctionTok{length}\NormalTok{(u)),}
  \AttributeTok{simulated.formula =}\NormalTok{ simulated.formula,}
  \AttributeTok{probit.model =} \ConstantTok{TRUE}\NormalTok{, }\AttributeTok{DIC =} \ConstantTok{TRUE}\NormalTok{,}
  \AttributeTok{simulated.formula.probit =}\NormalTok{ simulated.formula.probit}
\NormalTok{)}
\NormalTok{heritability.sim }\OtherTok{\textless{}{-}} \FunctionTok{get.all.heritabilities}\NormalTok{(}
\NormalTok{  simulation.res2}\SpecialCharTok{$}\NormalTok{fit, simulation.res2}\SpecialCharTok{$}\NormalTok{fit.probit,}
\NormalTok{  simulation.res2}\SpecialCharTok{$}\NormalTok{p, n.samples,}
  \AttributeTok{fixed =} \ConstantTok{FALSE}
\NormalTok{)}
\end{Highlighting}
\end{Shaded}

Method to export heritability estimates in a TeX table

\begin{Shaded}
\begin{Highlighting}[]
\NormalTok{get.mode }\OtherTok{\textless{}{-}} \ControlFlowTok{function}\NormalTok{(vec) \{}
  \CommentTok{\#\textquotesingle{} General helper to get mode of a vector}
\NormalTok{  d }\OtherTok{\textless{}{-}} \FunctionTok{density}\NormalTok{(vec)}
\NormalTok{  d}\SpecialCharTok{$}\NormalTok{x[}\FunctionTok{which.max}\NormalTok{(d}\SpecialCharTok{$}\NormalTok{y)]}
\NormalTok{\}}


\NormalTok{print.one.metric }\OtherTok{\textless{}{-}} \ControlFlowTok{function}\NormalTok{(fit, param, digits) \{}
  \CommentTok{\#\textquotesingle{} Helper for heritability table, rounding estiamtes}
  \FunctionTok{paste}\NormalTok{(}
    \FunctionTok{round}\NormalTok{(}\FunctionTok{mean}\NormalTok{(}\FunctionTok{get}\NormalTok{(param, fit)), digits), }\StringTok{" \& "}\NormalTok{,}
    \FunctionTok{round}\NormalTok{(}\FunctionTok{get.mode}\NormalTok{(}\FunctionTok{get}\NormalTok{(param, fit)), digits), }\StringTok{" \& "}\NormalTok{,}
    \FunctionTok{round}\NormalTok{(}\FunctionTok{sd}\NormalTok{(}\FunctionTok{get}\NormalTok{(param, fit)), digits),}
    \AttributeTok{sep =} \StringTok{""}
\NormalTok{  )}
\NormalTok{\}}


\NormalTok{print.heritability.table }\OtherTok{\textless{}{-}} \ControlFlowTok{function}\NormalTok{(digits, h2, }\AttributeTok{simulation =}\NormalTok{ T) \{}
  \CommentTok{\#\textquotesingle{} Output LaTeX table of heritability}
  \CommentTok{\#\textquotesingle{}}
  \CommentTok{\#\textquotesingle{} Writes table of heritability with posterior mean, posterior mode}
  \CommentTok{\#\textquotesingle{} and standard deviation, to a TeX file. Works for both datasets.}
  \CommentTok{\#\textquotesingle{} @param digits Number of significant digits}
  \CommentTok{\#\textquotesingle{} @param h2 Heritability DF with different scales}
  \CommentTok{\#\textquotesingle{} @param simulation Simulation flag for the table\textquotesingle{}s filename}
\NormalTok{  filename }\OtherTok{\textless{}{-}} \FunctionTok{ifelse}\NormalTok{(simulation, }\StringTok{"heritability simulation"}\NormalTok{,}
      \StringTok{"heritability application"}\NormalTok{)}
\NormalTok{  header }\OtherTok{\textless{}{-}} \FunctionTok{paste}\NormalTok{(}
    \StringTok{"\% TABLE FROM R:"}\NormalTok{, }\FunctionTok{format}\NormalTok{(}\FunctionTok{Sys.time}\NormalTok{(), }\StringTok{"\%a \%b \%d \%X \%Y"}\NormalTok{), }\StringTok{"}\SpecialCharTok{\textbackslash{}n}\StringTok{"}\NormalTok{,}
    \StringTok{"}\SpecialCharTok{\textbackslash{}\textbackslash{}}\StringTok{begin\{tabular\}\{lccc\}}\SpecialCharTok{\textbackslash{}n}\StringTok{"}\NormalTok{,}
    \StringTok{"}\SpecialCharTok{\textbackslash{}\textbackslash{}}\StringTok{hline}\SpecialCharTok{\textbackslash{}n}\StringTok{"}\NormalTok{,}
    \StringTok{"Model \& Mean \& Mode \& Standard deviation  }\SpecialCharTok{\textbackslash{}\textbackslash{}\textbackslash{}\textbackslash{}}\StringTok{ }\SpecialCharTok{\textbackslash{}n}\StringTok{"}\NormalTok{,}
    \StringTok{"}\SpecialCharTok{\textbackslash{}\textbackslash{}}\StringTok{hline }\SpecialCharTok{\textbackslash{}n}\StringTok{"}
\NormalTok{  )}
\NormalTok{  main }\OtherTok{\textless{}{-}} \FunctionTok{paste}\NormalTok{(}
    \StringTok{" Gaussian $h\^{}2\_}\SpecialCharTok{\textbackslash{}\textbackslash{}}\StringTok{text\{obs\}$ \&"}\NormalTok{,}
    \FunctionTok{print.one.metric}\NormalTok{(h2, }\StringTok{"gaussian"}\NormalTok{, digits), }\StringTok{"}\SpecialCharTok{\textbackslash{}\textbackslash{}\textbackslash{}\textbackslash{}}\StringTok{ }\SpecialCharTok{\textbackslash{}n}\StringTok{"}\NormalTok{,}
    \StringTok{"Probit $h\^{}2\_\{}\SpecialCharTok{\textbackslash{}\textbackslash{}}\StringTok{Psi\}$ \&"}\NormalTok{,}
    \FunctionTok{print.one.metric}\NormalTok{(h2, }\StringTok{"probit.qgglmm"}\NormalTok{, digits), }\StringTok{"}\SpecialCharTok{\textbackslash{}\textbackslash{}\textbackslash{}\textbackslash{}}\StringTok{ }\SpecialCharTok{\textbackslash{}n}\StringTok{"}\NormalTok{,}
    \StringTok{" \& \& \& }\SpecialCharTok{\textbackslash{}\textbackslash{}\textbackslash{}\textbackslash{}}\StringTok{ }\SpecialCharTok{\textbackslash{}n}\StringTok{"}\NormalTok{,}
    \StringTok{"Gaussian $h\^{}2\_}\SpecialCharTok{\textbackslash{}\textbackslash{}}\StringTok{text\{liab\}$ \&"}\NormalTok{,}
    \FunctionTok{print.one.metric}\NormalTok{(h2, }\StringTok{"gaussian.liability"}\NormalTok{, digits), }\StringTok{"}\SpecialCharTok{\textbackslash{}\textbackslash{}\textbackslash{}\textbackslash{}}\StringTok{ }\SpecialCharTok{\textbackslash{}n}\StringTok{"}\NormalTok{,}
    \StringTok{"Probit $h\^{}2\_\{}\SpecialCharTok{\textbackslash{}\textbackslash{}}\StringTok{Phi\}$ \&"}\NormalTok{,}
    \FunctionTok{print.one.metric}\NormalTok{(h2, }\StringTok{"probit.scaled"}\NormalTok{, digits), }\StringTok{"}\SpecialCharTok{\textbackslash{}\textbackslash{}\textbackslash{}\textbackslash{}}\StringTok{ }\SpecialCharTok{\textbackslash{}n}\StringTok{"}\NormalTok{,}
    \StringTok{"}\SpecialCharTok{\textbackslash{}\textbackslash{}}\StringTok{bottomrule"}
\NormalTok{  )}
\NormalTok{  footer }\OtherTok{\textless{}{-}} \StringTok{"}\SpecialCharTok{\textbackslash{}\textbackslash{}}\StringTok{end\{tabular\}"}

  \FunctionTok{write}\NormalTok{(}\FunctionTok{paste}\NormalTok{(header, main, footer, }\AttributeTok{sep =} \StringTok{"}\SpecialCharTok{\textbackslash{}n}\StringTok{"}\NormalTok{),}
        \AttributeTok{file=}\FunctionTok{paste0}\NormalTok{(}\StringTok{"../figures/"}\NormalTok{, filename, }\StringTok{".tex"}\NormalTok{))}
\NormalTok{\}}
\FunctionTok{print.heritability.table}\NormalTok{(}\DecValTok{3}\NormalTok{, heritability, }\ConstantTok{FALSE}\NormalTok{)}
\FunctionTok{print.heritability.table}\NormalTok{(}\DecValTok{3}\NormalTok{, heritability.sim, }\ConstantTok{TRUE}\NormalTok{)}
\end{Highlighting}
\end{Shaded}

Code for part of discussion:

\begin{Shaded}
\begin{Highlighting}[]
\FunctionTok{write}\NormalTok{(}\FunctionTok{paste0}\NormalTok{(}\StringTok{"$}\SpecialCharTok{\textbackslash{}\textbackslash{}}\StringTok{hat p="}\NormalTok{, }\FunctionTok{round}\NormalTok{(simulation.res2}\SpecialCharTok{$}\NormalTok{p,}\DecValTok{2}\NormalTok{),}
             \StringTok{"$, that the true heritability of the observation scale"}\NormalTok{,}
             \StringTok{" would be $"}\NormalTok{,}
             \FunctionTok{round}\NormalTok{(}\DecValTok{1}\SpecialCharTok{/}\DecValTok{3}\SpecialCharTok{*}\DecValTok{1}\SpecialCharTok{/}\FunctionTok{threshold.scaling.param}\NormalTok{(simulation.res2}\SpecialCharTok{$}\NormalTok{p),}\DecValTok{3}\NormalTok{),}
             \StringTok{"$."}\NormalTok{),}
             \AttributeTok{file=}\StringTok{"../figures/trueh2discussion.tex"}\NormalTok{)}
\end{Highlighting}
\end{Shaded}

The table gives some indication, but we also want to look qualitatively
on the densities. We start by just plotting a grid to compare each
Gaussian vs.~Probit scale two by two.

\begin{Shaded}
\begin{Highlighting}[]
\NormalTok{plot\_grid\_of\_heritability }\OtherTok{\textless{}{-}} \ControlFlowTok{function}\NormalTok{(heritability, SAVE.PLOT,}
\NormalTok{                                      plot.fn, }\AttributeTok{colorscheme =} \ConstantTok{NA}\NormalTok{) \{}
  \CommentTok{\#\textquotesingle{} Plot 3x2 grid of h\^{}2 comparisons}
  \CommentTok{\#\textquotesingle{}}
  \CommentTok{\#\textquotesingle{} Compare density of posterior heritability between all scales for}
  \CommentTok{\#\textquotesingle{} Gaussian model to all scales of the probit model. First row is}
  \CommentTok{\#\textquotesingle{} observation{-}scale compares to latent, psi and phi, respectively.}
  \CommentTok{\#\textquotesingle{} Second is the same for liability scale in the Gaussian case,}
  \CommentTok{\#\textquotesingle{} and the same three cases for probit.}
  \CommentTok{\#\textquotesingle{} @param heritability DF of all heritability estimates}
  \CommentTok{\#\textquotesingle{} @param SAVE.PLOT Flag for storing plot to disk}
  \CommentTok{\#\textquotesingle{} @param plot.fn Filename, must include file extension}
  \CommentTok{\#\textquotesingle{} @param colorscheme (optional) Color palette to use for density}
  \CommentTok{\#\textquotesingle{} @return List of all individual plots, as well as the grid plot}
\NormalTok{  p1 }\OtherTok{\textless{}{-}} \FunctionTok{ggplot}\NormalTok{() }\SpecialCharTok{+}
    \FunctionTok{geom\_density}\NormalTok{(}
      \AttributeTok{data =} \FunctionTok{melt}\NormalTok{(heritability[, }\FunctionTok{c}\NormalTok{(}\StringTok{"gaussian"}\NormalTok{, }\StringTok{"probit.latent"}\NormalTok{)]),}
      \FunctionTok{aes}\NormalTok{(}\AttributeTok{x =}\NormalTok{ value, }\AttributeTok{fill =}\NormalTok{ variable), }\AttributeTok{alpha =} \FloatTok{0.5}
\NormalTok{    ) }\SpecialCharTok{+}
    \FunctionTok{theme}\NormalTok{(}\AttributeTok{legend.position =} \StringTok{"none"}\NormalTok{, }\AttributeTok{axis.title =} \FunctionTok{element\_blank}\NormalTok{()) }\SpecialCharTok{+}
    \FunctionTok{labs}\NormalTok{(}\AttributeTok{title =} \FunctionTok{TeX}\NormalTok{(}\StringTok{"$h\^{}2\_\{obs\}$ vs. $h\^{}2\_\{lat\}$"}\NormalTok{)) }\SpecialCharTok{+}
    \ControlFlowTok{if}\NormalTok{ (}\SpecialCharTok{!}\FunctionTok{is.na}\NormalTok{(colorscheme)) }\FunctionTok{scale\_fill\_brewer}\NormalTok{(}\AttributeTok{palette =}\NormalTok{ colorscheme)}
\NormalTok{  p2 }\OtherTok{\textless{}{-}} \FunctionTok{ggplot}\NormalTok{() }\SpecialCharTok{+}
    \FunctionTok{geom\_density}\NormalTok{(}
      \AttributeTok{data =} \FunctionTok{melt}\NormalTok{(heritability[, }\FunctionTok{c}\NormalTok{(}
        \StringTok{"gaussian.liability"}\NormalTok{,}
        \StringTok{"probit.latent"}
\NormalTok{      )]),}
      \FunctionTok{aes}\NormalTok{(}\AttributeTok{x =}\NormalTok{ value, }\AttributeTok{fill =}\NormalTok{ variable), }\AttributeTok{alpha =} \FloatTok{0.5}
\NormalTok{    ) }\SpecialCharTok{+}
    \FunctionTok{theme}\NormalTok{(}\AttributeTok{legend.position =} \StringTok{"none"}\NormalTok{, }\AttributeTok{axis.title =} \FunctionTok{element\_blank}\NormalTok{()) }\SpecialCharTok{+}
    \FunctionTok{labs}\NormalTok{(}\AttributeTok{title =} \FunctionTok{TeX}\NormalTok{(}\StringTok{"$h\^{}2\_\{liab\}$ vs. $h\^{}2\_\{lat\}$"}\NormalTok{)) }\SpecialCharTok{+}
    \ControlFlowTok{if}\NormalTok{ (}\SpecialCharTok{!}\FunctionTok{is.na}\NormalTok{(colorscheme)) }\FunctionTok{scale\_fill\_brewer}\NormalTok{(}\AttributeTok{palette =}\NormalTok{ colorscheme)}
\NormalTok{  p3 }\OtherTok{\textless{}{-}} \FunctionTok{ggplot}\NormalTok{() }\SpecialCharTok{+}
    \FunctionTok{geom\_density}\NormalTok{(}
      \AttributeTok{data =} \FunctionTok{melt}\NormalTok{(heritability[, }\FunctionTok{c}\NormalTok{(}
        \StringTok{"gaussian"}\NormalTok{,}
        \StringTok{"probit.qgglmm"}
\NormalTok{      )]),}
      \FunctionTok{aes}\NormalTok{(}\AttributeTok{x =}\NormalTok{ value, }\AttributeTok{fill =}\NormalTok{ variable), }\AttributeTok{alpha =} \FloatTok{0.5}
\NormalTok{    ) }\SpecialCharTok{+}
    \FunctionTok{theme}\NormalTok{(}\AttributeTok{legend.position =} \StringTok{"none"}\NormalTok{, }\AttributeTok{axis.title =} \FunctionTok{element\_blank}\NormalTok{()) }\SpecialCharTok{+}
    \FunctionTok{labs}\NormalTok{(}\AttributeTok{title =} \FunctionTok{TeX}\NormalTok{(}\StringTok{"$h\^{}2\_\{obs\}$ vs. $h\^{}2\_\{}\SpecialCharTok{\textbackslash{}\textbackslash{}}\StringTok{Psi\}$"}\NormalTok{)) }\SpecialCharTok{+}
    \ControlFlowTok{if}\NormalTok{ (}\SpecialCharTok{!}\FunctionTok{is.na}\NormalTok{(colorscheme)) }\FunctionTok{scale\_fill\_brewer}\NormalTok{(}\AttributeTok{palette =}\NormalTok{ colorscheme)}
\NormalTok{  p4 }\OtherTok{\textless{}{-}} \FunctionTok{ggplot}\NormalTok{() }\SpecialCharTok{+}
    \FunctionTok{geom\_density}\NormalTok{(}
      \AttributeTok{data =} \FunctionTok{melt}\NormalTok{(heritability[, }\FunctionTok{c}\NormalTok{(}
        \StringTok{"gaussian.liability"}\NormalTok{,}
        \StringTok{"probit.qgglmm"}
\NormalTok{      )]),}
      \FunctionTok{aes}\NormalTok{(}\AttributeTok{x =}\NormalTok{ value, }\AttributeTok{fill =}\NormalTok{ variable), }\AttributeTok{alpha =} \FloatTok{0.5}
\NormalTok{    ) }\SpecialCharTok{+}
    \FunctionTok{theme}\NormalTok{(}\AttributeTok{legend.position =} \StringTok{"none"}\NormalTok{, }\AttributeTok{axis.title =} \FunctionTok{element\_blank}\NormalTok{()) }\SpecialCharTok{+}
    \FunctionTok{labs}\NormalTok{(}\AttributeTok{title =} \FunctionTok{TeX}\NormalTok{(}\StringTok{"$h\^{}2\_\{liab\}$ vs. $h\^{}2\_\{}\SpecialCharTok{\textbackslash{}\textbackslash{}}\StringTok{Psi\}$"}\NormalTok{)) }\SpecialCharTok{+}
    \ControlFlowTok{if}\NormalTok{ (}\SpecialCharTok{!}\FunctionTok{is.na}\NormalTok{(colorscheme)) }\FunctionTok{scale\_fill\_brewer}\NormalTok{(}\AttributeTok{palette =}\NormalTok{ colorscheme)}
\NormalTok{  p5 }\OtherTok{\textless{}{-}} \FunctionTok{ggplot}\NormalTok{() }\SpecialCharTok{+}
    \FunctionTok{geom\_density}\NormalTok{(}
      \AttributeTok{data =} \FunctionTok{melt}\NormalTok{(heritability[, }\FunctionTok{c}\NormalTok{(}
        \StringTok{"gaussian"}\NormalTok{,}
        \StringTok{"probit.scaled"}
\NormalTok{      )]),}
      \FunctionTok{aes}\NormalTok{(}\AttributeTok{x =}\NormalTok{ value, }\AttributeTok{fill =}\NormalTok{ variable), }\AttributeTok{alpha =} \FloatTok{0.5}
\NormalTok{    ) }\SpecialCharTok{+}
    \FunctionTok{theme}\NormalTok{(}\AttributeTok{legend.position =} \StringTok{"none"}\NormalTok{, }\AttributeTok{axis.title =} \FunctionTok{element\_blank}\NormalTok{()) }\SpecialCharTok{+}
    \FunctionTok{labs}\NormalTok{(}\AttributeTok{title =} \FunctionTok{TeX}\NormalTok{(}\StringTok{"$h\^{}2\_\{obs\}$ vs. $h\^{}2\_\{}\SpecialCharTok{\textbackslash{}\textbackslash{}}\StringTok{Phi\}$"}\NormalTok{)) }\SpecialCharTok{+}
    \ControlFlowTok{if}\NormalTok{ (}\SpecialCharTok{!}\FunctionTok{is.na}\NormalTok{(colorscheme)) }\FunctionTok{scale\_fill\_brewer}\NormalTok{(}\AttributeTok{palette =}\NormalTok{ colorscheme)}
\NormalTok{  p6 }\OtherTok{\textless{}{-}} \FunctionTok{ggplot}\NormalTok{() }\SpecialCharTok{+}
    \FunctionTok{geom\_density}\NormalTok{(}
      \AttributeTok{data =} \FunctionTok{melt}\NormalTok{(heritability[, }\FunctionTok{c}\NormalTok{(}
        \StringTok{"gaussian.liability"}\NormalTok{,}
        \StringTok{"probit.scaled"}
\NormalTok{      )]),}
      \FunctionTok{aes}\NormalTok{(}\AttributeTok{x =}\NormalTok{ value, }\AttributeTok{fill =}\NormalTok{ variable), }\AttributeTok{alpha =} \FloatTok{0.5}
\NormalTok{    ) }\SpecialCharTok{+}
    \FunctionTok{theme}\NormalTok{(}\AttributeTok{legend.position =} \StringTok{"none"}\NormalTok{, }\AttributeTok{axis.title =} \FunctionTok{element\_blank}\NormalTok{()) }\SpecialCharTok{+}
    \FunctionTok{labs}\NormalTok{(}\AttributeTok{title =} \FunctionTok{TeX}\NormalTok{(}\StringTok{"$h\^{}2\_\{liab\}$ vs. $h\^{}2\_\{}\SpecialCharTok{\textbackslash{}\textbackslash{}}\StringTok{Phi\}$"}\NormalTok{)) }\SpecialCharTok{+}
    \ControlFlowTok{if}\NormalTok{ (}\SpecialCharTok{!}\FunctionTok{is.na}\NormalTok{(colorscheme)) }\FunctionTok{scale\_fill\_brewer}\NormalTok{(}\AttributeTok{palette =}\NormalTok{ colorscheme)}
  \FunctionTok{set\_null\_device}\NormalTok{(cairo\_pdf)}
\NormalTok{  p }\OtherTok{\textless{}{-}} \FunctionTok{plot\_grid}\NormalTok{(p1, p3, p5, p2, p4, p6, }\FunctionTok{ggplot}\NormalTok{() }\SpecialCharTok{+}
    \FunctionTok{theme\_void}\NormalTok{(),}
  \FunctionTok{get\_legend}\NormalTok{(}
    \FunctionTok{ggplot}\NormalTok{(}\FunctionTok{data.frame}\NormalTok{(}\AttributeTok{v =} \FunctionTok{c}\NormalTok{(}\StringTok{"Gaussian"}\NormalTok{, }\StringTok{"Probit"}\NormalTok{), }\AttributeTok{x =} \FunctionTok{c}\NormalTok{(}\DecValTok{0}\NormalTok{, }\DecValTok{0}\NormalTok{))) }\SpecialCharTok{+}
      \FunctionTok{geom\_density}\NormalTok{(}\FunctionTok{aes}\NormalTok{(}\AttributeTok{x =}\NormalTok{ x, }\AttributeTok{fill =}\NormalTok{ v), }\AttributeTok{alpha =} \FloatTok{0.5}\NormalTok{) }\SpecialCharTok{+}
      \FunctionTok{scale\_fill\_discrete}\NormalTok{(}\AttributeTok{breaks =} \FunctionTok{c}\NormalTok{(}\StringTok{"Gaussian"}\NormalTok{, }\StringTok{"Probit"}\NormalTok{)) }\SpecialCharTok{+}
      \FunctionTok{theme}\NormalTok{(}\AttributeTok{legend.title =} \FunctionTok{element\_blank}\NormalTok{()) }\SpecialCharTok{+}
      \ControlFlowTok{if}\NormalTok{ (}\SpecialCharTok{!}\FunctionTok{is.na}\NormalTok{(colorscheme)) }\FunctionTok{scale\_fill\_brewer}\NormalTok{(}\AttributeTok{palette =}
\NormalTok{                                                   colorscheme)}
\NormalTok{  ),}
  \AttributeTok{axis =} \StringTok{"tblr"}
\NormalTok{  )}
  \ControlFlowTok{if}\NormalTok{ (SAVE.PLOT) }\FunctionTok{ggsave}\NormalTok{(}\FunctionTok{paste0}\NormalTok{(}\StringTok{"../figures/"}\NormalTok{, plot.fn), p)}
  \FunctionTok{list}\NormalTok{(}\AttributeTok{p1 =}\NormalTok{ p1, }\AttributeTok{p2 =}\NormalTok{ p2, }\AttributeTok{p3 =}\NormalTok{ p3, }\AttributeTok{p4 =}\NormalTok{ p4, }\AttributeTok{p5 =}\NormalTok{ p5, }\AttributeTok{p6 =}\NormalTok{ p6, }\AttributeTok{p =}\NormalTok{ p)}
\NormalTok{\}}
\NormalTok{plot.h2.appl }\OtherTok{\textless{}{-}} \FunctionTok{plot\_grid\_of\_heritability}\NormalTok{(}
\NormalTok{  heritability, SAVE.PLOT, }\StringTok{"grid\_application\_gaussian\_vs\_binom.pdf"}\NormalTok{,}
  \StringTok{"Dark2"}
\NormalTok{)}
\CommentTok{\# For simulation:}
\NormalTok{plot.h2.sim }\OtherTok{\textless{}{-}} \FunctionTok{plot\_grid\_of\_heritability}\NormalTok{(}
\NormalTok{  heritability.sim, SAVE.PLOT,}\StringTok{"grid\_simulation\_gaussian\_vs\_binom.pdf"}\NormalTok{,}
  \StringTok{"Spectral"}
\NormalTok{)}

\NormalTok{plot.h2.appl}\SpecialCharTok{$}\NormalTok{p}
\NormalTok{plot.h2.sim}\SpecialCharTok{$}\NormalTok{p}
\end{Highlighting}
\end{Shaded}

Key takeaways:

\begin{itemize}
\tightlist
\item
  The Gaussian model to liability scale doesn't fit well with the other
  latent models.
\item
  The scalings from binomial latent onto data scale fit well with the
  Gaussian one
\end{itemize}

The ones of greatest importance are plots \((1,2)\) (p3) and \((2,3)\)
(p6), so we extract them in particular

\begin{Shaded}
\begin{Highlighting}[]
\NormalTok{plot.h2.appl}\SpecialCharTok{$}\NormalTok{p3 }\SpecialCharTok{+}
  \FunctionTok{theme}\NormalTok{(}\AttributeTok{legend.position =} \StringTok{"right"}\NormalTok{, }\AttributeTok{legend.title =} \FunctionTok{element\_blank}\NormalTok{()) }\SpecialCharTok{+}
  \FunctionTok{scale\_fill\_brewer}\NormalTok{(}\AttributeTok{palette =} \StringTok{"Dark2"}\NormalTok{,}\AttributeTok{labels =} \FunctionTok{c}\NormalTok{(}
    \FunctionTok{TeX}\NormalTok{(}\StringTok{"Gaussian $h\^{}2\_\{obs\}$"}\NormalTok{), }\FunctionTok{TeX}\NormalTok{(}\StringTok{"Probit $h\^{}2\_}\SpecialCharTok{\textbackslash{}\textbackslash{}}\StringTok{Psi$"}\NormalTok{))}
\NormalTok{    ) }\SpecialCharTok{+}
  \FunctionTok{theme}\NormalTok{(}\AttributeTok{legend.text.align =} \DecValTok{0}\NormalTok{, }\AttributeTok{legend.position =} \StringTok{"bottom"}\NormalTok{,}
        \AttributeTok{legend.text =} \FunctionTok{element\_text}\NormalTok{(}\AttributeTok{size =} \DecValTok{23}\NormalTok{)) }\SpecialCharTok{+} \FunctionTok{ggtitle}\NormalTok{(}\StringTok{""}\NormalTok{)}

\ControlFlowTok{if}\NormalTok{ (SAVE.PLOT) }\FunctionTok{ggsave}\NormalTok{(}
  \StringTok{"../figures/heritability\_application\_obsscale.pdf"}\NormalTok{)}

\NormalTok{plot.h2.appl}\SpecialCharTok{$}\NormalTok{p6 }\SpecialCharTok{+}
  \FunctionTok{theme}\NormalTok{(}\AttributeTok{legend.position =} \StringTok{"right"}\NormalTok{, }\AttributeTok{legend.title =} \FunctionTok{element\_blank}\NormalTok{()) }\SpecialCharTok{+}
  \FunctionTok{scale\_fill\_brewer}\NormalTok{(}\AttributeTok{palette =} \StringTok{"Dark2"}\NormalTok{, }\AttributeTok{labels =} \FunctionTok{c}\NormalTok{(}
    \FunctionTok{TeX}\NormalTok{(}\StringTok{"Gaussian $h\^{}2\_\{liab\}$"}\NormalTok{), }\FunctionTok{TeX}\NormalTok{(}\StringTok{"Probit $h\^{}2\_}\SpecialCharTok{\textbackslash{}\textbackslash{}}\StringTok{Phi$"}\NormalTok{))) }\SpecialCharTok{+}
  \FunctionTok{theme}\NormalTok{(}\AttributeTok{legend.text.align =} \DecValTok{0}\NormalTok{, }\AttributeTok{legend.position =} \StringTok{"bottom"}\NormalTok{,}
        \AttributeTok{legend.text =} \FunctionTok{element\_text}\NormalTok{(}\AttributeTok{size =} \DecValTok{23}\NormalTok{)) }\SpecialCharTok{+} \FunctionTok{ggtitle}\NormalTok{(}\StringTok{""}\NormalTok{)}
\ControlFlowTok{if}\NormalTok{ (SAVE.PLOT) }\FunctionTok{ggsave}\NormalTok{(}
  \StringTok{"../figures/heritability\_application\_liabscale.pdf"}\NormalTok{)}

\NormalTok{plot.h2.sim}\SpecialCharTok{$}\NormalTok{p3 }\SpecialCharTok{+}
  \FunctionTok{theme}\NormalTok{(}\AttributeTok{legend.position =} \StringTok{"right"}\NormalTok{, }\AttributeTok{legend.title =} \FunctionTok{element\_blank}\NormalTok{()) }\SpecialCharTok{+}
  \FunctionTok{scale\_fill\_brewer}\NormalTok{(}\AttributeTok{palette =} \StringTok{"Spectral"}\NormalTok{, }\AttributeTok{labels =} \FunctionTok{c}\NormalTok{(}
    \FunctionTok{TeX}\NormalTok{(}\StringTok{"Gaussian $h\^{}2\_\{obs\}$"}\NormalTok{), }\FunctionTok{TeX}\NormalTok{(}\StringTok{"Probit $h\^{}2\_}\SpecialCharTok{\textbackslash{}\textbackslash{}}\StringTok{Psi$"}\NormalTok{))}
\NormalTok{    ) }\SpecialCharTok{+}
  \FunctionTok{theme}\NormalTok{(}\AttributeTok{legend.text.align =} \DecValTok{0}\NormalTok{, }\AttributeTok{legend.position =} \StringTok{"bottom"}\NormalTok{,}
        \AttributeTok{legend.text =} \FunctionTok{element\_text}\NormalTok{(}\AttributeTok{size =} \DecValTok{23}\NormalTok{)) }\SpecialCharTok{+} \FunctionTok{ggtitle}\NormalTok{(}\StringTok{""}\NormalTok{)}
\ControlFlowTok{if}\NormalTok{ (SAVE.PLOT) }\FunctionTok{ggsave}\NormalTok{(}
  \StringTok{"../figures/heritability\_simulation\_obsscale.pdf"}\NormalTok{)}
\NormalTok{plot.h2.sim}\SpecialCharTok{$}\NormalTok{p6 }\SpecialCharTok{+}
  \FunctionTok{theme}\NormalTok{(}\AttributeTok{legend.position =} \StringTok{"right"}\NormalTok{, }\AttributeTok{legend.title =} \FunctionTok{element\_blank}\NormalTok{()) }\SpecialCharTok{+}
  \FunctionTok{scale\_fill\_brewer}\NormalTok{(}\AttributeTok{palette =} \StringTok{"Spectral"}\NormalTok{, }\AttributeTok{labels =} \FunctionTok{c}\NormalTok{(}
    \FunctionTok{TeX}\NormalTok{(}\StringTok{"Gaussian $h\^{}2\_\{liab\}$"}\NormalTok{),}
    \FunctionTok{TeX}\NormalTok{(}\StringTok{"Probit $h\^{}2\_}\SpecialCharTok{\textbackslash{}\textbackslash{}}\StringTok{Phi$"}\NormalTok{))) }\SpecialCharTok{+}
  \FunctionTok{theme}\NormalTok{(}\AttributeTok{legend.text.align =} \DecValTok{0}\NormalTok{, }\AttributeTok{legend.position =} \StringTok{"bottom"}\NormalTok{,}
        \AttributeTok{legend.text =} \FunctionTok{element\_text}\NormalTok{(}\AttributeTok{size =} \DecValTok{23}\NormalTok{)) }\SpecialCharTok{+} \FunctionTok{ggtitle}\NormalTok{(}\StringTok{""}\NormalTok{)}
\ControlFlowTok{if}\NormalTok{ (SAVE.PLOT) }\FunctionTok{ggsave}\NormalTok{(}
  \StringTok{"../figures/heritability\_simulation\_liabscale.pdf"}\NormalTok{)}
\end{Highlighting}
\end{Shaded}

Finally, we want to look at DIC values for simulation model. It can't be
compared to the song sparrow data directly, but how much the Gaussian
and probit differ can be compared.

\begin{Shaded}
\begin{Highlighting}[]
\FunctionTok{data.frame}\NormalTok{(}
  \AttributeTok{Gaussian =}\NormalTok{ simulation.res2}\SpecialCharTok{$}\NormalTok{fit}\SpecialCharTok{$}\NormalTok{dic}\SpecialCharTok{$}\NormalTok{dic,}
  \AttributeTok{Probit =}\NormalTok{ simulation.res2}\SpecialCharTok{$}\NormalTok{fit.probit}\SpecialCharTok{$}\NormalTok{dic}\SpecialCharTok{$}\NormalTok{dic,}
  \AttributeTok{row.names =} \StringTok{"Deviance Information Criteria"}
\NormalTok{)}
\end{Highlighting}
\end{Shaded}

\hypertarget{fixed-effects-for-simulation}{%
\section*{Fixed effects for
simulation}\label{fixed-effects-for-simulation}}

We now add a sex covariate to the linear predictor. We use that \[
\operatorname{Var}[\beta_{\text{sex}}\mathbf{x}_{\text{sex}}] = \beta_{\text{sex}}^2  \sigma^2_\text{sex}
\]

\begin{Shaded}
\begin{Highlighting}[]
\NormalTok{linear\_predictor\_fixedeffects }\OtherTok{\textless{}{-}} \ControlFlowTok{function}\NormalTok{(u, simulated.d.ped) \{}
  \CommentTok{\#\textquotesingle{} \textbackslash{}Tilde\{\textbackslash{}eta\} = a + N(0, varE) + betaSex x\_\{sex\}}
\NormalTok{  varE }\OtherTok{\textless{}{-}} \DecValTok{1}
\NormalTok{  betaSex }\OtherTok{\textless{}{-}} \DecValTok{100}
\NormalTok{  out }\OtherTok{\textless{}{-}} \FunctionTok{c}\NormalTok{()}
\NormalTok{  intercept }\OtherTok{\textless{}{-}} \DecValTok{0}
\NormalTok{  residuals }\OtherTok{\textless{}{-}} \FunctionTok{rnorm}\NormalTok{(}\FunctionTok{length}\NormalTok{(u), }\AttributeTok{mean =} \DecValTok{0}\NormalTok{, }\AttributeTok{sd =} \FunctionTok{sqrt}\NormalTok{(varE))}
  \ControlFlowTok{for}\NormalTok{ (idx }\ControlFlowTok{in} \FunctionTok{seq\_along}\NormalTok{(u)) \{}
\NormalTok{    out }\OtherTok{\textless{}{-}} \FunctionTok{c}\NormalTok{(}
\NormalTok{      out,}
\NormalTok{      intercept }\SpecialCharTok{+}\NormalTok{ betaSex }\SpecialCharTok{*}\NormalTok{ simulated.d.ped}\SpecialCharTok{$}\NormalTok{sex[idx] }\SpecialCharTok{+}\NormalTok{ u[idx] }\SpecialCharTok{+}
\NormalTok{        residuals[idx]}
\NormalTok{    )}
\NormalTok{  \}}
\NormalTok{  out}
\NormalTok{\}}
\NormalTok{simulated.formula.fixedeffects }\OtherTok{\textless{}{-}}\NormalTok{ simulated.response }\SpecialCharTok{\textasciitilde{}}\NormalTok{ sex }\SpecialCharTok{+}
  \FunctionTok{f}\NormalTok{(id,}
    \AttributeTok{model =} \StringTok{"generic0"}\NormalTok{,}
    \AttributeTok{Cmatrix =}\NormalTok{ simulated.Cmatrix,}
    \AttributeTok{constr =} \ConstantTok{FALSE}\NormalTok{,}
    \AttributeTok{hyper =} \FunctionTok{list}\NormalTok{(}
      \AttributeTok{prec =} \FunctionTok{list}\NormalTok{(}
        \AttributeTok{initial =} \FunctionTok{log}\NormalTok{(}\DecValTok{1} \SpecialCharTok{/} \DecValTok{10}\NormalTok{), }\AttributeTok{prior =} \StringTok{"pc.prec"}\NormalTok{,}
        \AttributeTok{param =} \FunctionTok{c}\NormalTok{(}\DecValTok{1}\NormalTok{, }\FloatTok{0.05}\NormalTok{)}
\NormalTok{      ) }\CommentTok{\# PC priors}
\NormalTok{    )}
\NormalTok{  )}

\CommentTok{\# u = a + 100*x\_sex + N(0,1) {-} balanced binary trait}
\NormalTok{m }\OtherTok{\textless{}{-}} \FunctionTok{plot.h2.deviation}\NormalTok{(}
  \AttributeTok{dichotomize =} \StringTok{"round\_balanced"}\NormalTok{, }\AttributeTok{title =} \FunctionTok{TeX}\NormalTok{(}\StringTok{"$}\SpecialCharTok{\textbackslash{}\textbackslash{}}\StringTok{beta\_\{sex\}=100$"}\NormalTok{),}
  \AttributeTok{SAVE.PLOT =}\NormalTok{ SAVE.PLOT, }\AttributeTok{plot.fn =} \StringTok{"fixedeffects\_beta100"}\NormalTok{,}
  \AttributeTok{sigma.scale =} \StringTok{"log"}\NormalTok{,}
  \AttributeTok{lin.pred =}\NormalTok{ linear\_predictor\_fixedeffects,}
  \AttributeTok{simulated.formula =}\NormalTok{ simulated.formula.fixedeffects,}
  \AttributeTok{Ve =} \DecValTok{100}\SpecialCharTok{\^{}}\DecValTok{2}\NormalTok{, }\AttributeTok{fixedeffects =} \ConstantTok{TRUE}\NormalTok{, }\AttributeTok{dynamic.priors=}\ConstantTok{TRUE}
\NormalTok{)}
\NormalTok{m}\SpecialCharTok{$}\NormalTok{p }\SpecialCharTok{+} \FunctionTok{xlim}\NormalTok{(}\FunctionTok{c}\NormalTok{(}\DecValTok{1}\NormalTok{,}\DecValTok{10}\SpecialCharTok{\^{}}\DecValTok{4}\NormalTok{)) }\SpecialCharTok{+} \FunctionTok{theme}\NormalTok{(}\AttributeTok{legend.position =} \StringTok{"none"}\NormalTok{)}
\ControlFlowTok{if}\NormalTok{(SAVE.PLOT)\{}
  \FunctionTok{ggsave}\NormalTok{(}\StringTok{"../figures/simulation\_deviance\_fixedeffects\_beta100.pdf"}\NormalTok{)}
\NormalTok{  \} }\CommentTok{\# Re{-}save figure with specified x{-}lim.}

\CommentTok{\# Re{-}run with smaller magnitude for fixed effect}
\NormalTok{linear\_predictor\_fixedeffects }\OtherTok{\textless{}{-}} \ControlFlowTok{function}\NormalTok{(u, simulated.d.ped) \{}
\NormalTok{  varE }\OtherTok{\textless{}{-}} \DecValTok{1}
\NormalTok{  betaSex }\OtherTok{\textless{}{-}} \DecValTok{10}
\NormalTok{  out }\OtherTok{\textless{}{-}} \FunctionTok{c}\NormalTok{()}
\NormalTok{  intercept }\OtherTok{\textless{}{-}} \DecValTok{0} \CommentTok{\#{-}4.5}
\NormalTok{  residuals }\OtherTok{\textless{}{-}} \FunctionTok{rnorm}\NormalTok{(}\FunctionTok{length}\NormalTok{(u), }\AttributeTok{mean =} \DecValTok{0}\NormalTok{, }\AttributeTok{sd =} \FunctionTok{sqrt}\NormalTok{(varE))}
  \ControlFlowTok{for}\NormalTok{ (idx }\ControlFlowTok{in} \FunctionTok{seq\_along}\NormalTok{(u)) \{}
\NormalTok{    out }\OtherTok{\textless{}{-}} \FunctionTok{c}\NormalTok{(}
\NormalTok{      out,}
\NormalTok{      intercept }\SpecialCharTok{+}\NormalTok{ betaSex }\SpecialCharTok{*}\NormalTok{ simulated.d.ped}\SpecialCharTok{$}\NormalTok{sex[idx] }\SpecialCharTok{+}\NormalTok{ u[idx] }\SpecialCharTok{+}
\NormalTok{        residuals[idx]}
\NormalTok{    )}
\NormalTok{  \}}
\NormalTok{  out}
\NormalTok{\}}

\CommentTok{\# u = a + 10*x\_sex + N(0,1) {-} balanced binary trait}
\NormalTok{m2 }\OtherTok{\textless{}{-}} \FunctionTok{plot.h2.deviation}\NormalTok{(}
  \AttributeTok{dichotomize =} \StringTok{"round\_balanced"}\NormalTok{, }\AttributeTok{title =} \FunctionTok{TeX}\NormalTok{(}\StringTok{"$}\SpecialCharTok{\textbackslash{}\textbackslash{}}\StringTok{beta\_\{sex\}=10$"}\NormalTok{),}
  \AttributeTok{SAVE.PLOT =}\NormalTok{ SAVE.PLOT, }\AttributeTok{plot.fn =} \StringTok{"fixedeffects\_beta10"}\NormalTok{,}
  \AttributeTok{sigma.scale =} \StringTok{"log"}\NormalTok{,}
  \AttributeTok{lin.pred =}\NormalTok{ linear\_predictor\_fixedeffects,}
  \AttributeTok{simulated.formula =}\NormalTok{ simulated.formula.fixedeffects,}
  \AttributeTok{Ve =} \DecValTok{10}\SpecialCharTok{\^{}}\DecValTok{2}\NormalTok{, }\AttributeTok{fixedeffects =} \ConstantTok{TRUE}\NormalTok{, }\AttributeTok{dynamic.priors =} \ConstantTok{TRUE}
\NormalTok{)}
\NormalTok{m2}\SpecialCharTok{$}\NormalTok{p}

\CommentTok{\# u = a + 10*x\_sex + N(0,1) {-} somewhat unbalanced}
\NormalTok{m3 }\OtherTok{\textless{}{-}} \FunctionTok{plot.h2.deviation}\NormalTok{(}
  \AttributeTok{dichotomize =} \FloatTok{0.1}\NormalTok{,}
  \AttributeTok{title =} \StringTok{"Simulation heritability for }\SpecialCharTok{\textbackslash{}n}\StringTok{fixed effects model"}\NormalTok{,}
  \AttributeTok{SAVE.PLOT =}\NormalTok{ SAVE.PLOT, }\AttributeTok{plot.fn =} \StringTok{"fixedeffects\_beta10\_unbalanced"}\NormalTok{,}
  \AttributeTok{sigma.scale =} \StringTok{"log"}\NormalTok{,}
  \AttributeTok{lin.pred =}\NormalTok{ linear\_predictor\_fixedeffects,}
  \AttributeTok{simulated.formula =}\NormalTok{ simulated.formula.fixedeffects,}
  \AttributeTok{Ve =} \DecValTok{10}\SpecialCharTok{\^{}}\DecValTok{2}\NormalTok{, }\AttributeTok{fixedeffects =} \ConstantTok{TRUE}\NormalTok{, }\AttributeTok{dynamic.priors =} \ConstantTok{TRUE}
\NormalTok{)}
\NormalTok{m3}\SpecialCharTok{$}\NormalTok{p}

\CommentTok{\# How unbalanced is the response?}
\FunctionTok{summary}\NormalTok{(m3}\SpecialCharTok{$}\NormalTok{p}\SpecialCharTok{$}\NormalTok{data}\SpecialCharTok{$}\NormalTok{plist)}
\end{Highlighting}
\end{Shaded}

Similar to the case without fixed effects, we provide code for plotting
based on results form remote server.

\begin{Shaded}
\begin{Highlighting}[]
\FunctionTok{load}\NormalTok{(}\StringTok{"markovfixed\_50\_runs.Rdata"}\NormalTok{)}
\NormalTok{mp4 }\OtherTok{\textless{}{-}} \FunctionTok{markov.plotter}\NormalTok{(res.fixed1)}
\NormalTok{mp5 }\OtherTok{\textless{}{-}} \FunctionTok{markov.plotter}\NormalTok{(res.fixed2)}
\NormalTok{mp6 }\OtherTok{\textless{}{-}} \FunctionTok{markov.plotter}\NormalTok{(res.fixed3,}
  \AttributeTok{legend.name=}\StringTok{"Simulation heritability for}\SpecialCharTok{\textbackslash{}n}\StringTok{fixed effects model"}\NormalTok{)}
\NormalTok{fixed.fn }\OtherTok{\textless{}{-}} \StringTok{"../figures/simulation\_deviance\_fixedeffects\_"}
\FunctionTok{ggsave}\NormalTok{(}\FunctionTok{paste0}\NormalTok{(fixed.fn, }\StringTok{"beta100.pdf"}\NormalTok{),}
\NormalTok{       mp4}\SpecialCharTok{+}\FunctionTok{xlim}\NormalTok{(}\FunctionTok{c}\NormalTok{(}\DecValTok{1}\NormalTok{,}\DecValTok{10}\SpecialCharTok{\^{}}\DecValTok{4}\NormalTok{)))}
\FunctionTok{ggsave}\NormalTok{(}\FunctionTok{paste0}\NormalTok{(fixed.fn, }\StringTok{"beta10.pdf"}\NormalTok{), mp5)}
\FunctionTok{ggsave}\NormalTok{(}\FunctionTok{paste0}\NormalTok{(fixed.fn, }\StringTok{"beta10\_unbalanced.pdf"}\NormalTok{), mp6)}
\NormalTok{mp6.legend }\OtherTok{\textless{}{-}}\NormalTok{ cowplot}\SpecialCharTok{::}\FunctionTok{get\_legend}\NormalTok{(}
\NormalTok{  mp6 }\SpecialCharTok{+}\FunctionTok{theme}\NormalTok{(}\AttributeTok{legend.position =} \StringTok{"right"}\NormalTok{, }\AttributeTok{legend.text.align =} \DecValTok{0}\NormalTok{))}
\FunctionTok{pdf}\NormalTok{(}\StringTok{"../figures/simulation\_deviance\_fixedeffects\_legend.pdf"}\NormalTok{,}
    \AttributeTok{width =} \FloatTok{7.87402}\NormalTok{, }\AttributeTok{height =} \FloatTok{7.87402}\NormalTok{)}
\FunctionTok{grid.newpage}\NormalTok{()}
\FunctionTok{grid.draw}\NormalTok{(mp6.legend)}
\FunctionTok{dev.off}\NormalTok{()}
\end{Highlighting}
\end{Shaded}

For sufficiently large choice of \(\beta\) corresponding to sex, we get
progressively worse results as is expected.

\hypertarget{fixed-effect-model-performance}{%
\subsection*{Fixed effect model
performance}\label{fixed-effect-model-performance}}

Another aspect we can examine, is how the grid plots of heritability
scales would look like if we use a Gaussian and probit model with
(somewhat dominating) fixed effect. This is implemented below.

\begin{Shaded}
\begin{Highlighting}[]
\NormalTok{plot.fixedeffects.h2 }\OtherTok{\textless{}{-}} \ControlFlowTok{function}\NormalTok{(sA, .dichotomize, }\AttributeTok{include.fixed =}\NormalTok{ T,}
                                 \AttributeTok{sE =} \DecValTok{1}\NormalTok{, }\AttributeTok{beta =} \DecValTok{10}\NormalTok{, }\AttributeTok{SAVE.PLOT =}\NormalTok{ T,}
                                 \AttributeTok{plot.legend=}\NormalTok{F) \{}
  \CommentTok{\#\textquotesingle{} Plot h2 density of gaussian and backtransformed probit model,}
  \CommentTok{\#\textquotesingle{} }
  \CommentTok{\#\textquotesingle{} Fit simulation models with fixed effects, compute h2 for}
  \CommentTok{\#\textquotesingle{} Gaussian and probit case, backtransform probit h2 and plot}
  \CommentTok{\#\textquotesingle{} @param sA Additive genetic variance sigma\^{}2\_A}
  \CommentTok{\#\textquotesingle{} @param .dichotomize character denoting dichotomization method}
  \CommentTok{\#\textquotesingle{} @param include.fixed Wether or not to include in denom. of h2}
  \CommentTok{\#\textquotesingle{} @param sE Error variance sigma\^{}2\_E}
  \CommentTok{\#\textquotesingle{} @param beta Weight for fixed effect in linear predictor}
  \CommentTok{\#\textquotesingle{} @param SAVE.PLOT Flag for storing plot to disk}
  \CommentTok{\#\textquotesingle{} @param plot.legend Flag for including legend in saved plot}
\NormalTok{  .pc.prior }\OtherTok{\textless{}{-}} \FunctionTok{c}\NormalTok{(}\DecValTok{10}\SpecialCharTok{\^{}}\NormalTok{(}\FunctionTok{ceiling}\NormalTok{(}\FunctionTok{log10}\NormalTok{(sA))), }\FloatTok{0.05}\NormalTok{)}
\NormalTok{  .linear\_predictor\_fixedeffects }\OtherTok{\textless{}{-}} \ControlFlowTok{function}\NormalTok{(u, simulated.d.ped) \{}
\NormalTok{  out }\OtherTok{\textless{}{-}} \FunctionTok{c}\NormalTok{()}
\NormalTok{  intercept }\OtherTok{\textless{}{-}} \DecValTok{0}
\NormalTok{  residuals }\OtherTok{\textless{}{-}} \FunctionTok{rnorm}\NormalTok{(}\FunctionTok{length}\NormalTok{(u), }\AttributeTok{mean =} \DecValTok{0}\NormalTok{, }\AttributeTok{sd =} \FunctionTok{sqrt}\NormalTok{(sE))}
  \ControlFlowTok{for}\NormalTok{ (idx }\ControlFlowTok{in} \FunctionTok{seq\_along}\NormalTok{(u)) \{}
\NormalTok{    out }\OtherTok{\textless{}{-}} \FunctionTok{c}\NormalTok{(}
\NormalTok{      out,}
\NormalTok{      intercept }\SpecialCharTok{+}\NormalTok{ beta }\SpecialCharTok{*}\NormalTok{ simulated.d.ped}\SpecialCharTok{$}\NormalTok{sex[idx] }\SpecialCharTok{+}\NormalTok{ u[idx] }\SpecialCharTok{+}
\NormalTok{        residuals[idx]}
\NormalTok{    )}
\NormalTok{    \}}
\NormalTok{  out}
\NormalTok{  \}}
\NormalTok{  fits }\OtherTok{\textless{}{-}} \FunctionTok{simulated.heritability}\NormalTok{(}
    \AttributeTok{idgen =} \DecValTok{100}\NormalTok{, }\AttributeTok{dichotomize =}\NormalTok{ .dichotomize, }\AttributeTok{pc.prior =}\NormalTok{ .pc.prior,}
    \AttributeTok{sigmaA =}\NormalTok{ sA, }\AttributeTok{linear.predictor =}\NormalTok{ .linear\_predictor\_fixedeffects,}
    \AttributeTok{simulated.formula =}\NormalTok{ simulated.formula.fixedeffects,}
    \AttributeTok{probit.model =} \ConstantTok{TRUE}\NormalTok{,}
    \AttributeTok{simulated.formula.probit =}\NormalTok{ simulated.formula.probit}
\NormalTok{  )}
\NormalTok{  h2.sim.fixed }\OtherTok{\textless{}{-}} \FunctionTok{get.all.heritabilities}\NormalTok{(fits}\SpecialCharTok{$}\NormalTok{fit, fits}\SpecialCharTok{$}\NormalTok{fit.probit,}
\NormalTok{                                         fits}\SpecialCharTok{$}\NormalTok{p, n.samples,}
                                         \AttributeTok{fixed =} \ConstantTok{FALSE}\NormalTok{)}

\NormalTok{  p }\OtherTok{\textless{}{-}} \FunctionTok{ggplot}\NormalTok{(}\FunctionTok{melt}\NormalTok{(h2.sim.fixed[, }\FunctionTok{c}\NormalTok{(}\StringTok{"gaussian"}\NormalTok{, }\StringTok{"probit.qgglmm"}\NormalTok{)])) }\SpecialCharTok{+}
    \FunctionTok{geom\_density}\NormalTok{(}\FunctionTok{aes}\NormalTok{(}\AttributeTok{x =}\NormalTok{ value, }\AttributeTok{fill =}\NormalTok{ variable), }\AttributeTok{alpha =} \FloatTok{0.5}\NormalTok{) }\SpecialCharTok{+}
    \FunctionTok{scale\_fill\_discrete}\NormalTok{(}
      \AttributeTok{name =} \StringTok{""}\NormalTok{,}
      \AttributeTok{labels =} \FunctionTok{c}\NormalTok{(}\FunctionTok{TeX}\NormalTok{(}\StringTok{"Gaussian $h\^{}2\_\{obs\}"}\NormalTok{),}\FunctionTok{TeX}\NormalTok{(}\StringTok{"Probit $h\^{}2\_}\SpecialCharTok{\textbackslash{}\textbackslash{}}\StringTok{Psi$"}\NormalTok{))}
\NormalTok{    ) }\SpecialCharTok{+}
    \FunctionTok{theme}\NormalTok{(}\AttributeTok{legend.text.align =} \DecValTok{0}\NormalTok{) }\SpecialCharTok{+}
    \FunctionTok{xlab}\NormalTok{(}\StringTok{"Heritability"}\NormalTok{) }\SpecialCharTok{+}
    \FunctionTok{ylab}\NormalTok{(}\StringTok{"Density"}\NormalTok{) }\SpecialCharTok{+} \FunctionTok{theme}\NormalTok{(}\AttributeTok{legend.position =} \StringTok{"bottom"}\NormalTok{)}
  \ControlFlowTok{if}\NormalTok{ (SAVE.PLOT) \{}
\NormalTok{    p.legend }\OtherTok{\textless{}{-}}\NormalTok{ cowplot}\SpecialCharTok{::}\FunctionTok{get\_legend}\NormalTok{(p)}
    \FunctionTok{pdf}\NormalTok{(}\StringTok{"../figures/fixedeffects\_gaussian\_probit\_legend.pdf"}\NormalTok{,}
        \AttributeTok{width =} \FloatTok{5.5}\NormalTok{, }\AttributeTok{height =} \DecValTok{1}\NormalTok{)}
    \FunctionTok{grid.newpage}\NormalTok{()}
    \FunctionTok{grid.draw}\NormalTok{(p.legend)}
    \FunctionTok{dev.off}\NormalTok{()}
\NormalTok{    legend.pos }\OtherTok{\textless{}{-}} \ControlFlowTok{if}\NormalTok{(plot.legend) }\StringTok{"right"} \ControlFlowTok{else} \StringTok{"none"}
\NormalTok{    plot.height }\OtherTok{\textless{}{-}} \ControlFlowTok{if}\NormalTok{(plot.legend) }\DecValTok{10} \ControlFlowTok{else} \DecValTok{20}
\NormalTok{    fn\_append }\OtherTok{\textless{}{-}} \ControlFlowTok{if}\NormalTok{(plot.legend) }\StringTok{"\_wide"} \ControlFlowTok{else} \ConstantTok{NULL}
\NormalTok{    fn\_append }\OtherTok{\textless{}{-}} \ControlFlowTok{if}\NormalTok{(beta }\SpecialCharTok{!=} \DecValTok{10}\NormalTok{) }\FunctionTok{paste0}\NormalTok{(fn\_append, }\StringTok{"\_beta\_"}\NormalTok{, beta) }\ControlFlowTok{else}
\NormalTok{      fn\_append}
    \FunctionTok{ggsave}\NormalTok{(}
      \FunctionTok{paste0}\NormalTok{(}\StringTok{"../figures/fixedeffects\_gaussian\_probit\_sA"}\NormalTok{, sA,}
      \StringTok{"\_p\_"}\NormalTok{, }\DecValTok{10}\SpecialCharTok{*}\FunctionTok{round}\NormalTok{(fits}\SpecialCharTok{$}\NormalTok{p,}\DecValTok{1}\NormalTok{), fn\_append, }\StringTok{".pdf"}\NormalTok{),}
\NormalTok{      p}\SpecialCharTok{+}\FunctionTok{theme}\NormalTok{(}\AttributeTok{legend.position =}\NormalTok{ legend.pos), }\AttributeTok{width =} \DecValTok{20}\NormalTok{,}
      \AttributeTok{height =}\NormalTok{ plot.height, }\AttributeTok{units =} \StringTok{"cm"}
\NormalTok{    )}
\NormalTok{  \}}
\NormalTok{\}}
\FunctionTok{plot.fixedeffects.h2}\NormalTok{(}\DecValTok{10}\NormalTok{, }\FloatTok{0.1}\NormalTok{,}\AttributeTok{plot.legend=}\NormalTok{T)}
\FunctionTok{plot.fixedeffects.h2}\NormalTok{(}\DecValTok{10}\NormalTok{, }\FloatTok{0.1}\NormalTok{)}
\FunctionTok{plot.fixedeffects.h2}\NormalTok{(}\DecValTok{10}\NormalTok{, }\StringTok{"round\_balanced"}\NormalTok{)}
\FunctionTok{plot.fixedeffects.h2}\NormalTok{(}\DecValTok{500}\NormalTok{, }\FloatTok{0.1}\NormalTok{)}
\FunctionTok{plot.fixedeffects.h2}\NormalTok{(}\DecValTok{500}\NormalTok{,}\StringTok{"round\_balanced"}\NormalTok{)}

\CommentTok{\# For appendix:}
\ControlFlowTok{for}\NormalTok{(beta\_sex }\ControlFlowTok{in} \FunctionTok{c}\NormalTok{(}\DecValTok{1}\NormalTok{,}\DecValTok{5}\NormalTok{))\{}
  \FunctionTok{plot.fixedeffects.h2}\NormalTok{(}\AttributeTok{sA=}\DecValTok{10}\NormalTok{, }\AttributeTok{.dichotomize =} \StringTok{"round\_balanced"}\NormalTok{,}
                       \AttributeTok{beta=}\NormalTok{beta\_sex)}
  \FunctionTok{plot.fixedeffects.h2}\NormalTok{(}\AttributeTok{sA=}\DecValTok{500}\NormalTok{, }\AttributeTok{.dichotomize =} \StringTok{"round\_balanced"}\NormalTok{,}
                       \AttributeTok{beta=}\NormalTok{beta\_sex)}
\NormalTok{\}}
\end{Highlighting}
\end{Shaded}

\hypertarget{iid-noise-to-probit-simulation}{%
\section*{IID noise to probit
simulation}\label{iid-noise-to-probit-simulation}}

\begin{Shaded}
\begin{Highlighting}[]
\NormalTok{alternative.probit.sim }\OtherTok{\textless{}{-}} \ControlFlowTok{function}\NormalTok{(sigmaA, linear.predictor,}
                                   \AttributeTok{fit.gaussian =} \ConstantTok{NULL}\NormalTok{) \{}
  \CommentTok{\#\textquotesingle{} Simulate and fit model with and without extra noise}
  \CommentTok{\#\textquotesingle{}}
  \CommentTok{\#\textquotesingle{} Modified version of \textasciigrave{}simulated\_heritability()\textasciigrave{} to fit probit}
  \CommentTok{\#\textquotesingle{} model, one with an extra IID noise in INLA formula,}
  \CommentTok{\#\textquotesingle{} and one without.}
  \CommentTok{\#\textquotesingle{} @param sigmaA Additive genetic variance}
  \CommentTok{\#\textquotesingle{} @param linear.predictor Callable of two variables, for}
  \CommentTok{\#\textquotesingle{} simulating response}
  \CommentTok{\#\textquotesingle{} @param fit.gaussian Flag for also fitting Gaussian model.}
  \CommentTok{\#\textquotesingle{} Will fit as long}
  \CommentTok{\#\textquotesingle{} as it\textquotesingle{}s not \textasciigrave{}NULL\textasciigrave{}.}
  \CommentTok{\#\textquotesingle{} @return List of two probit fits, gaussian fit (or \textasciigrave{}NULL\textasciigrave{}) and p,}
  \CommentTok{\#\textquotesingle{} the simulation\textquotesingle{}s phenotypic mean.}

  \CommentTok{\# Init}
\NormalTok{  idgen }\OtherTok{\textless{}{-}} \DecValTok{100}
\NormalTok{  NeNc }\OtherTok{\textless{}{-}} \FloatTok{0.5}
\NormalTok{  nGen }\OtherTok{\textless{}{-}} \DecValTok{9}

  \CommentTok{\# Generate pedigree}
\NormalTok{  ped }\OtherTok{\textless{}{-}} \FunctionTok{generatePedigree}\NormalTok{(}
    \AttributeTok{nId =}\NormalTok{ idgen, }\AttributeTok{nGeneration =}\NormalTok{ nGen, }\AttributeTok{nFather =}\NormalTok{ idgen }\SpecialCharTok{*}\NormalTok{ NeNc,}
    \AttributeTok{nMother =}\NormalTok{ idgen }\SpecialCharTok{*}\NormalTok{ NeNc}
\NormalTok{  )}
\NormalTok{  ped }\OtherTok{\textless{}{-}}\NormalTok{ ped[, }\FunctionTok{c}\NormalTok{(}\DecValTok{1}\NormalTok{, }\DecValTok{3}\NormalTok{, }\DecValTok{2}\NormalTok{, }\DecValTok{5}\NormalTok{)]}
  \FunctionTok{names}\NormalTok{(ped) }\OtherTok{\textless{}{-}} \FunctionTok{c}\NormalTok{(}\StringTok{"id"}\NormalTok{, }\StringTok{"dam"}\NormalTok{, }\StringTok{"sire"}\NormalTok{, }\StringTok{"sex"}\NormalTok{)}
\NormalTok{  u }\OtherTok{\textless{}{-}} \FunctionTok{rbv}\NormalTok{(ped[, }\FunctionTok{c}\NormalTok{(}\DecValTok{1}\NormalTok{, }\DecValTok{2}\NormalTok{, }\DecValTok{3}\NormalTok{)], sigmaA)}
\NormalTok{  simulated.d.ped }\OtherTok{\textless{}{-}}\NormalTok{ nadiv}\SpecialCharTok{::}\FunctionTok{prepPed}\NormalTok{(ped, }\AttributeTok{gender =} \StringTok{"sex"}\NormalTok{)}
\NormalTok{  simulated.Cmatrix }\OtherTok{\textless{}{-}}\NormalTok{ nadiv}\SpecialCharTok{::}\FunctionTok{makeAinv}\NormalTok{(ped[, }\FunctionTok{c}\NormalTok{(}\DecValTok{1}\NormalTok{, }\DecValTok{2}\NormalTok{, }\DecValTok{3}\NormalTok{)])}\SpecialCharTok{$}\NormalTok{Ainv}
\NormalTok{  simulated.d.ped}\SpecialCharTok{$}\NormalTok{ind }\OtherTok{\textless{}{-}} \FunctionTok{seq\_len}\NormalTok{(}\FunctionTok{nrow}\NormalTok{(simulated.d.ped))}

  \CommentTok{\# Generate binary response}
\NormalTok{  simulated.response }\OtherTok{\textless{}{-}} \FunctionTok{ifelse}\NormalTok{(}
    \FunctionTok{linear.predictor}\NormalTok{(u, simulated.d.ped) }\SpecialCharTok{\textless{}=} \DecValTok{0}\NormalTok{, }\DecValTok{0}\NormalTok{, }\DecValTok{1}\NormalTok{)}
\NormalTok{  p }\OtherTok{\textless{}{-}} \FunctionTok{mean}\NormalTok{(simulated.response)}

  \CommentTok{\# INLA fitting}
\NormalTok{  formula.overdisp }\OtherTok{\textless{}{-}}\NormalTok{ simulated.response }\SpecialCharTok{\textasciitilde{}} \FunctionTok{f}\NormalTok{(id,}
    \AttributeTok{model =} \StringTok{"generic0"}\NormalTok{, }\AttributeTok{Cmatrix =}\NormalTok{ simulated.Cmatrix,}
    \AttributeTok{constr =} \ConstantTok{FALSE}\NormalTok{,}
    \AttributeTok{hyper =} \FunctionTok{list}\NormalTok{(}\AttributeTok{prec =} \FunctionTok{list}\NormalTok{(}
      \AttributeTok{initial =} \FunctionTok{log}\NormalTok{(}\DecValTok{1} \SpecialCharTok{/} \DecValTok{10}\NormalTok{), }\AttributeTok{prior =} \StringTok{"pc.prec"}\NormalTok{,}
      \AttributeTok{param =} \FunctionTok{c}\NormalTok{(}\DecValTok{1}\NormalTok{, }\FloatTok{0.05}\NormalTok{)}
\NormalTok{    ))}
\NormalTok{  ) }\SpecialCharTok{+}
    \FunctionTok{f}\NormalTok{(ind,}
      \AttributeTok{model =} \StringTok{"iid"}\NormalTok{, }\AttributeTok{constr =} \ConstantTok{TRUE}
\NormalTok{    )}
\NormalTok{  formula.standard }\OtherTok{\textless{}{-}}\NormalTok{ simulated.response }\SpecialCharTok{\textasciitilde{}} \FunctionTok{f}\NormalTok{(id,}
    \AttributeTok{model =} \StringTok{"generic0"}\NormalTok{, }\AttributeTok{Cmatrix =}\NormalTok{ simulated.Cmatrix,}
    \AttributeTok{constr =} \ConstantTok{FALSE}\NormalTok{,}
    \AttributeTok{hyper =} \FunctionTok{list}\NormalTok{(}\AttributeTok{prec =} \FunctionTok{list}\NormalTok{(}
      \AttributeTok{initial =} \FunctionTok{log}\NormalTok{(}\DecValTok{1} \SpecialCharTok{/} \DecValTok{10}\NormalTok{), }\AttributeTok{prior =} \StringTok{"pc.prec"}\NormalTok{,}
      \AttributeTok{param =} \FunctionTok{c}\NormalTok{(}\DecValTok{1}\NormalTok{, }\FloatTok{0.05}\NormalTok{)}
\NormalTok{    ))}
\NormalTok{  )}


\NormalTok{  fit.overdisp }\OtherTok{\textless{}{-}} \FunctionTok{inla}\NormalTok{(}
    \AttributeTok{formula =}\NormalTok{ formula.overdisp, }\AttributeTok{family =} \StringTok{"binomial"}\NormalTok{,}
    \AttributeTok{data =}\NormalTok{ simulated.d.ped,}
    \AttributeTok{control.compute =} \FunctionTok{list}\NormalTok{(}\AttributeTok{return.marginals.predictor =} \ConstantTok{TRUE}\NormalTok{)}
\NormalTok{  )}
\NormalTok{  fit.standard }\OtherTok{\textless{}{-}} \FunctionTok{inla}\NormalTok{(}
    \AttributeTok{formula =}\NormalTok{ formula.standard, }\AttributeTok{family =} \StringTok{"binomial"}\NormalTok{,}
    \AttributeTok{data =}\NormalTok{ simulated.d.ped,}
    \AttributeTok{control.compute =} \FunctionTok{list}\NormalTok{(}\AttributeTok{return.marginals.predictor =} \ConstantTok{TRUE}\NormalTok{)}
\NormalTok{  )}
  \ControlFlowTok{if}\NormalTok{ (}\SpecialCharTok{!}\FunctionTok{is.null}\NormalTok{(fit.gaussian)) \{}
    \CommentTok{\# Also compute Gaussian model}
\NormalTok{    fit.gaussian }\OtherTok{\textless{}{-}} \FunctionTok{inla}\NormalTok{(}
      \AttributeTok{formula =}\NormalTok{ formula.standard, }\AttributeTok{family =} \StringTok{"gaussian"}\NormalTok{,}
      \AttributeTok{data =}\NormalTok{ simulated.d.ped}
\NormalTok{    )}
\NormalTok{  \}}


  \FunctionTok{list}\NormalTok{(}
    \AttributeTok{fit.overdisp =}\NormalTok{ fit.overdisp,}
    \AttributeTok{fit.standard =}\NormalTok{ fit.standard,}
    \AttributeTok{fit.gaussian =}\NormalTok{ fit.gaussian,}
    \AttributeTok{overdisp.p =}\NormalTok{ p}
\NormalTok{  )}
\NormalTok{\}}
\end{Highlighting}
\end{Shaded}

Here, we fit a probit with the formula
\(y_i = \beta_0 + a_i + \gamma_{0,i}\), where the last is a random iid
effect. The simulated data has overdisperion in its data via the
residual vector being \(\mathcal N(0,3)\).

\begin{Shaded}
\begin{Highlighting}[]
\NormalTok{overdisperion\_wrapper }\OtherTok{\textless{}{-}} \ControlFlowTok{function}\NormalTok{(nsamps, vA, vE, SAVE.PLOT) \{}
  \CommentTok{\#\textquotesingle{} Wrapper for running overdispersion tests}
  \CommentTok{\#\textquotesingle{}}
  \CommentTok{\#\textquotesingle{} Wrapper for calling alternative model fitting with extra noise,}
  \CommentTok{\#\textquotesingle{} and for}
  \CommentTok{\#\textquotesingle{} plotting thee results.}
  \CommentTok{\#\textquotesingle{} @param nsamps Number of samples for posterior heritability}
  \CommentTok{\#\textquotesingle{} @param vA Additive genetic variance}
  \CommentTok{\#\textquotesingle{} @param vE Additional noise (should be more than 1)}
  \CommentTok{\#\textquotesingle{} @param SAVE.PLOT Flag for storing plot to disk}

  \FunctionTok{list2env}\NormalTok{( }\CommentTok{\# Loads fit.overdisp, fit.standard, fit.gaussian, p}
    \FunctionTok{alternative.probit.sim}\NormalTok{(vA, }\ControlFlowTok{function}\NormalTok{(u, .) u }\SpecialCharTok{+} \FunctionTok{rnorm}\NormalTok{(}\FunctionTok{length}\NormalTok{(u),}
                                                        \DecValTok{0}\NormalTok{, }\FunctionTok{sqrt}\NormalTok{(vE)),}
      \AttributeTok{fit.gaussian =} \ConstantTok{TRUE}
\NormalTok{    ), .GlobalEnv}
\NormalTok{  )}
\NormalTok{  df.probit.comp }\OtherTok{\textless{}{-}} \FunctionTok{data.frame}\NormalTok{(}
    \AttributeTok{Overdispersion =} \FunctionTok{get.h2.from.qgparams}\NormalTok{(fit.overdisp,}
                                          \StringTok{"binom1.probit"}\NormalTok{,}
\NormalTok{      nsamps,}
      \AttributeTok{averaging =} \ConstantTok{TRUE}
\NormalTok{    ),}
    \AttributeTok{Standard =} \FunctionTok{get.h2.from.qgparams}\NormalTok{(fit.standard, }\StringTok{"binom1.probit"}\NormalTok{,}
\NormalTok{      nsamps,}
      \AttributeTok{averaging =} \ConstantTok{TRUE}
\NormalTok{    ),}
    \AttributeTok{Gaussian =} \FunctionTok{get.h2}\NormalTok{(fit.gaussian, nsamps)}
\NormalTok{  )}

\NormalTok{  curr.plot }\OtherTok{\textless{}{-}} \FunctionTok{ggplot}\NormalTok{(}\AttributeTok{data =} \FunctionTok{melt}\NormalTok{(df.probit.comp)) }\SpecialCharTok{+}
    \FunctionTok{geom\_density}\NormalTok{(}\FunctionTok{aes}\NormalTok{(}\AttributeTok{x =}\NormalTok{ value, }\AttributeTok{fill =}\NormalTok{ variable), }\AttributeTok{alpha =} \FloatTok{0.5}\NormalTok{) }\SpecialCharTok{+}
    \FunctionTok{scale\_fill\_discrete}\NormalTok{(}\AttributeTok{name =} \StringTok{""}\NormalTok{, }\AttributeTok{labels =} \FunctionTok{c}\NormalTok{(}
      \FunctionTok{TeX}\NormalTok{(}\StringTok{"$h\^{}2\_}\SpecialCharTok{\textbackslash{}\textbackslash{}}\StringTok{Psi$ with iid effect"}\NormalTok{),}
      \FunctionTok{TeX}\NormalTok{(}\StringTok{"$h\^{}2\_}\SpecialCharTok{\textbackslash{}\textbackslash{}}\StringTok{Psi$ without iid effect"}\NormalTok{),}
      \FunctionTok{TeX}\NormalTok{(}\StringTok{"$h\^{}2\_\{obs\}$"}\NormalTok{)}
\NormalTok{    )) }\SpecialCharTok{+}
    \FunctionTok{xlab}\NormalTok{(}\StringTok{"Heritability"}\NormalTok{) }\SpecialCharTok{+}
    \FunctionTok{ylab}\NormalTok{(}\StringTok{"Density"}\NormalTok{) }\SpecialCharTok{+}
    \FunctionTok{theme}\NormalTok{(}\AttributeTok{legend.position =} \StringTok{"bottom"}\NormalTok{) }\SpecialCharTok{+}
    \FunctionTok{xlim}\NormalTok{(}\FunctionTok{c}\NormalTok{(}\DecValTok{0}\NormalTok{, }\FunctionTok{quantile}\NormalTok{(}\FunctionTok{melt}\NormalTok{(df.probit.comp)}\SpecialCharTok{$}\NormalTok{value, }\FloatTok{0.95}\NormalTok{)))}
  \ControlFlowTok{if}\NormalTok{ (SAVE.PLOT) \{}
\NormalTok{    p.legend }\OtherTok{\textless{}{-}}\NormalTok{ cowplot}\SpecialCharTok{::}\FunctionTok{get\_legend}\NormalTok{(curr.plot)}
    \FunctionTok{pdf}\NormalTok{(}\StringTok{"../figures/overdisperions\_legend.pdf"}\NormalTok{, }\AttributeTok{width =} \DecValTok{9}\NormalTok{, }\AttributeTok{height =} \DecValTok{1}\NormalTok{)}
    \FunctionTok{grid.newpage}\NormalTok{()}
    \FunctionTok{grid.draw}\NormalTok{(p.legend)}
    \FunctionTok{dev.off}\NormalTok{()}
    \FunctionTok{ggsave}\NormalTok{(}
      \FunctionTok{paste0}\NormalTok{(}\StringTok{"../figures/overdispersion\_vE{-}vA\_"}\NormalTok{, vE, }\StringTok{"{-}"}\NormalTok{, vA, }\StringTok{".pdf"}\NormalTok{),}
\NormalTok{      curr.plot }\SpecialCharTok{+} \FunctionTok{theme}\NormalTok{(}\AttributeTok{legend.position =} \StringTok{"none"}\NormalTok{)}
\NormalTok{    )}
\NormalTok{  \}}
\NormalTok{\}}

\FunctionTok{overdisperion\_wrapper}\NormalTok{(}\DecValTok{10000}\NormalTok{, }\FloatTok{0.5}\NormalTok{, }\DecValTok{2}\NormalTok{, SAVE.PLOT)}
\FunctionTok{overdisperion\_wrapper}\NormalTok{(}\DecValTok{10000}\NormalTok{, }\FloatTok{0.5}\NormalTok{, }\DecValTok{5}\NormalTok{, SAVE.PLOT)}
\FunctionTok{overdisperion\_wrapper}\NormalTok{(}\DecValTok{10000}\NormalTok{, }\FloatTok{0.5}\NormalTok{, }\DecValTok{10}\NormalTok{, SAVE.PLOT)}
\FunctionTok{overdisperion\_wrapper}\NormalTok{(}\DecValTok{10000}\NormalTok{, }\DecValTok{10}\NormalTok{, }\DecValTok{10}\NormalTok{, SAVE.PLOT)}
\end{Highlighting}
\end{Shaded}

\hypertarget{illustrative-figures}{%
\section*{Illustrative figures}\label{illustrative-figures}}

Addendum - Illustrative figure for fitted values in a probit model. This
is intended to demonstrate the different scales you get when using
GLMMs.

\begin{Shaded}
\begin{Highlighting}[]
\NormalTok{mod }\OtherTok{\textless{}{-}} \FunctionTok{simulated.heritability}\NormalTok{(}
  \AttributeTok{linear.predictor =} \ControlFlowTok{function}\NormalTok{(u, .) u }\SpecialCharTok{+} \FunctionTok{rnorm}\NormalTok{(}\FunctionTok{length}\NormalTok{(u)),}
  \AttributeTok{simulated.formula =}\NormalTok{ simulated.formula,}
  \AttributeTok{probit.model =}\NormalTok{ T,}
  \AttributeTok{simulated.formula.probit =}\NormalTok{ simulated.formula.probit}
\NormalTok{)}
\NormalTok{pmod }\OtherTok{\textless{}{-}}\NormalTok{ mod}\SpecialCharTok{$}\NormalTok{fit.probit}
\NormalTok{eta\_samples }\OtherTok{\textless{}{-}} \FunctionTok{marginal.latent.samples}\NormalTok{(pmod, }\DecValTok{20}\NormalTok{)}

\CommentTok{\# Plotting}
\NormalTok{sample\_ids }\OtherTok{\textless{}{-}} \FunctionTok{sort}\NormalTok{(}\FunctionTok{c}\NormalTok{(LETTERS, letters))[}\DecValTok{1}\SpecialCharTok{:}\DecValTok{40}\NormalTok{]}

\NormalTok{eta\_df }\OtherTok{\textless{}{-}} \FunctionTok{data.frame}\NormalTok{(}
  \AttributeTok{elem =} \FunctionTok{c}\NormalTok{(}
    \FunctionTok{rep}\NormalTok{(sample\_ids[}\DecValTok{1}\SpecialCharTok{:}\DecValTok{20}\NormalTok{], }\AttributeTok{each =} \DecValTok{900}\NormalTok{), }\CommentTok{\# Latent eta}
    \FunctionTok{rep}\NormalTok{(sample\_ids[}\DecValTok{21}\SpecialCharTok{:}\DecValTok{40}\NormalTok{], }\AttributeTok{each =} \DecValTok{900}\NormalTok{)), }\CommentTok{\# Phi(eta)}
  \AttributeTok{value =} \FunctionTok{c}\NormalTok{(}
    \FunctionTok{unlist}\NormalTok{(eta\_samples),}
    \FunctionTok{pnorm}\NormalTok{(}\FunctionTok{unlist}\NormalTok{(eta\_samples))}
\NormalTok{  )}
\NormalTok{)}
\NormalTok{custom\_palette }\OtherTok{\textless{}{-}} \FunctionTok{c}\NormalTok{(}
  \FunctionTok{colorRampPalette}\NormalTok{(}\FunctionTok{c}\NormalTok{(}\StringTok{"pink"}\NormalTok{, }\StringTok{"darkred"}\NormalTok{))(}\DecValTok{20}\NormalTok{),}
  \FunctionTok{colorRampPalette}\NormalTok{(}\FunctionTok{c}\NormalTok{(}\StringTok{"lightblue"}\NormalTok{, }\StringTok{"darkblue"}\NormalTok{))(}\DecValTok{20}\NormalTok{), }\StringTok{"darkgreen"}
\NormalTok{)}

\FunctionTok{ggplot}\NormalTok{(eta\_df) }\SpecialCharTok{+}
  \FunctionTok{geom\_line}\NormalTok{(}\FunctionTok{aes}\NormalTok{(}\AttributeTok{x =}\NormalTok{ value, }\AttributeTok{color =}\NormalTok{ elem), }\AttributeTok{stat =} \StringTok{"density"}\NormalTok{,}
            \AttributeTok{alpha =} \FloatTok{0.25}\NormalTok{, }\AttributeTok{size =} \DecValTok{2}\NormalTok{) }\SpecialCharTok{+}
  \CommentTok{\# True values}
  \FunctionTok{geom\_vline}\NormalTok{(}\AttributeTok{xintercept=}\DecValTok{0}\NormalTok{,}\AttributeTok{size=}\DecValTok{2}\NormalTok{,}\AttributeTok{color=}\StringTok{\textquotesingle{}green4\textquotesingle{}}\NormalTok{) }\SpecialCharTok{+}
  \FunctionTok{geom\_vline}\NormalTok{(}\AttributeTok{xintercept=}\DecValTok{1}\NormalTok{, }\AttributeTok{size=}\DecValTok{2}\NormalTok{, }\AttributeTok{color=}\StringTok{\textquotesingle{}green4\textquotesingle{}}\NormalTok{) }\SpecialCharTok{+}
  \FunctionTok{scale\_color\_manual}\NormalTok{(}\AttributeTok{values =}\NormalTok{ custom\_palette) }\SpecialCharTok{+}
  \FunctionTok{theme}\NormalTok{(}\AttributeTok{legend.position =} \StringTok{"none"}\NormalTok{) }\SpecialCharTok{+}
  \FunctionTok{xlab}\NormalTok{(}\StringTok{"(Predicted) response"}\NormalTok{) }\SpecialCharTok{+}
  \FunctionTok{ylab}\NormalTok{(}\StringTok{"Density"}\NormalTok{)}
\ControlFlowTok{if}\NormalTok{ (SAVE.PLOT) \{}
  \FunctionTok{ggsave}\NormalTok{(}\StringTok{"../figures/illustration\_probit\_scales\_fitted\_values.pdf"}\NormalTok{,}
    \AttributeTok{width =} \DecValTok{20}\NormalTok{, }\AttributeTok{height =} \DecValTok{10}\NormalTok{, }\AttributeTok{units =} \StringTok{"cm"}
\NormalTok{  )}
  \CommentTok{\# Generate legend}
\NormalTok{  plegend }\OtherTok{\textless{}{-}}\NormalTok{ ggpubr}\SpecialCharTok{::}\FunctionTok{get\_legend}\NormalTok{(}
    \FunctionTok{ggplot}\NormalTok{(}\FunctionTok{melt}\NormalTok{(}\FunctionTok{data.frame}\NormalTok{(}\AttributeTok{r =} \FunctionTok{rnorm}\NormalTok{(}\DecValTok{1}\NormalTok{), }\AttributeTok{b =} \FunctionTok{rnorm}\NormalTok{(}\DecValTok{1}\NormalTok{),}
                           \AttributeTok{t =} \FunctionTok{rnorm}\NormalTok{(}\DecValTok{1}\NormalTok{)))) }\SpecialCharTok{+}
      \FunctionTok{geom\_line}\NormalTok{(}\FunctionTok{aes}\NormalTok{(}\AttributeTok{x =}\NormalTok{ value, }\AttributeTok{color =}\NormalTok{ variable), }\AttributeTok{stat =} \StringTok{"density"}\NormalTok{,}
                \AttributeTok{size =} \DecValTok{2}\NormalTok{, }\AttributeTok{alpha =} \FloatTok{0.9}\NormalTok{) }\SpecialCharTok{+}
      \FunctionTok{scale\_color\_manual}\NormalTok{(}
        \AttributeTok{name =} \StringTok{""}\NormalTok{,}
        \AttributeTok{labels =} \FunctionTok{c}\NormalTok{(}\FunctionTok{TeX}\NormalTok{(}\StringTok{"$}\SpecialCharTok{\textbackslash{}\textbackslash{}}\StringTok{eta$"}\NormalTok{), }\FunctionTok{TeX}\NormalTok{(}\StringTok{"$}\SpecialCharTok{\textbackslash{}\textbackslash{}}\StringTok{Phi(}\SpecialCharTok{\textbackslash{}\textbackslash{}}\StringTok{eta)$"}\NormalTok{),}
                   \StringTok{"True observations"}\NormalTok{),}
        \AttributeTok{values =} \FunctionTok{c}\NormalTok{(}\AttributeTok{r =} \StringTok{"darkred"}\NormalTok{, }\AttributeTok{b =} \StringTok{"darkblue"}\NormalTok{, }\AttributeTok{t =} \StringTok{"green4"}\NormalTok{)}
\NormalTok{      ) }\SpecialCharTok{+}
      \FunctionTok{theme}\NormalTok{(}\AttributeTok{legend.text.align =} \DecValTok{0}\NormalTok{, }\AttributeTok{legend.position =} \StringTok{"bottom"}\NormalTok{,}
            \AttributeTok{text =} \FunctionTok{element\_text}\NormalTok{(}\AttributeTok{family =}\NormalTok{ texfont)}
\NormalTok{      )}
\NormalTok{  )}
  \FunctionTok{pdf}\NormalTok{(}\StringTok{"../figures/illustration\_probit\_scales\_fitted\_values\_legend.pdf"}\NormalTok{,}
    \AttributeTok{width =} \FloatTok{4.2}\NormalTok{, }\AttributeTok{height =} \DecValTok{1}
\NormalTok{  )}
  \FunctionTok{grid.newpage}\NormalTok{()}
  \FunctionTok{grid.draw}\NormalTok{(plegend)}
  \FunctionTok{dev.off}\NormalTok{()}
\NormalTok{\}}
\end{Highlighting}
\end{Shaded}

The second is an illustrative figure for binomial vs linear regression
in general.

\begin{Shaded}
\begin{Highlighting}[]
\FunctionTok{data}\NormalTok{(mtcars)}
\FunctionTok{library}\NormalTok{(ggplot2)}
\FunctionTok{library}\NormalTok{(cowplot)}
\NormalTok{p1 }\OtherTok{\textless{}{-}} \FunctionTok{ggplot}\NormalTok{(mtcars, }\FunctionTok{aes}\NormalTok{(}\AttributeTok{x =}\NormalTok{ hp, }\AttributeTok{y =}\NormalTok{ vs)) }\SpecialCharTok{+}
  \FunctionTok{geom\_point}\NormalTok{(}\AttributeTok{alpha =}\NormalTok{ .}\DecValTok{5}\NormalTok{) }\SpecialCharTok{+}
  \FunctionTok{ggtitle}\NormalTok{(}\StringTok{"Binomial regression"}\NormalTok{) }\SpecialCharTok{+}
  \FunctionTok{stat\_smooth}\NormalTok{(}\AttributeTok{method =} \StringTok{"glm"}\NormalTok{, }\AttributeTok{se =}\NormalTok{ F,}
              \AttributeTok{method.args =} \FunctionTok{list}\NormalTok{(}\AttributeTok{family =}\NormalTok{ binomial)) }\SpecialCharTok{+}
  \FunctionTok{ylim}\NormalTok{(}\FunctionTok{c}\NormalTok{(}\SpecialCharTok{{-}}\FloatTok{0.2}\NormalTok{, }\FloatTok{1.01}\NormalTok{)) }\SpecialCharTok{+}
  \FunctionTok{theme}\NormalTok{(}\AttributeTok{title =} \FunctionTok{element\_text}\NormalTok{(}\AttributeTok{size =} \DecValTok{16}\NormalTok{))}
\NormalTok{p2 }\OtherTok{\textless{}{-}} \FunctionTok{ggplot}\NormalTok{(mtcars, }\FunctionTok{aes}\NormalTok{(}\AttributeTok{x =}\NormalTok{ hp, }\AttributeTok{y =}\NormalTok{ vs)) }\SpecialCharTok{+}
  \FunctionTok{geom\_point}\NormalTok{(}\AttributeTok{alpha =}\NormalTok{ .}\DecValTok{5}\NormalTok{) }\SpecialCharTok{+}
  \FunctionTok{stat\_smooth}\NormalTok{(}\AttributeTok{method =} \StringTok{"lm"}\NormalTok{, }\AttributeTok{se =}\NormalTok{ F) }\SpecialCharTok{+}
  \FunctionTok{ggtitle}\NormalTok{(}\StringTok{"Linear regression"}\NormalTok{) }\SpecialCharTok{+}
  \FunctionTok{ylim}\NormalTok{(}\FunctionTok{c}\NormalTok{(}\SpecialCharTok{{-}}\FloatTok{0.2}\NormalTok{, }\FloatTok{1.01}\NormalTok{)) }\SpecialCharTok{+}
  \FunctionTok{theme}\NormalTok{(}\AttributeTok{title =} \FunctionTok{element\_text}\NormalTok{(}\AttributeTok{size =} \DecValTok{16}\NormalTok{))}

\FunctionTok{plot\_grid}\NormalTok{(p1, p2, }\AttributeTok{ncol =} \DecValTok{2}\NormalTok{, }\AttributeTok{align =} \StringTok{"v"}\NormalTok{, }\AttributeTok{axis =} \StringTok{"tb"}\NormalTok{)}
\ControlFlowTok{if}\NormalTok{ (SAVE.PLOT) \{}
  \FunctionTok{ggsave}\NormalTok{(}\StringTok{"../figures/linear{-}vs{-}logistic{-}example.pdf"}\NormalTok{,}
    \AttributeTok{width =} \DecValTok{20}\NormalTok{, }\AttributeTok{height =} \DecValTok{10}\NormalTok{, }\AttributeTok{units =} \StringTok{"cm"}
\NormalTok{  )}
\NormalTok{\}}
\end{Highlighting}
\end{Shaded}

The final plot is an illustration of the threshold model.

\begin{Shaded}
\begin{Highlighting}[]
\NormalTok{threshold.illustration }\OtherTok{\textless{}{-}} \ControlFlowTok{function}\NormalTok{()\{}
\NormalTok{  x }\OtherTok{\textless{}{-}} \FunctionTok{seq}\NormalTok{(}\SpecialCharTok{{-}}\DecValTok{3}\NormalTok{, }\DecValTok{3}\NormalTok{, }\AttributeTok{length.out =} \DecValTok{100}\NormalTok{)}
\NormalTok{  threshold }\OtherTok{\textless{}{-}} \SpecialCharTok{{-}}\FloatTok{0.4}
\NormalTok{  df }\OtherTok{\textless{}{-}} \FunctionTok{data.frame}\NormalTok{(}\AttributeTok{x =}\NormalTok{ x, }\AttributeTok{y =} \FunctionTok{dnorm}\NormalTok{(x))}
\NormalTok{  df}\SpecialCharTok{$}\NormalTok{samps }\OtherTok{\textless{}{-}} \FunctionTok{c}\NormalTok{(}\FunctionTok{rep}\NormalTok{(}\ConstantTok{NA}\NormalTok{,}\DecValTok{10}\NormalTok{), }\FunctionTok{runif}\NormalTok{(}\DecValTok{80}\NormalTok{, }\DecValTok{0}\NormalTok{, }\FunctionTok{dnorm}\NormalTok{(x[}\DecValTok{11}\SpecialCharTok{:}\DecValTok{90}\NormalTok{])), }\FunctionTok{rep}\NormalTok{(}\ConstantTok{NA}\NormalTok{,}\DecValTok{10}\NormalTok{))}
\NormalTok{  df}\SpecialCharTok{$}\NormalTok{sampscol }\OtherTok{\textless{}{-}} \FunctionTok{ifelse}\NormalTok{(df}\SpecialCharTok{$}\NormalTok{x }\SpecialCharTok{\textgreater{}=}\NormalTok{ threshold, }\StringTok{"a"}\NormalTok{, }\StringTok{"b"}\NormalTok{)}
\NormalTok{  cols }\OtherTok{\textless{}{-}} \FunctionTok{c}\NormalTok{(}\FunctionTok{hcl}\NormalTok{(}\AttributeTok{h=}\FunctionTok{seq}\NormalTok{(}\DecValTok{15}\NormalTok{,}\DecValTok{375}\NormalTok{,}\AttributeTok{length=}\DecValTok{3}\NormalTok{), }\AttributeTok{l=}\DecValTok{65}\NormalTok{, }\AttributeTok{c=}\DecValTok{100}\NormalTok{)[}\DecValTok{1}\SpecialCharTok{:}\DecValTok{2}\NormalTok{])}
  \CommentTok{\# Create the ggplot}
  \FunctionTok{ggplot}\NormalTok{(df, }\FunctionTok{aes}\NormalTok{(}\AttributeTok{x =}\NormalTok{ x, }\AttributeTok{y =}\NormalTok{ y)) }\SpecialCharTok{+}
    \FunctionTok{geom\_point}\NormalTok{(}\FunctionTok{aes}\NormalTok{(}\AttributeTok{x =}\NormalTok{ x, }\AttributeTok{y =}\NormalTok{ samps, }\AttributeTok{colour =}\NormalTok{ sampscol)) }\SpecialCharTok{+}
    \FunctionTok{geom\_line}\NormalTok{(}\AttributeTok{linewidth=}\FloatTok{0.7}\NormalTok{) }\SpecialCharTok{+} \FunctionTok{ylab}\NormalTok{(}\StringTok{""}\NormalTok{) }\SpecialCharTok{+} \FunctionTok{xlab}\NormalTok{(}\StringTok{""}\NormalTok{) }\SpecialCharTok{+}
    \FunctionTok{geom\_vline}\NormalTok{(}\FunctionTok{aes}\NormalTok{(}\AttributeTok{xintercept=}\NormalTok{threshold, }\AttributeTok{linetype=}\StringTok{\textquotesingle{}Threshold M\textquotesingle{}}\NormalTok{),}
               \AttributeTok{linewidth=}\FloatTok{0.7}\NormalTok{) }\SpecialCharTok{+}
    \FunctionTok{theme\_classic}\NormalTok{(}\DecValTok{24}\NormalTok{) }\SpecialCharTok{+} \FunctionTok{theme}\NormalTok{(}\AttributeTok{text=}\FunctionTok{element\_text}\NormalTok{(}\AttributeTok{family=}\NormalTok{texfont)) }\SpecialCharTok{+}
    \FunctionTok{scale\_color\_manual}\NormalTok{(}\AttributeTok{name=}\StringTok{""}\NormalTok{, }\AttributeTok{values=}\NormalTok{cols,}
                       \AttributeTok{labels=}\FunctionTok{c}\NormalTok{(}\StringTok{"Phenotype 2"}\NormalTok{,}\StringTok{"Phenotype 1"}\NormalTok{)) }\SpecialCharTok{+}
    \FunctionTok{scale\_linetype\_manual}\NormalTok{(}\AttributeTok{name=}\StringTok{""}\NormalTok{, }\AttributeTok{values=}\FunctionTok{c}\NormalTok{(}\StringTok{\textquotesingle{}Threshold M\textquotesingle{}}\OtherTok{=}\DecValTok{2}\NormalTok{))}
  \FunctionTok{ggsave}\NormalTok{(}\StringTok{"../figures/illustration\_thresholdmodel.pdf"}\NormalTok{,}
         \AttributeTok{width=}\DecValTok{20}\NormalTok{, }\AttributeTok{height=}\DecValTok{10}\NormalTok{, }\AttributeTok{units=}\StringTok{"cm"}\NormalTok{)  }
\NormalTok{\}}
\FunctionTok{threshold.illustration}\NormalTok{()}
\end{Highlighting}
\end{Shaded}
