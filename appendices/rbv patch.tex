\hypertarget{how-the-bug-arises}{%
\section*{How the bug arises}\label{how-the-bug-arises}}

The function \texttt{rbv} generates random breeding values based on a
pedigree. The pedigree must either be a data frame with columns
\emph{id, dam, sire}, or a \texttt{phylo} object. The issue arises when
using GeneticsPed to generate the pedigree since this returns a
multi-class object rather than just one single object.

\hypertarget{minimal-viable-product-to-reproduce-issue}{%
\section*{Minimal viable product to reproduce
issue}\label{minimal-viable-product-to-reproduce-issue}}

\begin{Shaded}
\begin{Highlighting}[]
\FunctionTok{library}\NormalTok{(GeneticsPed)}
\end{Highlighting}
\end{Shaded}

\begin{verbatim}
## Warning: package 'GeneticsPed' was built under R version 4.2.2
\end{verbatim}

\begin{verbatim}
## Loading required package: MASS
\end{verbatim}

\begin{verbatim}
## 
## Attaching package: 'GeneticsPed'
\end{verbatim}

\begin{verbatim}
## The following object is masked from 'package:stats':
## 
##     family
\end{verbatim}

\begin{Shaded}
\begin{Highlighting}[]
\FunctionTok{library}\NormalTok{(MCMCglmm)}
\end{Highlighting}
\end{Shaded}

\begin{verbatim}
## Loading required package: Matrix
\end{verbatim}

\begin{verbatim}
## Warning: package 'Matrix' was built under R version 4.2.2
\end{verbatim}

\begin{verbatim}
## Loading required package: coda
\end{verbatim}

\begin{verbatim}
## Warning: package 'coda' was built under R version 4.2.2
\end{verbatim}

\begin{verbatim}
## Loading required package: ape
\end{verbatim}

\begin{verbatim}
## Warning: package 'ape' was built under R version 4.2.2
\end{verbatim}

\begin{Shaded}
\begin{Highlighting}[]
\NormalTok{ped0 }\OtherTok{\textless{}{-}} \FunctionTok{generatePedigree}\NormalTok{(}\AttributeTok{nId =} \DecValTok{100}\NormalTok{, }\AttributeTok{nGeneration =} \DecValTok{9}\NormalTok{, }
                         \AttributeTok{nFather =} \DecValTok{50}\NormalTok{, }\AttributeTok{nMother =} \DecValTok{50}\NormalTok{)}
\NormalTok{pedigree }\OtherTok{\textless{}{-}}\NormalTok{ ped0[ , }\FunctionTok{c}\NormalTok{(}\DecValTok{1}\NormalTok{,}\DecValTok{3}\NormalTok{,}\DecValTok{2}\NormalTok{)]}
\FunctionTok{names}\NormalTok{(pedigree) }\OtherTok{\textless{}{-}} \FunctionTok{c}\NormalTok{(}\StringTok{"id"}\NormalTok{, }\StringTok{"dam"}\NormalTok{, }\StringTok{"sire"}\NormalTok{)}
\FunctionTok{attr}\NormalTok{(pedigree, }\StringTok{"class"}\NormalTok{) }\CommentTok{\# Contains 2 classes}
\end{Highlighting}
\end{Shaded}

\begin{verbatim}
## [1] "Pedigree"   "data.frame"
\end{verbatim}

Trying to use rbv will end in an error

\begin{Shaded}
\begin{Highlighting}[]
\NormalTok{u }\OtherTok{\textless{}{-}} \FunctionTok{rbv}\NormalTok{(pedigree, }\FloatTok{0.4}\NormalTok{)}
\end{Highlighting}
\end{Shaded}

The error code is

\begin{verbatim}
Error in if (attr(pedigree, "class") == "phylo") { : 
  the condition has length > 1
\end{verbatim}

\hypertarget{patching-the-issue}{%
\section*{Patching the issue}\label{patching-the-issue}}

The only line required to change is in \texttt{rbv.R}, and is the fifth
line. Simply change it from

\begin{verbatim}
if(attr(pedigree, "class")=="phylo"){ped=FALSE}
\end{verbatim}

and change it to:

\begin{verbatim}
if(any(attr(pedigree, "class")=="phylo)){ped=FALSE}
\end{verbatim}

Then, rebuild the modified package. A patched tarball file is available
on Github
\href{https://github.com/frederni/TMA4900-MasterThesis/blob/main/MCMCglmm-rbv-patch.tar.gz}{here}.
