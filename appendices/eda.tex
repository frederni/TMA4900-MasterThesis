\hypertarget{data-frame-overview}{%
\section*{Data frame overview}\label{data-frame-overview}}

\hypertarget{pedigree}{%
\subsection*{Pedigree}\label{pedigree}}

The first object is \texttt{d.ped} which contains the pedigree
information.

\begin{Shaded}
\begin{Highlighting}[]
\FunctionTok{summary}\NormalTok{(d.ped)}
\end{Highlighting}
\end{Shaded}

\begin{verbatim}
##     ninecode             gendam             gensire         
##  Min.   :109137448   Min.   :109137468   Min.   :109137448  
##  1st Qu.:146164012   1st Qu.:146130794   1st Qu.:146130313  
##  Median :176124850   Median :176124382   Median :176124004  
##  Mean   :196520240   Mean   :188116000   Mean   :185463038  
##  3rd Qu.:243185045   3rd Qu.:226189260   3rd Qu.:226189228  
##  Max.   :999999999   Max.   :999999999   Max.   :266176829  
##                      NA's   :59          NA's   :59
\end{verbatim}

It has columns \emph{ninecode}, \emph{gendam}, and \emph{gensire}.
The first column cannot be \texttt{NA} and is the unique identifier for
an individual, whereas \texttt{gendam} and \texttt{gensire} are
references (foreign keys) to the known maternal and paternal link,
respectively. Both of these columns have 59 NAs. In fact, these NAs
overlap completely since they are the founder population with no defined
paternal or maternal link:

\begin{Shaded}
\begin{Highlighting}[]
\NormalTok{d.ped[}\FunctionTok{is.na}\NormalTok{(d.ped}\SpecialCharTok{$}\NormalTok{gendam), }\StringTok{"gensire"}\NormalTok{]}
\end{Highlighting}
\end{Shaded}

\begin{verbatim}
##  [1] NA NA NA NA NA NA NA NA NA NA NA NA NA NA NA NA NA NA NA NA NA NA NA NA NA
## [26] NA NA NA NA NA NA NA NA NA NA NA NA NA NA NA NA NA NA NA NA NA NA NA NA NA
## [51] NA NA NA NA NA NA NA NA NA
\end{verbatim}

We see that \emph{gensire} is NA for all instances where \emph{gendam}
is also NA. This is the founder population with no defined parental
linkage.

\hypertarget{d.q}{%
\subsection*{d.Q}\label{d.q}}

This table has the columns \emph{g1}, \emph{foc0} and \emph{ninecode}
(ID).

\begin{Shaded}
\begin{Highlighting}[]
\FunctionTok{head}\NormalTok{(d.Q)}
\end{Highlighting}
\end{Shaded}

\begin{verbatim}
##   foc0 g1  ninecode
## 1    1  0 109137407
## 2    1  0 109137408
## 3    1  0 109137418
## 4    1  0 109137420
## 5    1  0 109137421
## 6    1  0 109137425
\end{verbatim}

Considering only the first results, it might seem like \texttt{foc0} and
\texttt{g1} are binary/categorical variables, but plotting the values
across indices show that the order of the rows are structured so that
they start at 1 and 0 respectively.

\begin{Shaded}
\begin{Highlighting}[]
\FunctionTok{par}\NormalTok{(}\AttributeTok{mfrow =} \FunctionTok{c}\NormalTok{(}\DecValTok{1}\NormalTok{, }\DecValTok{2}\NormalTok{))}
\FunctionTok{plot}\NormalTok{(d.Q}\SpecialCharTok{$}\NormalTok{g1, }\AttributeTok{main =} \StringTok{"g1"}\NormalTok{)}
\FunctionTok{plot}\NormalTok{(d.Q}\SpecialCharTok{$}\NormalTok{foc0, }\AttributeTok{main =} \StringTok{"foc0"}\NormalTok{)}
\end{Highlighting}
\end{Shaded}

\includegraphics{EDA_files/figure-latex/unnamed-chunk-2-1.pdf}

We can also look at the correlation between these two values

\begin{Shaded}
\begin{Highlighting}[]
\FunctionTok{cor}\NormalTok{(d.Q}\SpecialCharTok{$}\NormalTok{foc0, d.Q}\SpecialCharTok{$}\NormalTok{g1)}
\end{Highlighting}
\end{Shaded}

\begin{verbatim}
## [1] -1
\end{verbatim}

Hence, we have a very strong negative correlation here. We can also look
at the individuals whose ID were in the \emph{founder population}:

\begin{Shaded}
\begin{Highlighting}[]
\NormalTok{founder\_population.id }\OtherTok{\textless{}{-}}\NormalTok{ d.ped[}\FunctionTok{is.na}\NormalTok{(d.ped}\SpecialCharTok{$}\NormalTok{gendam), }\StringTok{"ninecode"}\NormalTok{]}
\FunctionTok{table}\NormalTok{(d.Q[}\FunctionTok{which}\NormalTok{(d.Q}\SpecialCharTok{$}\NormalTok{ninecode }\SpecialCharTok{\%in\%}\NormalTok{ founder\_population.id),}
          \FunctionTok{c}\NormalTok{(}\StringTok{"foc0"}\NormalTok{, }\StringTok{"g1"}\NormalTok{)])}
\end{Highlighting}
\end{Shaded}

\begin{verbatim}
##     g1
## foc0  0  1
##    0  0 33
##    1 26  0
\end{verbatim}

The values seem to be relatively balanced between \(0\) and \(1\) in the
founder population. This supports the idea that they measure the
immigration contribution to the genetic composition of the individuals.
All immigrant individuals are completely immigrant, have no pedigree and
are thus part of the founder population. The latter are those who are
the ``initial'' natives on the island, meaning that their values must be
exactly zero.

\hypertarget{ped.prune}{%
\subsection*{ped.prune}\label{ped.prune}}

This is a pruned pedigree, only considering the 1993-2018 observations
but also combining the knowledge of the 1975-1992 observations into
them.

\hypertarget{qg.data.gg.ind}{%
\subsection*{qg.data.gg.ind}\label{qg.data.gg.ind}}

This object has the following shape:

\begin{Shaded}
\begin{Highlighting}[]
\FunctionTok{head}\NormalTok{(qg.data.gg.inds)}
\end{Highlighting}
\end{Shaded}

\begin{verbatim}
##    ninecode natalyr sex.use nestrec surv.ind.to.ad brood.date sex.use.x1
## 1 111111112    2012       0    3086              0        120          1
## 2 111111121    2015       0    3237              0        141          1
## 3 143173366    1993       1    1838              1         96          1
## 4 143173381    1993       2    1867              1        102          2
## 5 143173382    1993       1    1867              0        102          1
## 6 143173384    1993       1    1851              0        102          1
##       f.coef      foc0        g1 natalyr.no sex
## 1 0.11155218 0.4085679 0.5914321         38   0
## 2 0.04814660 0.3299752 0.6700248         41   0
## 3 0.05108643 0.5283203 0.4716797         19   0
## 4 0.03125000 0.6250000 0.3750000         19   1
## 5 0.03417969 0.4335938 0.5664062         19   0
## 6 0.02148438 0.6328125 0.3671875         19   0
\end{verbatim}

The response variable we will use is \texttt{surv.ind.to.ad}. Below are
some elementary properties of the data.

\begin{verbatim}
## [1] "Earliest year: 1993"
\end{verbatim}

\begin{verbatim}
## [1] "Number not survived: 1817" "Number survived: 661"
\end{verbatim}

\begin{verbatim}
## [1] "natal year correlation: 1"
\end{verbatim}

\begin{verbatim}
## [1] "correlation between sex and sex.x1: 0.842997540673555"
\end{verbatim}

An overview of the columns:

\begin{itemize}
\tightlist
\item
  \emph{ninecode}: Individual ID
\item
  \emph{natalyr}: The year the individual was born, e.g.~2015.
\item
  \emph{sex.use}: \textbf{Not in use}
\item
  \emph{nestrec}: ID for nest number
\item
  \emph{brood.date}: Day of the year when the first offspring in
  individuals nest hatched
\item
  \emph{sex.use.x1}: Sex of individual, 1 or 2
\item
  \emph{f.coef}: Inbreeding coefficient
\item
  \emph{foc0}: ``How foreign'' individual is, related to \texttt{f.coef}
\item
  \emph{g1}: Inverse of \emph{foc0}.
\item
  \emph{natalyr.no}: The same as the natal year, starting with 1974 as 0
  (2015=41).
\end{itemize}

\hypertarget{vizualization-of-juvenile-survival}{%
\section*{Vizualization of juvenile
survival}\label{vizualization-of-juvenile-survival}}

We will have a look at how the response, juvenile survival, relates to
the other covariates in our data.

First, we look at sex:

\begin{verbatim}
## Warning: package 'ggplot2' was built under R version 4.2.2
\end{verbatim}

\includegraphics{EDA_files/figure-latex/unnamed-chunk-7-1.pdf}

It seems like the sex in relation to survival is relatively balanced
here. We can note that it seems like a larger portion of those surviving
are females. Next, we examine the breeding coefficient.

\includegraphics{EDA_files/figure-latex/unnamed-chunk-8-1.pdf}

Here we see that survival is a bit more skewed toward lower inbreeding
coefficients. We may also plot the proportion of individuals who survived
over each year:

\begin{verbatim}
## `geom_smooth()` using method = 'loess' and formula = 'y ~ x'
\end{verbatim}

\includegraphics{EDA_files/figure-latex/unnamed-chunk-9-1.pdf}

There seems to be very little trending over the years, but possibly a
small negative trend. We also examine if there is some correspondence
between genetic group coefficient (\texttt{g1}) and juvenile survival.

\includegraphics{EDA_files/figure-latex/unnamed-chunk-10-1.pdf}

This shows a similar result to the inbreeding coefficient, namely a skew
towards the right (lower values of coefficient) in the group that
survived. Finally, we plot the survival probability based on brood date:

\begin{verbatim}
## `geom_smooth()` using method = 'loess' and formula = 'y ~ x'
\end{verbatim}

\includegraphics{EDA_files/figure-latex/unnamed-chunk-11-1.pdf}

This last plot seems to indicate that survival is relatively stable and
somewhat decreasing for those hatched relatively late. For the largest
values of brood date, we get an increasing trend but also much
uncertainty since not that many were hatched this late.